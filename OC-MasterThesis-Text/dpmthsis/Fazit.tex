\chapter{Fazit aus Ausblick}
\label{Fazit}

Da das klassische GP bereits in verschiedenen Domänen für Problemlösungen verwendet wird, ist es sinnvoll den Bereich des CGP näher zu betrachten.
Laut Miller werde der Rekombinationsschritt in CGPs normalerweise nicht ausgeführt, da dieser die Effizienz eines CGP-Modells nicht zielführend steigert \cite{miller_cartesian_2020}.
Trotzdem wäre es sinnvoll neben der Mutation einen weiteren genetischen Operator einzubinden, um die Effizienz des CGPs zu steigern und somit komplexere Ausgangsprobleme lösbar zu machen.
Die Aussagen, dass Rekombinationsschritte bisher nicht sinnvoll in das Standard-CGP eingepflegt werden können basiert auf Papern von Miller aus den Jahren 1999 und 2011.
Diese Aussage wird in dieser Arbeit erneut überprüft.
Da der Erfolg der Rekombination stark von der Wahl der Rekombinationsrate abhängt, werden in dieser Arbeit ebenfalls unterschiedliche Arten für Rekombinationsraten betrachtet und bewertet.
Für diese Arbeit ergeben sich demnach die folgenden zwei Forschungsfragen:\\
1. Kann mit heutigem Forschungsstand nachgewiesen werden, dass Rekombination in Standard-CGP sinnvoll eingesetzt werden kann im Vergleich zu CGP ohne Rekombinationsschritt?\\
2. Hat die Art und Weise der Rekombinationsratenberechnung eine Auswirkung auf die Effektivität des CGPs?

Für die Beantwortung der Forschungsfragen wurden vier verschiedene Testprobleme aufgebaut, die durch CGP-Modelle mit verschiedenen Konfigurationen gelöst werden sollten.
Die verschiedenen CGP-Konfigurationen beinhalteten unterschiedliche Rekombinationsarten Var,ianten der Rekombinationsratenberechnung, sowie die völlige Abwesenheit des Rekombinationsschritts.
Bei zwei der vier Testszenarien wurde eine Hyperparameteranalyse durchlaufen, um die Effizienz der besten Ergebnisse für jede CGP-Konfiguration miteinander zu vergleichen.
Für die anderen zwei Testfälle wurden die Rekombiantionsraten-Arten facettenreich parametriert, um bewerten zu können, wie sich die Parametrierung der Rekombinationsraten auf die Effizienz des CGPs auswirken.\\
Für die Analyse der Ergebnisse wurden zwei unterschiedliche Verfahren angewendet.
Einerseits wurden die aufgezeichneten Rohdaten der Testdurchläufe näher analysiert, andererseits wurde eine Bewertung mit Hilfe der bayes'schen Analyse ausgeführt.\\
TODO: Ergebnisse Arbeit hier einfügen + dazugehörige offene Fragen (danach dann bewertet, ob offene Fragen weiter runter kommen sollen oder hier oben Sinn machen)

Zusammenfassend lassen sich die zwei Forschungsfragen wie folgt beantworten:\\
In Standard-CGPs kann Rekombination durchaus sinnvoll eingesetzt werden.
Dafür ist es allerdings relevant die richtige Parametrierung mit Hilfe einer Hyperparameteranalyse zu ermitteln.
Teil dieser Analyse sollten ebenfalls die Wahl der richtigen Rekombinationsart und Rekombinationsraten-Art sein.
Weiter wurde herausgefunden, dass die Rekombination ein größeres Potential für die Verbesserung der Effizienz des CGPs aufweist, das im Standard-CGP-Verfahren nicht vollständig ausgeschöpft wird.
Hierzu könnten zukünftig weitere Tests ausgeführt werden, um ein geeigneteres CGP-Verfahren zu entwickeln, das Rekombination und Mutation besser vereinen kann.\\
Des weiteren hat die Art und Weise der Rekombinationsratenberechnung einen großen Einfluss auf die Effizienz des CGPs. 
Hohe Rekombinationsraten zu Beginn des Trainings und ein anschließender Abbruch der Rekombination erscheinen besonders sinnvoll.
Zu welcher Iteration die Rekombination im besten Fall ausgesetzt werden soll und ob dies schleichend oder abrupt stattfinden sollte, könnte durch weitere Forschung näher betrachtet werden.

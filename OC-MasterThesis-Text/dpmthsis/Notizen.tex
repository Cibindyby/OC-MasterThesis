\chapter{Notizen}


Notizen aus Zettel:
\begin{itemize}
    \item mehrere Keijzer-6 Formeln vorhanden erwähnen -> aber aus "Better GP Benchmarks..." verwenden
    \item Tournament-Selektion: Größere Turniere erhöhen den Selektionsdruck, da fittere Individuen häufiger gewinnen.
    Kleinere Turniere erhalten mehr Diversität, da auch weniger fitte Individuen eine Chance haben
    => aus Perplexity: "Genetic Algorithm, Tournament Selection, ..." von Miller
    \item Probleme zwischen Rust und Julia: Problem mit Array-Zählung ab 1 bei Julia
    \item Probleme wenn x=zu groß in sin(x) oder cos(x) => kopieren der Eingabe
    \item one-fifth rule: Es gibt mehrere Möglichkeiten das anzupassen; Dabei gibt es viele Gedanken zu beachten
    -> 1. Möglichkeit: Fitness über zb 50 Generationen beobachten und dann Rate anpassen (haben mindestens 20 Prozent der Rekombinationen eine bessere Fitness gebracht?)
    => Problem: Mutation wird damit auch mitbewertet
    => Idee: nach jedem Rekombinationsschritt (vor dem Mutationsschritt) Eltern und Kinder vergleichen (sind mind. 20 Prozent der Kinder besser als beide Eltern?)
    \item HPO Problem: 
    -> Random-Sampler = HPO wird in 4 Tagen nicht fertig, weil einzelne Runden zu lange brauchen
    -> EvalAfterIteration runter drehen (Aber nicht zu weit runter, damit Ergebis "wird nicht fertig" krass bewertet wird) => an Hennings Ergebnissen abgleichen => kann für schwerere Szenarien nicht genug runter gedreht werden
    -> eventuell muss neuer Sampler her: BHOB mit und ohne Hyperband ausprobiert => keine sinnvollen HPO-Ergebnisse und HPO wird nicht fertig
    -> Lösung: Nur Crossoverrate und Offset neu bewerten in HPO mit RandomSampler; Rest Ergebnisse von Henning nehmen
    \item no crossover braucht keine HPO mehr, weil keine Crossoverrate und kein Offset benutzt wird
    \item no crossover hpo von Henning hat immer zwei mu-lambda Werte -> besseren nehmen
    \item innerhalb von HPO bewerten, wie sich die fitness ändert in Crossover-Verlauf
\end{itemize}


\chapter{Praktischer Teil}
\label{praktischer Teil}

\section{Aufbau der Experimente}
\label{sec:aufbauExperimente}

Für die Beantwortung der Forschungsfragen werden unterschiedliche Experimente ausgeführt, deren Ergebnisse in dieser Arbeit ausgewertet und interpretiert werden sollen.
Der Programmcode für die Experimente wurde in der Sprache Julia verfasst.
Dabei fand eine Orientierung an folgendem Code von Henning Cui statt: \cite{cuihen_cuihencgp_with_crossover_strategies_2024}\\
Um Ergebnisse von Ausgangsproblemen aus unterschiedlichen Domänen für eine umfangreichere Bewertung zur Verfügung zu stellen, werden mehrere Benchmark-Testszenarien überprüft.
Diese werden in den folgenden Abschnitten näher betrachtet.

\subsection{Testszenarien}
\label{subsec:testszenarien}

\subsubsection{Boolean Problems}
\label{subsubsec:booleanProblems}

TODO: Notizen zu Boolean Benchmark Problems ausformulieren\\
Notizen:
\cite{kalkreuth_towards_2023}\\
Logic synthesis (LS) in GP can be considered a black-box and optimization problem domain that has played a major role in the application scope of GP research throughout its history. In general, LS by means of GP refers to the application of GP models to synthesize expression that match the input-output mapping of Boolean functions. Boolean functions can be formally expressed and mathematically described with Boolean expressions. LS as tackled with GP paradigm predominantly addresses two major tasks in this problem domain: (1) Synthesis of a Boolean expression that produces the correct output given the inputs of the Boolean function. (2) OptimizationoftheBooleanexpressionthatrepresentsacertain Boolean function. The latter task is approached by defining one or more optimization objectives. Both tasks are performed in accordance with Boolean logic and algebra. Besides, algebraic expressions, Boolean functions are commonly represented with truth tables that describe the inputoutput mapping of the respective function


Die vier verwendeten Boolean Benchmark Probleme für die Evaluation der verschiedenen Rekombinationsalgorithmen werden aus dem Paper von Cui et al. herangezogen.
Die Aussagen in diesem Abschnitt geziehen sich demnach ebenfalls auf dieses Paper.\cite{cui_equidistant_2023}\\
TODO: hier die Argumente für die Auswahl dieser vier BBP treffen\\

TODO: Die vier Probleme näher beschreiben: Inputs, Outputs und Berechnung

\subsubsection{Symbolische Regression}
\label{subsubsec:symbolicRegression}

\section{Ergebnisse}
\label{Ergebnisse}

\begin{itemize}
    \item Versuchsaufbau und Durchführung
    \begin{itemize}
        \item an Hennings Code orientiert
        \item Testszenarien -> welche Testdaten und Berechnung fitness
        \begin{itemize}
            \item boolean Problems
            \item symbolic regression
            \item ggf. Ameisenoptimierungsproblem
        \end{itemize}
        \item weitere Einstellungen (Welche Selektionsverfahren, Mutationsverfahren, Rekomb...)
        \item Hyperparameteroptimierung
        \begin{itemize}
        	\item HPO wird in Grundlagen nicht erklärt also hier kurz erwähnen, dass die besten Parameter gesucht werden also eine HPO ausgeführt wird
            \item Welche Bib benutzt
            \item Welche Parameter optimiert; mit welchen Ranges
            \item 10 Ausführungen pro Optimierungsschritt
            \item Probleme Rechenzeit HPO
            \item verschiedene Sampler keinen Mehrwert
        \end{itemize}
        \item Ausführen mehrerer Durchläufe, um statistische Auswertung vorzunehmen
    \end{itemize}
    \item Ergebnisse und Evaluation
    \begin{itemize}
        \item Metriken zur Auswertung
        \item händische Auswertung der Ergebnisse, eventuell auch durch Median, um zu sehen in welchen Fällen es mehr Streuung gibt?
        \item bayesian Auswertung der Ergebnisse
        \item Bewertung der statistischen Aussagekraft mit ANOVA
    \end{itemize}
\end{itemize}
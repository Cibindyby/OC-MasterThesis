\chapter{Ergebnisse}
\label{Ergebnisse}

In den folgenden Abschnitten werden die Ergebnisse dieser Arbeit aufgezeigt und erklärt.
Sie sind unterteilt in die drei Evaluations-Strategien, die in Abschnitt \ref{sec:Evaluation} erläutert wurden: Analyse der Rohdaten, bayes'sche Analyse und graphische Evaluation.
Zur Vereinfachung wird die konstante Rekombinationsrate kurz \glqq Konstant\grqq\space genannt, die linear fallende Rekombinationsrate \glqq Clegg \grqq\space und die Rekombinationsrate mit der One-Fifth Regel \glqq One-Fifth\grqq.

\section{Ergebnisse Rohdatenanalyse}
\label{sec:ergebnisseRohdaten}

Die Analyse der Rohdaten besteht aus zwei Teilen.
Die Ergebnisse der Hyperparameteranalyse werden für Parity und Keijzer näher betrachtet.
Es werden dabei folgende Werte für jede CGP-Konfiguration aufgezeigt:\\
\textbf{Anzahl der Rechenknoten:} Anzahl der Rechenknoten in jedem Chromosom des CGP-Modells\\
\textbf{$\lambda$ (Nachwuchs):} Anzahl der Nachwuchs-Chromosomen in jeder Generation; $\lambda$-Wert in der ($\mu$+$\lambda$)-ES\\
\textbf{Start-Rekombinationsrate:} Rekombinationsrate bei der Initialisierung; nur relevant für die konstante Rekombinationsrate und die One-Fifth Regel (linear fallende Rekombinationsrate wird mit 0,9 initialisiert)\\
\textbf{Delta Rekombinationsrate:} Dieser Wert wird bei linear fallender Rekombinationsrate in jeder Generation von der bisherigen Rekombinationsrate abgezogen.\\
\textbf{$\mu$ (Elitisten):} Anzahl der Elitisten in jeder Generation; $\mu$-Wert in der ($\mu$+$\lambda$)-ES\\
\textbf{Offset:} Gibt an in wie vielen Trainings-Iterationen nach Initialisierung des CGP-Modells keine Rekombination ausgeführt wird.\\

Des Weiteren werden für alle Testszenarien die Ergebnisse des CGP-Trainings näher betrachtet.
Dabei werden die folgenden Metriken für jede CGP-Konfiguration in den Tabellen aufgezeigt:\\
\textbf{Anzahl positive Mutationen:} Gibt an wie häufig Mutation zu einer Verbesserung der Fitness geführt hat (summiert über alle 50 Durchläufe).\\
\textbf{Anzahl positive Rekombination:} Anzahl der Rekombinationsschritte, die die Fitness verbessert haben (summiert über alle 50 Durchläufe).\\
\textbf{Anzahl negative Mutationen:} Anzahl der Mutationsschritte, bei denen der vorherige Rekombinationsschritt eine Verbesserung der Fitness erzielt hat, die nach der Mutation verloren ging (summiert über alle 50 Durchläufe).\\
\textbf{Median Iterationen positiver Rekombination:} Median der Iterationen, bei denen der Rekombinationsschritt zur Verbesserung der Fitness beigetragen hat (über 50 Durchgänge hinweg)\\
\textbf{Median Iterationen bis Konvergenz:} Median der Iterationen bis das Stopp-Kriterium erfüllt wurde (über 50 Durchläufe hinweg)\\
\textbf{Stopp-Kriterium erfüllt:} Gibt an wie häufig das Stopp-Kriterium bei 50 Durchläufen erfüllt wurde. Dieser Wert wird verwendet, um Entscheidungen darüber zu treffen, wie die bayes'sche Analyse ausgeführt werden soll (für die komplexeren Testszenarien).

\subsection{Rohdatenanalyse: Parity}
\label{subsec:rohdatenParity}

Die folgende Tabelle \ref{table:parityHPO} zeigt die Hyperparameter, die für die Ausführung des CGP-Trainings für Parity verwendet wurden.
Dabei wurden die Ergebnisse der Hyperparameteranalyse verwendet, um die effizientesten CGP-Konfigurationen miteinander zu vergleichen.
Zu beachten ist, dass bei der Optimierung des Rekombinations-Offsets dem Optimierer ebenfalls die Möglichkeit gegeben wurde den Offset auf 0 zu stellen und somit auszuschalten.
Diese Möglichkeit wurde bei zwei aus neun CGP-Konfigurationen genutzt.
Für diese Fälle aus der Hyperparameteroptimierung wurden die nächstbesten Hyperparameter gewählt, die einen Offset enthalten.
Die jeweiligen Offset-Werte sind in der Tabelle rot markiert.
Alle Beobachtungen in diesem Abschnitt beziehen sich ausschließlich auf das Parity-Testszenario.
Demnach kann nicht grundsätzlich auf die Allgemeinheit geschlossen werden und alle Aussagen müssen kritisch begutachtet werden.

\begin{table}[H]
	\centering
	\begin{tabular}{c | c | c | c | c | c | c}
		\begin{turn}{270} \textbf{CGP-Konfigurationen} \end{turn} & \begin{turn}{270} \textbf{Anzahl Rechenknoten} \end{turn} & \begin{turn}{270} \textbf{$\lambda$ (Nachwuchs)} \end{turn} & \begin{turn}{270} \textbf{Start-Rekombinationsrate} \end{turn} & \begin{turn}{270} \textbf{Delta Rekombinationsrate} \end{turn} & \begin{turn}{270} \textbf{$\mu$ (Elitisten)} \end{turn} & \begin{turn}{270} \textbf{Offset} \end{turn}\\
		\hline
		keine Rekombination & 1000 & 20 & - & - & 18 & -\\
		\hline
		One-Point Konstant kein Offset & 1350 & 52 & 0,30 & - & 16 & - \\
		\hline
		One-Point Konstant mit Offset & 1500 & 50 & 0,8 & - & 18 & 25 \\
		\hline
		One-Point Clegg kein Offset & 1950 & 42 & - & 0,035 & 14 & - \\
		\hline
		One-Point Clegg mit Offset & 1600 & 58 & - & 0,01 & 20 & 25 \\
		\hline
		One-Point One-Fifth kein Offset & 1150 & 46 & 0,55 & - & 6 & - \\
		\hline
		One-Point One-Fifth mit Offset & 1100 & 54 & 0,35 & - & 20 & \color{red}30\color{black} \\
		\hline
		Two-Point Konstant kein Offset & 1900 & 58 & 0,4 & - & 20 & - \\
		\hline
		Two-Point Konstant mit Offset & 850 & 54 & 0,3 & - & 16 & 15 \\
		\hline
		Two-Point Clegg kein Offset & 750 & 56 & - & 0,02 & 18 & - \\
		\hline
		Two-Point Clegg mit Offset & 1700 & 28 & - & 0,04 & 12 & \color{red}20\color{black} \\
		\hline
		Two-Point One-Fifth kein Offset & 1800 & 44 & 0,55 & - & 16 & - \\
		\hline
		Two-Point One-Fifth mit Offset & 950 & 58 & 0,7 & - & 10 & 15 \\
		\hline
		Uniform Konstant kein Offset & 1800 & 58 & 0,8 & - & 20 & - \\
		\hline
		Uniform Konstant mit Offset & 1400 & 54 & 0,3 & - & 20 & 45 \\
		\hline
		Uniform Clegg kein Offset & 1600 & 50 & - & 0,015 & 20 & - \\
		\hline
		Uniform Clegg mit Offset & 1500 & 52 & - & 0,04 & 18 & 25 \\
		\hline
		Uniform One-Fifth kein Offset & 650 & 52 & 0,75 & - & 14 & - \\
		\hline
		Uniform One-Fifth mit Offset & 250 & 50 & 0,35 & - & 16 & 40 \\
	\end{tabular}
	\caption{Parity: Ergebnis Hyperparameteranalyse}
	\label{table:parityHPO}
\end{table}

Betrachtet wird zuerst die Anzahl der Rechenknoten. 
Zu beobachten ist hier, dass die CGP-Konfiguration ohne Rekombination weniger Rechenknoten benötigt als der Mittelwert aller Verfahren von ca. 1305 Rechenknoten pro Chromosom.
Die Modelle, die die One-Point Rekombination verwenden, weisen mit ca 1442 den höchsten Mittelwert an Rechenknoten auf.
Mit durchschnittlich 1350 Rechenknoten haben die Modelle mit Two-Point Rekombination ein wenig mehr Rechenknoten benötigt als der Durchschnitt.
Mit 1200 Rechenknoten im Mittel haben Modelle, die die Uniform Rekombination verwenden weniger Rechenknoten als der Durchschnitt benötigt.
Da eine höhere Anzahl der Rechenknoten eine höhere Rechenzeit mit sich zieht, ist es von Vorteil diese reduzieren zu können.
Unter dem Aspekt lässt sich in der Betrachtung der Hyperparameteranalyse von Parity ein Trend schätzen: wird keine Rekombination verwendet könnten Rechenzeit und Speicherressourcen gespart werden, indem weniger Rechenknoten benötigt werden. 
Falls Rekombination verwendet wird, könnte es sich lohnen die aufwendigeren Algorithmen zu verwenden, da es so aussieht als ob diese weniger Rechenknoten benötigen.
Diese Aussagen können allerdings nicht getroffen werden, indem nur ein einzelnes Testszenario ausgewertet wird, gibt allerdings Denkanstöße für weitere Betrachtungen.\\
Des Weiteren fällt auf, dass die Streuung der Anzahl der Rechenknoten immer größer zu werden scheint, umso komplexer das Rekombinationsverfahren gewählt wird. 
Während bei One-Point Rekombination alle Werte relativ nah am Mittelwert liegen, entsteht bei Uniform Rekombination eine große Kluft zwischen extrem großen und sehr kleinen Werten. 
So könnte es sich im Hinblick auf die Ressourcen lohnen Rekombination einzusetzen, wenn durch eine gute Hyperparameteranalyse festgestellt werden kann, mit welcher CPG-Konfiguration das Einsparpotential am besten ausgeschöpft werden könnte.
Zu beachten ist hierbei allerdings auch, dass komplexere Rekombinationsalgorithmen (und Rekombination im allgemeinen) grundsätzlich mehr Rechenzeit benötigen als wenn dieser Rekombinationsschritt ausgelassen wird.
Diese Metriken müssten sinnvoll miteinander abgeglichen werden.\\
Wird die Populationsgröße ($\mu$ + $\lambda$) näher betrachtet, fällt auf, dass auch hier das Verfahren ohne Rekombination eine deutlich kleinere Population benötigt als die meisten anderen.
Auch dieser Faktor spart Ressourcen ein und kann für die Entscheidung für oder gegen ein Verfahren eine Rolle spielen.
Die CGP-Konfiguration "Two-Point Clegg mit Offset hat ebenfalls eine sehr niedrige Populationsgröße erreicht, allerdings handelt es sich bei dieser Zeile der Tabelle nicht um die beste Parametrierung der Konfiguration.
Dies ist der Fall, da in dieser Zeile wie vorher beschrieben der Offset auf 0 gesetzt worden wäre, weshalb nur der zweitbeste Parametersatz gewählt wurde.
Der beste Parametersatz hätte die Werte $\lambda$=44 und $\mu$=12, was wiederum eine größere Populationsgröße ergeben würde.\\
Bei der Begutachtung der verschiedenen Rekombinationsraten (Start-Rekombinationsrate und Delta Rekombinationsrate) können keine genauen Zusammenhänge zwischen den CGP-Konfigurationen und der Höhe der Rekombinationsraten erkannt werden.
Für die Start-Rekombinationsrate kann allerdings bei allen Verfahren mit konstanter Rekombinationsrate beobachtet werden, dass extrem hohe oder sehr niedrige Rekombinationsraten vermieden werden.
Für die One-Fifth Regel wurde genau dieser mittelhohe Bereich zwischen 0 und 1 für die Hyperparameteranalyse freigegeben.
Dieser Bereich wird nahezu vollständig in den Ergebnissen ausgenutzt.
In weiteren Tests könnte überprüft werden, ob die Bereiche für den Start-Wert der Rekombinationsrate sinnvoll gewählt wurden oder ob die Hyperparameteranalyse für diese Fälle extremere Werte bevorzugen würde.
Auch bei der linear fallenden Rekombinationsrate wurde nahezu der gesamte Definitionsbereich für die Hyperparameteranalyse ausgenutzt.\\
Zuletzt wird die Offset-Spalte betrachtet.
Der Mittelwert der gewählten Offsets liegt bei ca. 27 Iterationen.
Am geringsten fällt der Mittelwert für die Two-Point Rekombination aus mit einem Durchschnittswert von ca. 17.
Betrachtet man nicht die nachbearbeiteten Zeilen der Hyperparameteranalyse, sondern die originalen Ergebnisse, ist der Mittelwert sogar noch kleiner mit 10 Iterationen ohne Rekombination.
Im Vergleich dazu liegt die Uniform Rekombination mit durchschnittlich ca. 37 Iterationen ohne Rekombination weit darüber.
Das bringt die Frage auf, ob die Uniform Rekombination weniger Effizienz bietet als beispielsweise die Two-Point Rekombination, da mehr Rekombinationsschritte ausgesetzt werden.
Die zwei neu eingepflegten Offset-Werte (rot) entsprechen ungefähr dem allgemeinen Mittelwert.\\


Die folgende Tabelle \ref{table:parityRohdaten} gibt einen Einblick in die Rohdaten der CGP-Trainings.
Alle CGP-Konfigurationen wurden dabei 50 mal am Parity-Datensatz getestet.
Die Tabelle zeigt einige wichtige Merkmale auf, die das CGP-Training ergeben hat.


\begin{table}[H]
	\centering
	\begin{tabular}{c | c | c | c | c | c }
		\begin{turn}{270} \textbf{CGP-Konfigurationen} \end{turn} & \begin{turn}{270} \textbf{Anzahl pos. Mutationen} \end{turn} & \begin{turn}{270} \textbf{Anzahl pos. Rekomb.} \end{turn} & \begin{turn}{270} \textbf{Anzahl neg. Mutationen} \end{turn} & \begin{turn}{270} \textbf{Median Iter. pos. Rekomb.} \end{turn} & \begin{turn}{270} \textbf{Median Iter. bis Konv.} \end{turn}\\
		\hline
		keine Rekombination & 133 & 0 & 0 & - & 58,5\\
		\hline
		One-Point Konstant kein Offset & 126 & 8 & 2 & 5,5 & 44,0\\
		\hline
		One-Point Konstant mit Offset & 136 & 0 & 0 & - & 49\\
		\hline
		One-Point Clegg kein Offset & 123 & 13 & 3 & 4 & 74\\
		\hline
		One-Point Clegg mit Offset & 126 & 0 & 0 & - & 43,5\\
		\hline
		One-Point One-Fifth kein Offset & 119 & 2 & 1 & 2,0 & 34\\
		\hline
		One-Point One-Fifth mit Offset & 127 & 0 & 0 & - & 27,5\\
		\hline
		Two-Point Konstant kein Offset & 113 & 10 & 2 & 3,0 & 39,5\\
		\hline
		Two-Point Konstant mit Offset & 129 & 0 & 0 & - & 64\\
		\hline
		Two-Point Clegg kein Offset & 106 & 20 & 8 & 6,5 & 40,0\\
		\hline
		Two-Point Clegg mit Offset & 125 & 0 & 0 & - & 70,5\\
		\hline
		Two-Point One-Fifth kein Offset & 126 & 4 & 2 & 3,0 & 59,5\\
		\hline
		Two-Point One-Fifth mit Offset & 125 & 0 & 0 & - & 43\\
		\hline
		Uniform Konstant kein Offset & 85 & 44 & 15 & 14,0 & 55\\
		\hline
		Uniform Konstant mit Offset & 131 & 0 & 0 & - & 37,5\\
		\hline
		Uniform Clegg kein Offset & 101 & 41 & 15 & 4 & 31\\
		\hline
		Uniform Clegg mit Offset & 130 & 0 & 0 & - & 51,5\\
		\hline
		Uniform One-Fifth kein Offset & 108 & 22 & 12 & 5,0 & 69,5\\
		\hline
		Uniform One-Fifth mit Offset & 123 & 0 & 0 & - & 54,0\\
	\end{tabular}
	\caption{Parity: Auswertung der Rohdaten}
	\label{table:parityRohdaten}
\end{table}

\subsection{Rohdatenanalyse: Keijzer}
\label{subsec:rohdatenKeijzer}

Die folgende Tabelle \ref{table:keijzerHPO} zeigt die Ergebnisse der Hyperparameteranalyse des Keijzer-Testszenarios.
Die Umstände der Hyperparameteroptimierung sind die gleichen wie bei Parity und können in Abschnitt \ref{subsec:rohdatenParity} nachgelesen werden.
Wie bei den Ergebnissen bei Parity gibt es auch bei der Offset-Optimierung im Keijzer-Testszenario zwei aus neun Fällen, bei denen der Offset ursprünglich auf 0 gesetzt wurde.
Diese Parametersätze wurden durch den jeweils zweitbesten Parametersatz ersetzt und sind in der Tabelle rot markiert.
Die Beobachtungen in diesem Abschnitt beziehen sich nur auf den Keijzer-Benchmark.
In Abschnitt \ref{subsec:rohdatenZusammenfassung} werden abschließende Erkenntnisse aus den Hyperparameteranalysen von Parity und Keijzer zusammengetragen.

\begin{table}[H]
	\centering
	\begin{tabular}{c | c | c | c | c | c | c}
		\begin{turn}{270} \textbf{CGP-Konfigurationen} \end{turn} & \begin{turn}{270} \textbf{Anzahl Rechenknoten} \end{turn} & \begin{turn}{270} \textbf{$\lambda$ (Nachwuchs)} \end{turn} & \begin{turn}{270} \textbf{Start-Rekombinationsrate} \end{turn} & \begin{turn}{270} \textbf{Delta Rekombinationsrate} \end{turn} & \begin{turn}{270} \textbf{$\mu$ (Elitisten)} \end{turn} & \begin{turn}{270} \textbf{Offset} \end{turn}\\
		\hline
		keine Rekombination & 1850 & 44 & - & - & 20 & -\\
		\hline
		One-Point Konstant kein Offset & 600 & 50 & 0,2 & - & 20 & - \\
		\hline
		One-Point Konstant mit Offset & 1200 & 50 & 1 & - & 4 & 120\\
		\hline
		One-Point Clegg kein Offset & 1100 & 60 & - & 0,05 & 16 & - \\
		\hline
		One-Point Clegg mit Offset & 850 & 60 & - & 0,005 & 16 & \color{red}300\color{black} \\
		\hline
		One-Point One-Fifth kein Offset & 350 & 34 & 0,75 & - & 18 & - \\
		\hline
		One-Point One-Fifth mit Offset & 2000 & 28 & 0,65 & - & 18 & 180 \\
		\hline
		Two-Point Konstant kein Offset & 1700 & 60 & 0,5 & - & 8 & - \\
		\hline
		Two-Point Konstant mit Offset & 600 & 48 & 0,3 & - & 16 & \color{red}180\color{black} \\
		\hline
		Two-Point Clegg kein Offset & 750 & 34 & - & 0,005 & 14 & - \\
		\hline
		Two-Point Clegg mit Offset & 800 & 36 & - & 0,045 & 20 & 210 \\
		\hline
		Two-Point One-Fifth kein Offset & 1350 & 40 & 0,35 & - & 20 & - \\
		\hline
		Two-Point One-Fifth mit Offset & 700 & 52 & 0,7 & - & 8 & 30 \\
		\hline
		Uniform Konstant kein Offset & 1100 & 52 & 0,8 & - & 8 & - \\
		\hline
		Uniform Konstant mit Offset & 1800 & 60 & 0,1 & - & 20 & 90 \\
		\hline
		Uniform Clegg kein Offset & 850 & 26 & - & 0,05 & 18 & - \\
		\hline
		Uniform Clegg mit Offset & 2000 & 44 & - & 0,05 & 14 & 180 \\
		\hline
		Uniform One-Fifth kein Offset & 1150 & 54 & 0,75 & - & 18 & - \\
		\hline
		Uniform One-Fifth mit Offset & 900 & 48 & 0,35 & - & 20 & 90 \\
	\end{tabular}
	\caption{Keijzer: Ergebnis Hyperparameteranalyse}
	\label{table:keijzerHPO}
\end{table}

Wie bereits in Abschnitt \ref{subsec:rohdatenParity} erklärt, kann eine Reduktion der Rechenknoten in einem CGP-Modell dazu führen, dass Systemressourcen und Rechenzeit gespart werden können.
Dies gilt natürlich nur unter Anbetracht, dass die Effizienz des CGPs zur Berechnung der Lösung nicht darunter leidet.
Wird der Mittelwert aller verwendeten Rechenknoten gebildet, liegt der Wert bei ungefähr 1305. 
Zu beobachten ist, dass die Anzahl der Rechenknoten beim CGP-Verfahren ohne Rekombination deutlich höher als dieser Durchschnittswert ausfällt.
Alle Mittelwerte der Anzahl an Rechenknoten der einzelnen Rekombinationsalgorithmen liegen unterhalb des allgemeinen Mittelwerts.
Das in diesem Sinne beste Verfahren ist in diesem Fall die Two-Point Rekombination.
Dieser Rekombinationsalgorithmus braucht durchschnittlich ca. 983 Rechenknoten und liegt damit weit unter dem allgemeinen Mittelwert.
Mit ca. 1117 Rechenknoten im Mittel folgt die One-Point Rekombination, während mit durchschnittlich 1300 Rechenknoten die Uniform Rekombination ungefähr dem allgemeinen Mittelwert entspricht.
Bei Betrachtung der Werte fällt außerdem auf, dass die One-Point Rekombination eine sehr große Kluft zwischen höchster und niedrigster Anzahl der Rechenknoten aufweist:
Einerseits wird der allgemein niedrigste Wert für die verwendeten Rechenknoten aufgezeigt (350), andererseits wird die obere Grenze ausgereizt (2000).
Bei der Two-Point Rekombination kann dieser Effekt nicht beobachtet werden.
Außerhalb eines Ausreißers (1700) sind alle Werte relativ ähnlich groß.
Ein Hinweis darauf mit welchen Parametern diese Beobachtungen in Verbindung stehen kann anhand der Tabelle nicht getroffen werden.
Weder von der Rekombinationsrate noch andere Parameter weisen offensichtliche Korrelationen auf.
Dieser Aspekt könnte in weiteren Untersuchungen näher betrachtet werden.\\
Ein Faktor, der ebenfalls Rechenzeit einsparen kann ist die Populationsgröße ($\mu$+$\lambda$).
Diese weisen im Keijzer-Testszenario keine deutlich ersichtlichen Zusammenhänge zu den CGP-Konfigurationen auf.
Die $\mu$- und $\lambda$-Werte weichen grundsätzlich nicht stark voneinander ab, mit einigen unregelmäßigen Schwankungen, bei denen die Werte deutlich kleiner werden.
Allerdings können diese Schwankungen nicht gleichzeitig bei $\mu$ und $\lambda$ festgestellt werden.
Für Two-Point und Uniform Rekombination verringert sich der $\lambda$-Wert bei der linear fallenden Rekombinationsrate ohne Offset.
Allerdings kann durch diese Ergebnisse nicht näher ergründet werden, ob darin ein näherer Zusammenhang liegt.\\
Bei den Rekombinationsraten fällt auf, dass für die Start-Rekombinationsrate der One-Fifth Regel nur sehr hohe oder sehr niedrige Raten gewählt wurden.
Dies könnte darauf hindeuten, dass noch extremere Werte gewählt worden wären, wenn die Grenzen der Start-Rekombinationsrate einen größeren Spielraum zugelassen hätten.
Für die One-Point und Uniform Rekombination wurden für die konstanten Start-Rekombinationsraten ebenfalls Werte nahe des Definitionsbereichs gewählt.
Für die Two-Point Rekombination trifft dies nicht zu.
Ebenfalls bei der linear fallenden Rekombinationsrate werden solche extremen Werte beobachtet.
Zusammenfassend lässt sich also sagen, dass für das Keijzer-Testszenario überwiegend die Grenzen des Definitionsbereichs als Rekombinationsraten gewählt wurden.\\
Die Offsets wurden durchschnittlich bei 153 Iterationen gewählt, wobei zu beachten ist, dass zwei Parametersätze ursprünglich bessere Ergebnisse erbracht hätten, ohne die Rekombination zu Beginn auszusetzen.
Die in der Tabelle \ref{table:keijzerHPO} rot markierten Offsets wären dementsprechend ursprünglich gleich 0 gewesen.
Mit den originalen Parametersätzen wäre der Mittelwerts des Offsets bei 100 gelegen.
Es fällt auf, dass die zweitbesten Parametersätze Offset-Werte besitzen, die deutlich höher sind, als der Mittelwert aller Offsets.
Der Offset der One-Point Rekombination mit linear fallender Rekombinationsrate ist dabei sogar der maximal mögliche Offset-Wert (300).
Diese extreme Schwankung der Offsets zwischen erstbestem und zweitbestem Parametersatz spricht dafür, dass die Güte eines CGPs nicht kausal mit der Höhe des Offsets zusammenhängt.\\

Folgend werden die Rohdaten der Keijzer-Testdurchläufe anhand von Tabelle \ref{table:keijzerRohdaten} bewertet.
Wie für Parity wurden dabei 50 Durchläufe für jede CGP-Konfiguration ausgeführt, um statistische Abweichungen einzufangen.


\begin{table}[H]
	\centering
	\begin{tabular}{c | c | c | c | c | c }
		\begin{turn}{270} \textbf{CGP-Konfigurationen} \end{turn} & \begin{turn}{270} \textbf{Anzahl pos. Mutationen} \end{turn} & \begin{turn}{270} \textbf{Anzahl pos. Rekomb.} \end{turn} & \begin{turn}{270} \textbf{Anzahl neg. Mutationen} \end{turn} & \begin{turn}{270} \textbf{Median Iter. pos. Rekomb.} \end{turn} & \begin{turn}{270} \textbf{Median Iter. bis Konv.} \end{turn}\\
		\hline
		keine Rekombination & 2260 & 0 & 0 & - & 624\\
		\hline
		One-Point Konstant kein Offset & 1183 & 26 & 16 & 13 & 756\\
		\hline
		One-Point Konstant mit Offset & 1531 & 0 & 0 & - & 374\\
		\hline
		One-Point Clegg kein Offset & 1584 & 59 & 28 & 5 & 243\\
		\hline
		One-Point Clegg mit Offset & 1534 & 0 & 0 & - & 396,5\\
		\hline
		One-Point One-Fifth kein Offset & 1307 & 35 & 18 & 3 & 766\\
		\hline
		One-Point One-Fifth mit Offset & 1776 & 0 & 0 & - & 229\\
		\hline
		Two-Point Konstant kein Offset & 1854 & 79 & 25 & 15 & 803\\
		\hline
		Two-Point Konstant mit Offset & 1361 & 0 & 0 & - & 548\\
		\hline
		Two-Point Clegg kein Offset & 1467 & 106 & 62 & 18,5 & 735\\
		\hline
		Two-Point Clegg mit Offset & 1862 & 0 & 0 & - & 201\\
		\hline
		Two-Point One-Fifth kein Offset & 1788 & 17 & 10 & 4 & 753\\
		\hline
		Two-Point One-Fifth mit Offset & 1124 & 0 & 0 & - & 158\\
		\hline
		Uniform Konstant kein Offset & 1224 & 480 & 239 & 24 & 363\\
		\hline
		Uniform Konstant mit Offset & 1890 & 0 & 0 & - & 260,5\\
		\hline
		Uniform Clegg kein Offset & 1386 & 77 & 57 & 6 & 723\\
		\hline
		Uniform Clegg mit Offset & 2232 & 0 & 0 & - & 385,5\\
		\hline
		Uniform One-Fifth kein Offset & 1491 & 106 & 50 & 8 & 155\\
		\hline
		Uniform One-Fifth mit Offset & 1749 & 0 & 0 & - & 445,5\\
	\end{tabular}
	\caption{Keijzer: Auswertung der Rohdaten}
	\label{table:keijzerRohdaten}
\end{table}

\subsection{Rohdatenanalyse: Encode}
\label{subsec:rohdatenEncode}

\begin{table}[H]
	\centering
	\begin{tabular}{c | c | c | c | c | c | c}
		\begin{turn}{270} \textbf{CGP-Konfigurationen} \end{turn} & \begin{turn}{270} \textbf{Anzahl pos. Mutationen} \end{turn} & \begin{turn}{270} \textbf{Anzahl pos. Rekomb.} \end{turn} & \begin{turn}{270} \textbf{Anzahl neg. Mutationen} \end{turn} & \begin{turn}{270} \textbf{Median Iter. pos. Rekomb.} \end{turn} & \begin{turn}{270} \textbf{Median Iter. bis Konv.} \end{turn} & \begin{turn}{270} \textbf{Stopp-Kriterium erfüllt} \end{turn}\\
		\hline
		One-Point Konstant: 0,125 & 1126 & 20 & 4 & 9,5 & 3362 & 9\\
		\hline
		One-Point Konstant: 0,25 & 1100 & 42 & 5 & 12,5 & 1963 & 9\\
		\hline
		One-Point Konstant: 0,375 & 1120 & 43 & 5 & 6 & 4578,5 & 8\\
		\hline
		One-Point Konstant: 0,5 & 1104 & 34 & 2 & 10,0 & 936 & 9\\
		\hline
		One-Point Konstant: 0,625 & 1096 & 45 & 4 & 12 & 3293,0 & 6\\
		\hline
		One-Point Konstant: 0,75 & 1103 & 55 & 11 & 16 & 1898,5 & 6\\
		\hline
		One-Point Konstant: 0,875 & 1093 & 59 & 6 & 15 & 1913,0 & 10\\
		\hline
		One-Point Konstant: 1,0 & 1086 & 58 & 9 & 8,0 & 3754,5 & 8\\
		\hline
		One-Point Clegg: 0,0005 & 1125 & 48 & 6 & 11,0 & 1905,5 & 12\\
		\hline
		One-Point Clegg: 0,0015 & 1062 & 51 & 5 & 12 & 2754,5 & 6\\
		\hline
		One-Point Clegg: 0,0025 & 1063 & 39 & 2 & 9 & 1530 & 5\\
		\hline
		One-Point Clegg: 0,0035 & 1114 & 34 & 2 & 10,5 & 2840,5 & 6\\
		\hline
		One-Point Clegg: 0,0045 & 1127 & 38 & 2 & 10,0 & 2558 & 9\\
		\hline
		One-Point Clegg: 0,0055 & 1079 & 55 & 7 & 9 & 4007,5 & 8\\
		\hline
		One-Point One-Fifth: 0,125 & 1138 & 19 & 1 & 7 & 3950 & 9\\
		\hline
		One-Point One-Fifth: 0,25 & 1135 & 22 & 3 & 7,0 & 4111,5 & 12\\
		\hline
		One-Point One-Fifth: 0,375 & 1155 & 20 & 1 & 4,0 & 1685 & 7\\
		\hline
		One-Point One-Fifth: 0,5 & 1123 & 26 & 0 & 5,0 & 3205 & 11\\
		\hline
		One-Point One-Fifth: 0,625 & 1100 & 35 & 3 & 6 & 3809 & 11\\
		\hline
		One-Point One-Fifth: 0,75 & 1089 & 25 & 1 & 6 & 795 & 7\\
		\hline
		One-Point One-Fifth: 0,875 & 1126 & 32 & 5 & 8,0 & 5213 & 5\\
		\hline
		One-Point One-Fifth: 1,0 & 1112 & 42 & 3 & 5,5 & 2863 & 5\\
	\end{tabular}
	\caption{Encode One-Point Rekombination: Auswertung der Rohdaten}
	\label{table:encodeOnePointRohdaten}
\end{table}

\begin{table}[H]
	\centering
	\begin{tabular}{c | c | c | c | c | c | c}
		\begin{turn}{270} \textbf{CGP-Konfigurationen} \end{turn} & \begin{turn}{270} \textbf{Anzahl pos. Mutationen} \end{turn} & \begin{turn}{270} \textbf{Anzahl pos. Rekomb.} \end{turn} & \begin{turn}{270} \textbf{Anzahl neg. Mutationen} \end{turn} & \begin{turn}{270} \textbf{Median Iter. pos. Rekomb.} \end{turn} & \begin{turn}{270} \textbf{Median Iter. bis Konv.} \end{turn} & \begin{turn}{270} \textbf{Stopp-Kriterium erfüllt} \end{turn}\\
		\hline
		Two-Point Konstant: 0,125 & 1044 & 27 & 2 & 6 & 4039,5 & 2\\
		\hline
		Two-Point Konstant: 0,25 & 1062 & 35 & 9 & 7 & 4292 & 3\\
		\hline
		Two-Point Konstant: 0,375 & 1076 & 45 & 4 & 5 & 1840 & 5\\
		\hline
		Two-Point Konstant: 0,5 & 1041 & 50 & 12 & 7,5 & 1666 & 3\\
		\hline
		Two-Point Konstant: 0,625 & 1050 & 61 & 15 & 7 & 7370 & 3\\
		\hline
		Two-Point Konstant: 0,75 & 1019 & 61 & 14 & 10 & 3138 & 1\\
		\hline
		Two-Point Konstant: 0,875 & 1023 & 73 & 12 & 9 & 4445 & 5\\
		\hline
		Two-Point Konstant: 1,0 & 1031 & 65 & 10 & 5 & 3476,5 & 6\\
		\hline
		Two-Point Clegg: 0,0005 & 1041 & 70 & 7 & 6,0 & 3383 & 5\\
		\hline
		Two-Point Clegg: 0,0015 & 1015 & 53 & 8 & 8 & 2524,5 & 2\\
		\hline
		Two-Point Clegg: 0,0025 & 1026 & 75 & 8 & 9 & 6024 & 3\\
		\hline
		Two-Point Clegg: 0,0035 & 977 & 64 & 9 & 7,0 & 2061 & 3\\
		\hline
		Two-Point Clegg: 0,0045 & 1046 & 51 & 9 & 8 & 2616 & 5\\
		\hline
		Two-Point Clegg: 0,0055 & 1019 & 52 & 8 & 9,5 & 3475 & 7\\
		\hline
		Two-Point One-Fifth: 0,125 & 1067 & 10 & 3 & 5,0 & 575,0 & 2\\
		\hline
		Two-Point One-Fifth: 0,25 & 1077 & 16 & 4 & 4,0 & 707 & 1\\
		\hline
		Two-Point One-Fifth: 0,375 & 1067 & 17 & 1 & 4 & 7538 & 1\\
		\hline
		Two-Point One-Fifth: 0,5 & 1063 & 24 & 5 & 5,0 & 3363 & 3\\
		\hline
		Two-Point One-Fifth: 0,625 & 1075 & 32 & 6 & 4,0 & 4919 & 5\\
		\hline
		Two-Point One-Fifth: 0,75 & 1078 & 34 & 8 & 4,0 & 1312,0 & 6\\
		\hline
		Two-Point One-Fifth: 0,875 & 1088 & 36 & 4 & 5,0 & 2529,5 & 6\\
		\hline
		Two-Point One-Fifth: 1,0 & 1049 & 43 & 7 & 4 & 1140 & 5\\
	\end{tabular}
	\caption{Encode Two-Point Rekombination: Auswertung der Rohdaten}
	\label{table:encodeTwoPointRohdaten}
\end{table}

\begin{table}[H]
	\centering
	\begin{tabular}{c | c | c | c | c | c | c}
		\begin{turn}{270} \textbf{CGP-Konfigurationen} \end{turn} & \begin{turn}{270} \textbf{Anzahl pos. Mutationen} \end{turn} & \begin{turn}{270} \textbf{Anzahl pos. Rekomb.} \end{turn} & \begin{turn}{270} \textbf{Anzahl neg. Mutationen} \end{turn} & \begin{turn}{270} \textbf{Median Iter. pos. Rekomb.} \end{turn} & \begin{turn}{270} \textbf{Median Iter. bis Konv.} \end{turn} & \begin{turn}{270} \textbf{Stopp-Kriterium erfüllt} \end{turn}\\
		\hline
		Uniform Konstant: 0,125 & 1105 & 52 & 12 & 7,0 & 4058 & 9\\
		\hline
		Uniform Konstant: 0,25 & 1099 & 69 & 3 & 13 & 3831,5 & 10\\
		\hline
		Uniform Konstant: 0,375 & 1076 & 64 & 10 & 9,0 & 4071,5 & 6\\
		\hline
		Uniform Konstant: 0,5 & 1105 & 68 & 2 & 12,5 & 993,5 & 8\\
		\hline
		Uniform Konstant: 0,625 & 1099 & 78 & 3 & 9,0 & 1306,5 & 6\\
		\hline
		Uniform Konstant: 0,75 & 1085 & 81 & 6 & 10 & 3459,0 & 8\\
		\hline
		Uniform Konstant: 0,875 & 1088 & 83 & 1 & 10 & 2490,5 & 6\\
		\hline
		Uniform Konstant: 1,0 & 1081 & 67 & 5 & 10 & 4785 & 9\\
		\hline
		Uniform Clegg: 0,0005 & 1092 & 76 & 5 & 8,0 & 2562 & 13\\
		\hline
		Uniform Clegg: 0,0015 & 1099 & 78 & 2 & 10,5 & 1777 & 11\\
		\hline
		Uniform Clegg: 0,0025 & 1096 & 74 & 7 & 11,0 & 2401,5 & 10\\
		\hline
		Uniform Clegg: 0,0035 & 1095 & 56 & 4 & 12,0 & 4674,5 & 4\\
		\hline
		Uniform Clegg: 0,0045 & 1080 & 70 & 2 & 13,0 & 5118,0 & 10\\
		\hline
		Uniform Clegg: 0,0055 & 1122 & 68 & 10 & 9,5 & 4044,5 & 8\\
		\hline
		Uniform One-Fifth: 0,125 & 1152 & 18 & 2 & 6,5 & 3876 & 7\\
		\hline
		Uniform One-Fifth: 0,25 & 1126 & 25 & 6 & 6 & 3393 & 11\\
		\hline
		Uniform One-Fifth: 0,375 & 1128 & 29 & 2 & 8 & 988,5 & 6\\
		\hline
		Uniform One-Fifth: 0,5 & 1156 & 44 & 5 & 7,0 & 3968 & 11\\
		\hline
		Uniform One-Fifth: 0,625 & 1122 & 40 & 4 & 8,0 & 1781 & 7\\
		\hline
		Uniform One-Fifth: 0,75 & 1104 & 47 & 4 & 6 & 3385,5 & 12\\
		\hline
		Uniform One-Fifth: 0,875 & 1075 & 51 & 10 & 10 & 1831,5 & 6\\
		\hline
		Uniform One-Fifth: 1,0 & 1129 & 55 & 7 & 6 & 4972 & 9\\
	\end{tabular}
	\caption{Encode Uniform Rekombination: Auswertung der Rohdaten}
	\label{table:encodeUniformRohdaten}
\end{table}

Encode ohne Rekombination:
\begin{itemize}
	\item Median Iterationen bis Konvergenz: 3847,5
	\item Stopp-Kriterium erfüllt: 8
\end{itemize}


\subsection{Rohdatenanalyse: Koza}
\label{subsec:rohdatenKoza}

\begin{table}[H]
	\centering
	\begin{tabular}{c | c | c | c | c | c | c}
		\begin{turn}{270} \textbf{CGP-Konfigurationen} \end{turn} & \begin{turn}{270} \textbf{Anzahl pos. Mutationen} \end{turn} & \begin{turn}{270} \textbf{Anzahl pos. Rekomb.} \end{turn} & \begin{turn}{270} \textbf{Anzahl neg. Mutationen} \end{turn} & \begin{turn}{270} \textbf{Median Iter. pos. Rekomb.} \end{turn} & \begin{turn}{270} \textbf{Median Iter. bis Konv.} \end{turn} & \begin{turn}{270} \textbf{Stopp-Kriterium erfüllt} \end{turn}\\
		\hline
		One-Point Konstant: 0,125 & 1338 & 8 & 8 & 46,0 & 616 & 41\\
		\hline
		One-Point Konstant: 0,25 & 1180 & 22 & 16 & 7,5 & 611 & 45\\
		\hline
		One-Point Konstant: 0,375 & 1493 & 41 & 25 & 46 & 2012 & 47\\
		\hline
		One-Point Konstant: 0,5 & 1477 & 54 & 40 & 20,0 & 602,5 & 36\\
		\hline
		One-Point Konstant: 0,625 & 1287 & 45 & 23 & 91 & 2302 & 45\\
		\hline
		One-Point Konstant: 0,75 & 1423 & 62 & 43 & 30,5 & 989 & 47\\
		\hline
		One-Point Konstant: 0,875 & 1593 & 77 & 48 & 20 & 848,0 & 34\\
		\hline
		One-Point Konstant: 1,0 & 1548 & 128 & 83 & 75,5 & 667 & 41\\
		\hline
		One-Point Clegg: 0,0005 & 1460 & 92 & 56 & 60,5 & 603 & 43\\
		\hline
		One-Point Clegg: 0,0015 & 1525 & 71 & 43 & 15 & 659,0 & 40\\
		\hline
		One-Point Clegg: 0,0025 & 1303 & 49 & 31 & 8 & 583 & 41\\
		\hline
		One-Point Clegg: 0,0035 & 1233 & 57 & 35 & 6 & 529 & 42\\
		\hline
		One-Point Clegg: 0,0045 & 1187 & 57 & 36 & 14 & 511,0 & 44\\
		\hline
		One-Point Clegg: 0,0055 & 1622 & 48 & 28 & 11,0 & 1320,0 & 42\\
		\hline
		One-Point One-Fifth: 0,125 & 1434 & 1 & 1 & 2 & 462,0 & 44\\
		\hline
		One-Point One-Fifth: 0,25 & 1544 & 3 & 6 & 2 & 1400,0 & 44\\
		\hline
		One-Point One-Fifth: 0,375 & 1005 & 10 & 8 & 2,5 & 838,5 & 46\\
		\hline
		One-Point One-Fifth: 0,5 & 1137 & 8 & 10 & 2,5 & 360,5 & 41\\
		\hline
		One-Point One-Fifth: 0,625 & 1207 & 14 & 12 & 3,5 & 947,0 & 44\\
		\hline
		One-Point One-Fifth: 0,75 & 1387 & 12 & 11 & 3,0 & 910 & 43\\
		\hline
		One-Point One-Fifth: 0,875 & 1160 & 26 & 14 & 4,0 & 301 & 43\\
		\hline
		One-Point One-Fifth: 1,0 & 1393 & 29 & 25 & 6 & 568 & 43\\
	\end{tabular}
	\caption{Koza One-Point Rekombination: Auswertung der Rohdaten}
	\label{table:kozaOnePointRohdaten}
\end{table}

\begin{table}[H]
	\centering
	\begin{tabular}{c | c | c | c | c | c | c}
		\begin{turn}{270} \textbf{CGP-Konfigurationen} \end{turn} & \begin{turn}{270} \textbf{Anzahl pos. Mutationen} \end{turn} & \begin{turn}{270} \textbf{Anzahl pos. Rekomb.} \end{turn} & \begin{turn}{270} \textbf{Anzahl neg. Mutationen} \end{turn} & \begin{turn}{270} \textbf{Median Iter. pos. Rekomb.} \end{turn} & \begin{turn}{270} \textbf{Median Iter. bis Konv.} \end{turn} & \begin{turn}{270} \textbf{Stopp-Kriterium erfüllt} \end{turn}\\
		\hline
		Two-Point Konstant: 0,125 & 1383 & 39 & 24 & 172 & 106 & 45\\
		\hline
		Two-Point Konstant: 0,25 & 1475 & 82 & 55 & 47,5 & 613,0 & 48\\
		\hline
		Two-Point Konstant: 0,375 & 1207 & 91 & 58 & 75 & 153,0 & 43\\
		\hline
		Two-Point Konstant: 0,5 & 1120 & 84 & 49 & 30,5 & 194 & 46\\
		\hline
		Two-Point Konstant: 0,625 & 1013 & 90 & 38 & 16,0 & 124,0 & 49\\
		\hline
		Two-Point Konstant: 0,75 & 1373 & 141 & 65 & 23 & 190 & 43\\
		\hline
		Two-Point Konstant: 0,875 & 1400 & 138 & 76 & 28,5 & 731,0 & 46\\
		\hline
		Two-Point Konstant: 1,0 & 1176 & 215 & 140 & 52 & 207,0 & 48\\
		\hline
		Two-Point Clegg: 0,0005 & 1446 & 189 & 99 & 31 & 205,0 & 47\\
		\hline
		Two-Point Clegg: 0,0015 & 1167 & 101 & 59 & 16 & 154,0 & 45\\
		\hline
		Two-Point Clegg: 0,0025 & 1396 & 100 & 53 & 19,0 & 504,5 & 44\\
		\hline
		Two-Point Clegg: 0,0035 & 1043 & 93 & 45 & 13 & 142 & 46\\
		\hline
		Two-Point Clegg: 0,0045 & 1023 & 96 & 50 & 15,5 & 96,0 & 46\\
		\hline
		Two-Point Clegg: 0,0055 & 1160 & 90 & 43 & 12,0 & 223,5 & 49\\
		\hline
		Two-Point One-Fifth: 0,125 & 1422 & 4 & 4 & 4,0 & 226 & 45\\
		\hline
		Two-Point One-Fifth: 0,25 & 1348 & 8 & 5 & 7,0 & 404,5 & 44\\
		\hline
		Two-Point One-Fifth: 0,375 & 1365 & 14 & 4 & 4,0 & 460 & 46\\
		\hline
		Two-Point One-Fifth: 0,5 & 1761 & 13 & 4 & 6 & 584 & 47\\
		\hline
		Two-Point One-Fifth: 0,625 & 1138 & 18 & 13 & 7,5 & 164,5 & 47\\
		\hline
		Two-Point One-Fifth: 0,75 & 1041 & 21 & 13 & 5 & 213,5 & 46\\
		\hline
		Two-Point One-Fifth: 0,875 & 1050 & 41 & 19 & 7 & 185,0 & 48\\
		\hline
		Two-Point One-Fifth: 1,0 & 1142 & 31 & 20 & 5 & 348 & 49\\
	\end{tabular}
	\caption{Koza Two-Point Rekombination: Auswertung der Rohdaten}
	\label{table:kozaTwoPointRohdaten}
\end{table}

\begin{table}[H]
	\centering
	\begin{tabular}{c | c | c | c | c | c | c}
		\begin{turn}{270} \textbf{CGP-Konfigurationen} \end{turn} & \begin{turn}{270} \textbf{Anzahl pos. Mutationen} \end{turn} & \begin{turn}{270} \textbf{Anzahl pos. Rekomb.} \end{turn} & \begin{turn}{270} \textbf{Anzahl neg. Mutationen} \end{turn} & \begin{turn}{270} \textbf{Median Iter. pos. Rekomb.} \end{turn} & \begin{turn}{270} \textbf{Median Iter. bis Konv.} \end{turn} & \begin{turn}{270} \textbf{Stopp-Kriterium erfüllt} \end{turn}\\
		\hline
		Uniform Konstant: 0,125 & 1178 & 123 & 75 & 32 & 316,0 & 44\\
		\hline
		Uniform Konstant: 0,25 & 1233 & 156 & 90 & 45,0 & 479 & 43\\
		\hline
		Uniform Konstant: 0,375 & 1166 & 245 & 132 & 37 & 257,5 & 47\\
		\hline
		Uniform Konstant: 0,5 & 1084 & 181 & 92 & 28 & 113 & 45\\
		\hline
		Uniform Konstant: 0,625 & 1277 & 292 & 152 & 31,5 & 430 & 47\\
		\hline
		Uniform Konstant: 0,75 & 1301 & 269 & 141 & 33 & 382,0 & 43\\
		\hline
		Uniform Konstant: 0,875 & 1102 & 324 & 181 & 32,0 & 467,0 & 44\\
		\hline
		Uniform Konstant: 1,0 & 1080 & 320 & 199 & 22,0 & 176,0 & 43\\
		\hline
		Uniform Clegg: 0,0005 & 1155 & 369 & 197 & 41 & 709 & 46\\
		\hline
		Uniform Clegg: 0,0015 & 793 & 255 & 147 & 17 & 81 & 48\\
		\hline
		Uniform Clegg: 0,0025 & 887 & 312 & 181 & 26,0 & 140,0 & 48\\
		\hline
		Uniform Clegg: 0,0035 & 1415 & 309 & 162 & 18 & 162 & 46\\
		\hline
		Uniform Clegg: 0,0045 & 1404 & 294 & 153 & 18,5 & 637 & 48\\
		\hline
		Uniform Clegg: 0,0055 & 1364 & 294 & 141 & 20,0 & 143 & 46\\
		\hline
		Uniform One-Fifth: 0,125 & 1545 & 11 & 6 & 6 & 147 & 41\\
		\hline
		Uniform One-Fifth: 0,25 & 1314 & 23 & 8 & 6 & 153 & 44\\
		\hline
		Uniform One-Fifth: 0,375 & 1384 & 36 & 14 & 8,5 & 273 & 47\\
		\hline
		Uniform One-Fifth: 0,5 & 1718 & 59 & 28 & 6 & 848,0 & 46\\
		\hline
		Uniform One-Fifth: 0,625 & 1244 & 67 & 35 & 6 & 363 & 47\\
		\hline
		Uniform One-Fifth: 0,75 & 1320 & 82 & 49 & 6,0 & 207,0 & 44\\
		\hline
		Uniform One-Fifth: 0,875 & 1406 & 86 & 49 & 10,0 & 470,0 & 46\\
		\hline
		Uniform One-Fifth: 1,0 & 1444 & 79 & 40 & 7 & 794,5 & 47\\
	\end{tabular}
	\caption{Koza Uniform Rekombination: Auswertung der Rohdaten}
	\label{table:kozaUniformRohdaten}
\end{table}

Koza ohne Rekombination:
\begin{itemize}
	\item Median Iterationen bis Konvergenz: 208
	\item Stopp-Kriterium erfüllt: 48
\end{itemize}


\subsection{Rohdatenanalyse: Zusammenfassung}
\label{subsec:rohdatenZusammenfassung}

Rechenknoten:
Parity:
- ohne Rekombination weniger als Mittelwert
- One-Point höchster Mittelwert
- Two-Point weniger etwas mehr als Mittelwert
- Uniform am wenigsten
- Trendschätzung: ohne Rekombination am wenigsten; aufwendigere Rekombination weniger Rechenknoten als leichtere
- Streuung Rechenknoten höher für komplexere Rekombination

Keijzer:
- ohne Rekombination deutlich höher als MIttelwert
- alle Rekombinationsverfahren unter Mittelwert
- Two-Point am wenigsten
- One-Point am zweitwenigsten
- Uniform etwas unter Mittelwert
- Streuung am höchsten für One-Point
- Two-Point kaum Streuung

Population:
Parity:
- keine Rekombination am wenigsten Population
- Rest höhere Population

Keijzer:
- keine starken Abweichungen über Rekombinationsalgorithmen hinweg
- einige unregelmäßige Schwankungen, bei denen die Werte vereinzelt deutlich kleiner werden
- Two-Point und Uniform Clegg ohne Offset: lambda Wert deutlich kleiner (aber Zusammenhang kann nur mit dieser einer Keijzer-HPO nicht bestätigt werden)

Rekombinationsraten:
Parity:
- konstant vermeidet hohe und tiefe Werte
- one fifth nutzt gesamten Definitionsbereich
- Clegg nahezu gesamter Definitionsbereich genutzt

Keijzer:
- One-Fifth nur sehr hohe oder sehr niedrige Werte
- konstant: onePoint und Uniform auch extreme Werte
- Clegg: auch extreme Werte

Offset:
Parity:
- Two-Point Rekombination geringster Offset
- Uniform deutlich mehr Offset als Two-Point und Mittelwert
- neu eingefügte Werte ca. gleich wie Mittelwert

Keijzer:
- Offsets, die ersetzt wurden extrem hoch
-> Güte des CGPs hängt nicht kausal mit der Höhe des Offsets zusammen
- Mittelwert Uniform am geringsten (aber nur wenn Ersatzwerte betrachtet werden, ansonsten Uniform am höchsten; ohne Ersatzwerte Two-Point am geringsten)



TODO: Argumentation; dass Offset nicht so viel bringt (damit Kapitel \glqq praktischer Teil\grqq\space darauf verweisen kann)\\
- HPO: mehrere Durchläufe wählen garkeinen Offset\\
- restliche Tests vergleichen zwischen mit Offset und ohne\\
- Rohdatenanalyse: Rekombinationserfolge; wenn kein Offset da ist

TODO: HPO mit Vorsicht bewerten -> ggf. Ausblick auf HPO mit mehr als 10 Durchläufen pro Bewertungsschritt für mehr Einblick

TODO: Wieso wurde Offset überhaupt in HPO Ergebnissen (7 aus 9) verwendet?





\section{Ergebnisse Bayes'sche Analyse}
\label{sec:ergebnisseBayes}

\subsection{Bayes'sche Analyse: Parity}
\label{subsec:bayesParity}

\begin{table}[H]
	\centering
	\begin{tabular}{c | c | c | c}
		\textbf{CPG-Konfiguration} & \textbf{HPDI (Iter.)} & \textbf{MW} & \textbf{PL-Platz}\\
		\hline
		Parity keine Rekombination & (237,958; 541,181) & 357,440 & 0,034518\\
		\hline
		Parity One-Point Konstant kein Offset & (72,803; 131,161) & 98,059 & 0,072723\\
		\hline
		Parity One-Point Konstant mit Offset & (117,5867; 241,124) & 168,301 & 0,059872\\
		\hline
		Parity One-Point Clegg kein Offset & (111,330; 204,672) & 151,207 & 0,058134\\
		\hline
		Parity One-Point Clegg mit Offset & (136,755; 318,250) & 208,318 & 0,064637\\
		\hline
		Parity One-Point One-Fifth kein Offset & (305,150; 879,234) & 516,320 & 0,041265\\
		\hline
		Parity One-Point One-Fifth mit Offset & (90,584; 192,603) & 132,238 & 0,072643\\
		\hline
		Parity Two-Point Konstant kein Offset & (161,166; 366,726) & 243,306 & 0,056980\\
		\hline
		Parity Two-Point Konstant mit Offset & (185,680; 398,045) & 271,755 & 0,043839\\
		\hline
		Parity Two-Point Clegg kein Offset & (143,411; 320,218) & 214,015 & 0,055768\\
		\hline
		Parity Two-Point Clegg mit Offset & (279,211; 671,061) & 429,804 & 0,041189\\
		\hline
		Parity Two-Point One-Fifth kein Offset & (165,870; 369,988) & 247,230 & 0,051913\\
		\hline
		Parity Two-Point One-Fifth mit Offset & (187,841; 467,866) & 294,748 & 0,049240\\
		\hline
		Parity Uniform Konstant kein Offset & (182,312; 414,373) & 275,283 & 0,043305\\
		\hline
		Parity Uniform Konstant mit Offset & (158,116; 357,830) & 238,048 & 0,060934\\
		\hline
		Parity Uniform Clegg kein Offset & (120,352; 267,732) & 179,796 & 0,063633\\
		\hline
		Parity Uniform Clegg mit Offset & (147,184; 317,828) & 215,359 & 0,051520\\
		\hline
		Parity Uniform One-Fifth kein Offset & (233,817; 551,718) & 356,681 & 0,040460\\
		\hline
		Parity Uniform One-Fifth mit Offset & (211,377; 519,471) & 329,524 & 0,037426\\
	\end{tabular}
	\label{table:parityBayesian}
	\caption{Parity: Bayes'sche Analyse}
\end{table}

\subsection{Bayes'sche Analyse: Keijzer}
\label{subsec:bayesKeijzer}

\begin{table}[H]
	\centering
	\begin{tabular}{c | c | c | c}
		\textbf{CGP-Konfiguration} & \textbf{HPDI (Iter.)} & \textbf{MW} & \textbf{PL-Platz}\\
		\hline
		Keijzer keine Rekombination & (5446,690; 16772,637) & 9551,657 & 0,046669\\
		\hline
		Keijzer One-Point Konstant kein Offset & (3155,154; 8685,286) & 5214,911 & 0,056954\\
		\hline
		Keijzer One-Point Konstant mit Offset & (5791,946; 19545,587) & 10837,605 & 0,047578\\
		\hline
		Keijzer One-Point Clegg kein Offset & (4055,470; 14388,586) & 7754,999 & 0,053452\\
		\hline
		Keijzer One-Point Clegg mit Offset & (5182,282; 17954,643) & 9757,394 & 0,046700\\
		\hline
		Keijzer One-Point One-Fifth kein Offset & (3746,883; 10479,686) & 6233,279 & 0,050072\\
		\hline
		Keijzer One-Point One-Fifth mit Offset & (4972,228; 16087,780) & 9402,618 & 0,057995\\
		\hline
		Keijzer Two-Point Konstant kein Offset & (4207,924; 13285,672) & 7464,243 & 0,061749\\
		\hline
		Keijzer Two-Point Konstant mit Offset & (3544,144; 10355,412) & 6056,680 & 0,064480\\
		\hline
		Keijzer Two-Point Clegg kein Offset & (5468,641; 17877,607) & 10017,892 & 0,033887\\
		\hline
		Keijzer Two-Point Clegg mit Offset & (4105,772; 13578,136) & 7663,952 & 0,055730\\
		\hline
		Keijzer Two-Point One-Fifth kein Offset & (4408,582; 12408,349) & 7381,426 & 0,053392\\
		\hline
		Keijzer Two-Point One-Fifth mit Offset & (6970,619; 22911,685) & 13432,511 & 0,052018\\
		\hline
		Keijzer Uniform Konstant kein Offset & (8010,603; 23268,991) & 14688,052 & 0,049104\\
		\hline
		Keijzer Uniform Konstant mit Offset & (3406,437; 10751,973) & 6052,727 & 0,069820\\
		\hline
		Keijzer Uniform Clegg kein Offset & (3856,791; 10850,478) & 6440,975 & 0,060934\\
		\hline
		Keijzer Uniform Clegg mit Offset & (6067,492; 20498,984) & 11238,547 & 0,043535\\
		\hline
		Keijzer Uniform One-Fifth kein Offset & (3511,828; 12119,350) & 6587,897 & 0,070569\\
		\hline
		Keijzer Uniform One-Fifth mit Offset & (4025,084; 11844,356) & 6893,370 & 0,025362\\
	\end{tabular}
	\label{table:keijzerBayesian}
	\caption{Keijzer: Bayes'sche Analyse}
\end{table}

\subsection{Bayes'sche Analyse: Encode}
\label{subsec:bayesEncode}

\begin{table}[H]
	\centering
	\begin{tabular}{c | c | c | c}
		\textbf{CGP-Konfiguration} & \textbf{HPDI (Fitn.)} & \textbf{MW} & \textbf{PL-Platz}\\
		\hline
		Encode keine Rekombination & (0,02727; 0,06853) & 0,04351 & 0,040300\\
		\hline
		Encode One-Point Konstant: 0,125 & (0,02416; 0,06463) & 0,03969 & 0,039235\\
		\hline
		Encode One-Point Konstant: 0,25 & (0,02342; 0,06255) & 0,03837 & 0,044399\\
		\hline
		Encode One-Point Konstant: 0,375 & (0,02216; 0,05684) & 0,036 & 0,052266\\
		\hline
		Encode One-Point Konstant: 0,5 & (0,02581; 0,06657) & 0,04185 & 0,041112\\
		\hline
		Encode One-Point Konstant: 0,625 & (0,02703; 0,06417) & 0,04188 & 0,039622\\
		\hline
		Encode One-Point Konstant: 0,75 & (0,02484; 0,05755) & 0,03791 & 0,047276\\
		\hline
		Encode One-Point Konstant: 0,875 & (0,02311; 0,06519) & 0,03911 & 0,042973\\
		\hline
		Encode One-Point Konstant: 1,0 & (0,02634; 0,06865) & 0,04257 & 0,040402\\
		\hline
		Encode One-Point Clegg: 0,0005 & (0,01874; 0,05612) & 0,03254 & 0,058275\\
		\hline
		Encode One-Point Clegg: 0,0015 & (0,03085; 0,07312) & 0,04756 & 0,033251\\
		\hline
		Encode One-Point Clegg: 0,0025 & (0,03174; 0,07112) & 0,04783 & 0,036894\\
		\hline
		Encode One-Point Clegg: 0,0035 & (0,0259; 0,05957) & 0,03961 & 0,043415\\
		\hline
		Encode One-Point Clegg: 0,0045 & (0,02262; 0,05981) & 0,03682 & 0,047645\\
		\hline
		Encode One-Point Clegg: 0,0055 & (0,02478; 0,06459) & 0,04044 & 0,044801\\
		\hline
		Encode One-Point One-Fifth: 0,125 & (0,02234; 0,06005) & 0,03702 & 0,048045\\
		\hline
		Encode One-Point One-Fifth: 0,25 & (0,02132; 0,06241) & 0,0368 & 0,053389\\
		\hline
		Encode One-Point One-Fifth: 0,375 & (0,02423; 0,05867) & 0,03784 & 0,049249\\
		\hline
		Encode One-Point One-Fifth: 0,5 & (0,01903; 0,05371) & 0,03227 & 0,060271\\
		\hline
		Encode One-Point One-Fifth: 0,625 & (0,02184; 0,0643) & 0,03777 & 0,045892\\
		\hline
		Encode One-Point One-Fifth: 0,75 & (0,02746; 0,06625) & 0,04286 & 0,041786\\
		\hline
		Encode One-Point One-Fifth: 0,875 & (0,02436; 0,05364) & 0,03622 & 0,049501\\
	\end{tabular}
	\label{table:encodeOnePointBayesian}
	\caption{Encode One-Point Rekombination: Bayes'sche Analyse}
\end{table}

\begin{table}[H]
	\centering
	\begin{tabular}{c | c | c | c}
		\textbf{CGP-Konfiguration} & \textbf{HPDI (Fitn.)} & \textbf{MW} & \textbf{PL-Platz}\\
		\hline
		Encode keine Rekombination & (0,02727; 0,06853) & 0,04351 & 0,064969\\
		\hline
		Encode Two-Point Konstant: 0,125 & (0,04657; 0,08164) & 0,06176 & 0,036705\\
		\hline
		Encode Two-Point Konstant: 0,25 & (0,0417; 0,08229) & 0,05853 & 0,038448\\
		\hline
		Encode Two-Point Konstant: 0,375 & (0,03172; 0,07007) & 0,04742 & 0,056475\\
		\hline
		Encode Two-Point Konstant: 0,5 & (0,04318; 0,08532) & 0,06094 & 0,037807\\
		\hline
		Encode Two-Point Konstant: 0,625 & (0,03854; 0,07511) & 0,05397 & 0,047998\\
		\hline
		Encode Two-Point Konstant: 0,75 & (0,05211; 0,08389) & 0,06618 & 0,035496\\
		\hline
		Encode Two-Point Konstant: 0,875 & (0,03614; 0,08027) & 0,05438 & 0,045400\\
		\hline
		Encode Two-Point Konstant: 1,0 & (0,03386; 0,07925) & 0,052 & 0,047936\\
		\hline
		Encode Two-Point Clegg: 0,0005 & (0,03588; 0,07852) & 0,05322 & 0,046264\\
		\hline
		Encode Two-Point Clegg: 0,0015 & (0,0457; 0,08078) & 0,06077 & 0,036224\\
		\hline
		Encode Two-Point Clegg: 0,0025 & (0,03801; 0,07255) & 0,05258 & 0,050576\\
		\hline
		Encode Two-Point Clegg: 0,0035 & (0,04651; 0,09123) & 0,06523 & 0,034689\\
		\hline
		Encode Two-Point Clegg: 0,0045 & (0,0321; 0,07079) & 0,04795 & 0,054090\\
		\hline
		Encode Two-Point Clegg: 0,0055 & (0,03634; 0,08965) & 0,0573 & 0,039299\\
		\hline
		Encode Two-Point One-Fifth: 0,125 & (0,04561; 0,08148) & 0,06082 & 0,042431\\
		\hline
		Encode Two-Point One-Fifth: 0,25 & (0,04353; 0,0687) & 0,05485 & 0,047932\\
		\hline
		Encode Two-Point One-Fifth: 0,375 & (0,04642; 0,07558) & 0,05903 & 0,041419\\
		\hline
		Encode Two-Point One-Fifth: 0,5 & (0,03834; 0,07567) & 0,0539 & 0,045002\\
		\hline
		Encode Two-Point One-Fifth: 0,625 & (0,036; 0,07918) & 0,0536 & 0,047777\\
		\hline
		Encode Two-Point One-Fifth: 0,75 & (0,03114; 0,07277) & 0,04805 & 0,053331\\
		\hline
		Encode Two-Point One-Fifth: 0,875 & (0,03244; 0,07634) & 0,05012 & 0,049732\\
	\end{tabular}
	\label{table:encodeTwoPointBayesian}
	\caption{Encode Two-Point Rekombination: Bayes'sche Analyse}
\end{table}

\begin{table}[H]
	\centering
	\begin{tabular}{c | c | c | c}
		\textbf{CGP-Konfiguration} & \textbf{HPDI (Fitn.)} & \textbf{MW} & \textbf{PL-Platz}\\
		\hline
		Encode keine Rekombination & (0,02727; 0,06853) & 0,04351 & 0,037060\\
		\hline
		Encode Uniform Konstant: 0,125 & (0,02473; 0,06504) & 0,04031 & 0,038501\\
		\hline
		Encode Uniform Konstant: 0,25 & (0,02356; 0,06581) & 0,03963 & 0,040006\\
		\hline
		Encode Uniform Konstant: 0,375 & (0,03039; 0,07186) & 0,047 & 0,034479\\
		\hline
		Encode Uniform Konstant: 0,5 & (0,02241; 0,05759) & 0,03583 & 0,050278\\
		\hline
		Encode Uniform Konstant: 0,625 & (0,02416; 0,05673) & 0,03719 & 0,046887\\
		\hline
		Encode Uniform Konstant: 0,75 & (0,02399; 0,06258) & 0,03882 & 0,050394\\
		\hline
		Encode Uniform Konstant: 0,875 & (0,02203; 0,05148) & 0,03393 & 0,052041\\
		\hline
		Encode Uniform Konstant: 1,0 & (0,022; 0,05696) & 0,03592 & 0,048403\\
		\hline
		Encode Uniform Clegg: 0,0005 & (0,01801; 0,0552) & 0,03171 & 0,053799\\
		\hline
		Encode Uniform Clegg: 0,0015 & (0,02101; 0,06014) & 0,03576 & 0,052238\\
		\hline
		Encode Uniform Clegg: 0,0025 & (0,01737; 0,04703) & 0,02884 & 0,057695\\
		\hline
		Encode Uniform Clegg: 0,0035 & (0,02831; 0,05921) & 0,04102 & 0,041420\\
		\hline
		Encode Uniform Clegg: 0,0045 & (0,0247; 0,06962) & 0,04197 & 0,037226\\
		\hline
		Encode Uniform Clegg: 0,0055 & (0,02211; 0,05691) & 0,03563 & 0,044593\\
		\hline
		Encode Uniform One-Fifth: 0,125 & (0,02138; 0,05172) & 0,03338 & 0,046835\\
		\hline
		Encode Uniform One-Fifth: 0,25 & (0,02075; 0,06059) & 0,03566 & 0,046254\\
		\hline
		Encode Uniform One-Fifth: 0,375 & (0,02667; 0,06216) & 0,04111 & 0,039581\\
		\hline
		Encode Uniform One-Fifth: 0,5 & (0,01608; 0,04588) & 0,02738 & 0,063355\\
		\hline
		Encode Uniform One-Fifth: 0,625 & (0,02459; 0,06074) & 0,03881 & 0,048598\\
		\hline
		Encode Uniform One-Fifth: 0,75 & (0,02107; 0,06167) & 0,03642 & 0,042721\\
		\hline
		Encode Uniform One-Fifth: 0,875 & (0,03182; 0,07718) & 0,04969 & 0,027637\\
	\end{tabular}
	\label{table:encodeUniformBayesian}
	\caption{Encode Uniform Rekombination: Bayes'sche Analyse}
\end{table}

\subsection{Bayes'sche Analyse: Koza}
\label{subsec:bayesKoza}

\begin{table}[H]
	\centering
	\begin{tabular}{c | c | c | c}
		\textbf{CPG-Konfiguration} & \textbf{HPDI (Iter.)} & \textbf{MW} & \textbf{PL-Platz}\\
		\hline
		Koza keine Rekombination & (3306,396; 9338,152) & 5573,228 & 0,066044\\
		\hline
		Koza One-Point Konstant: 0,125 & (11989,955; 39904,351) & 22076,286 & 0,039745\\
		\hline
		Koza One-Point Konstant: 0,25 & (5686,475; 17041,592) & 9857,749 & 0,053368\\
		\hline
		Koza One-Point Konstant: 0,375 & (7670,190; 19745,987) & 12308,128 & 0,041433\\
		\hline
		Koza One-Point Konstant: 0,5 & (21220,641; 77451,788) & 41120,883 & 0,030037\\
		\hline
		Koza One-Point Konstant: 0,625 & (8850,276; 22781,194) & 14213,974 & 0,042111\\
		\hline
		Koza One-Point Konstant: 0,75 & (6558,793; 17538,734) & 10732,756 & 0,050455\\
		\hline
		Koza One-Point Konstant: 0,875 & (22884,073; 83707,754) & 44778,187 & 0,034913\\
		\hline
		Koza One-Point Konstant: 1,0 & (11999,809; 40932,104) & 22223,655 & 0,045110\\
		\hline
		Koza One-Point Clegg: 0,0005 & (8804,961; 28842,331) & 16070,796 & 0,046374\\
		\hline
		Koza One-Point Clegg: 0,0015 & (12897,221; 44229,228) & 24132,461 & 0,034105\\
		\hline
		Koza One-Point Clegg: 0,0025 & (12097,518; 42109,886) & 22745,754 & 0,042530\\
		\hline
		Koza One-Point Clegg: 0,0035 & (10713,692; 36149,781) & 19751,179 & 0,041706\\
		\hline
		Koza One-Point Clegg: 0,0045 & (8152,673; 25944,800) & 14620,549 & 0,050148\\
		\hline
		Koza One-Point Clegg: 0,0055 & (10396,968; 32228,793) & 18312,576 & 0,045403\\
		\hline
		Koza One-Point One-Fifth: 0,125 & (7042,406; 22949,117) & 12782,824 & 0,055561\\
		\hline
		Koza One-Point One-Fifth: 0,25 & (8698,915; 24772,687) & 14733,520 & 0,046146\\
		\hline
		Koza One-Point One-Fifth: 0,375 & (5766,106; 16551,173) & 9801,340 & 0,052473\\
		\hline
		Koza One-Point One-Fifth: 0,5 & (10523,665; 39039,177) & 20390,489 & 0,048112\\
		\hline
		Koza One-Point One-Fifth: 0,625 & (7477,611; 22329,849) & 13015,454 & 0,043730\\
		\hline
		Koza One-Point One-Fifth: 0,75 & (9605,964; 28910,079) & 16579,745 & 0,046481\\
		\hline
		Koza One-Point One-Fifth: 0,875 & (7844,463; 27078,528) & 14728,439 & 0,044015\\
	\end{tabular}
	\label{table:kozaOnePointBayesian}
	\caption{Koza One-Point Rekombination: Bayes'sche Analyse}
\end{table}

\begin{table}[H]
	\centering
	\begin{tabular}{c | c | c | c}
		\textbf{CPG-Konfiguration} & \textbf{HPDI (Iter.)} & \textbf{MW} & \textbf{PL-Platz}\\
		\hline
		Koza keine Rekombination & (3306,396; 9338,152) & 5573,228 & 0,044276\\
		\hline
		Koza Two-Point Konstant: 0,125 & (5778,032; 20659,534) & 11104,449 & 0,050079\\
		\hline
		Koza Two-Point Konstant: 0,25 & (4377,352; 12699,617) & 7467,417 & 0,043383\\
		\hline
		Koza Two-Point Konstant: 0,375 & (7682,846; 29668,729) & 15237,451 & 0,042903\\
		\hline
		Koza Two-Point Konstant: 0,5 & (4545,700; 15211,867) & 8448,030 & 0,047157\\
		\hline
		Koza Two-Point Konstant: 0,625 & (2506,268; 7690,605) & 4414,865 & 0,052680\\
		\hline
		Koza Two-Point Konstant: 0,75 & (8106,535; 30489,526) & 15831,484 & 0,036253\\
		\hline
		Koza Two-Point Konstant: 0,875 & (5275,995; 16141,930) & 9274,253 & 0,039958\\
		\hline
		Koza Two-Point Konstant: 1,0 & (2467,382; 7467,193) & 4300,790 & 0,058410\\
		\hline
		Koza Two-Point Clegg: 0,0005 & (3539,092; 10910,749) & 6214,133 & 0,046851\\
		\hline
		Koza Two-Point Clegg: 0,0015 & (5136,595; 18389,726) & 9803,362 & 0,040274\\
		\hline
		Koza Two-Point Clegg: 0,0025 & (7068,983; 23388,559) & 12866,463 & 0,039712\\
		\hline
		Koza Two-Point Clegg: 0,0035 & (4920,069; 17123,138) & 9196,381 & 0,052770\\
		\hline
		Koza Two-Point Clegg: 0,0045 & (5065,402; 17637,588) & 9552,405 & 0,049098\\
		\hline
		Koza Two-Point Clegg: 0,0055 & (2005,714; 5792,200) & 3420,063 & 0,059170\\
		\hline
		Koza Two-Point One-Fifth: 0,125 & (5549,634; 19765,485) & 10495,382 & 0,042831\\
		\hline
		Koza Two-Point One-Fifth: 0,25 & (8535,956; 28712,980) & 15794,048 & 0,033087\\
		\hline
		Koza Two-Point One-Fifth: 0,375 & (6383,072; 19877,145) & 11333,015 & 0,033814\\
		\hline
		Koza Two-Point One-Fifth: 0,5 & (4715,287; 13037,660) & 7854,211 & 0,034314\\
		\hline
		Koza Two-Point One-Fifth: 0,625 & (3209,936; 9921,136) & 5663,114 & 0,051706\\
		\hline
		Koza Two-Point One-Fifth: 0,75 & (4615,524; 15765,276) & 8647,855 & 0,047033\\
		\hline
		Koza Two-Point One-Fifth: 0,875 & (2480,371; 7460,089) & 4313,097 & 0,054242\\
	\end{tabular}
	\label{table:kozaTwoPointBayesian}
	\caption{Koza Two-Point Rekombination: Bayes'sche Analyse}
\end{table}


\begin{table}[H]
	\centering
	\begin{tabular}{c | c | c | c}
		\textbf{CPG-Konfiguration} & \textbf{HPDI (Iter.)} & \textbf{MW} & \textbf{PL-Platz}\\
		\hline
		Koza keine Rekombination & (3306,396; 9338,152) & 5573,228 & 0,046653\\
		\hline
		Koza Uniform Konstant: 0,125 & (7116,082; 25435,336) & 13593,719 & 0,038759\\
		\hline
		Koza Uniform Konstant: 0,25 & (9984,851; 33237,558) & 18247,622 & 0,032483\\
		\hline
		Koza Uniform Konstant: 0,375 & (3172,788; 9799,001) & 5586,763 & 0,054234\\
		\hline
		Koza Uniform Konstant: 0,5 & (5845,776; 21080,773) & 11192,410 & 0,048150\\
		\hline
		Koza Uniform Konstant: 0,625 & (5056,513; 15161,941) & 8855,128 & 0,044384\\
		\hline
		Koza Uniform Konstant: 0,75 & (7678,984; 28599,746) & 14989,121 & 0,042284\\
		\hline
		Koza Uniform Konstant: 0,875 & (8075,534; 25962,001) & 14529,689 & 0,037376\\
		\hline
		Koza Uniform Konstant: 1,0 & (6997,916; 26553,700) & 13799,512 & 0,050811\\
		\hline
		Koza Uniform Clegg: 0,0005 & (5887,321; 17725,061) & 10263,6346 & 0,039935\\
		\hline
		Koza Uniform Clegg: 0,0015 & (1820,854; 5880,233) & 3284,824 & 0,070103\\
		\hline
		Koza Uniform Clegg: 0,0025 & (2105,563; 6348,965) & 3669,648 & 0,062192\\
		\hline
		Koza Uniform Clegg: 0,0035 & (6267,561; 20173,041) & 11336,109 & 0,042816\\
		\hline
		Koza Uniform Clegg: 0,0045 & (3259,699; 8595,464) & 5275,119 & 0,045873\\
		\hline
		Koza Uniform Clegg: 0,0055 & (4447,802; 15210,844) & 8263,146 & 0,047253\\
		\hline
		Koza Uniform One-Fifth: 0,125 & (9555,821; 36197,611) & 18956,129 & 0,041516\\
		\hline
		Koza Uniform One-Fifth: 0,25 & (6369,195; 24041,614) & 12519,802 & 0,045399\\
		\hline
		Koza Uniform One-Fifth: 0,375 & (4284,718; 13099,952) & 7548,330 & 0,045936\\
		\hline
		Koza Uniform One-Fifth: 0,5 & (6205,669; 18180,546) & 10642,776 & 0,039174\\
		\hline
		Koza Uniform One-Fifth: 0,625 & (5184,311; 15589,695) & 9047,982 & 0,039264\\
		\hline
		Koza Uniform One-Fifth: 0,75 & (7626,548; 26532,940) & 14283,760 & 0,040292\\
		\hline
		Koza Uniform One-Fifth: 0,875 & (4522,053; 14978,774) & 8243,582 & 0,045112\\
	\end{tabular}
	\label{table:kozaUniformBayesian}
	\caption{Koza Uniform Rekombination: Bayes'sche Analyse}
\end{table}

\subsection{Bayes'sche Analyse: Zusammenfassung}
\label{subsec:bayesZusammenfassung}

\section{Ergebnisse Graphische Evaluation}
\label{sec:ergebnissePlots}

\subsection{Graphische Evaluation: Parity}
\label{subsec:plotsParity}

\subsection{Graphische Evaluation: Keijzer}
\label{subsec:plotsKeijzer}

\subsection{Graphische Evaluation: Encode}
\label{subsec:plotsEncode}

\subsection{Graphische Evaluation: Koza}
\label{subsec:plotsKoza}

\subsection{Graphische Evaluation: Zusammenfassung}
\label{subsec:plotsZusammenfassung}

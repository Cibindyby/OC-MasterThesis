\chapter{Ergebnisse}
\label{Ergebnisse}
Ergebnisse

TODO: Argumentation, dass Offset nicht so viel bringt (damit Kapitel \glqq praktischer Teil\grqq\space darauf verweisen kann)\\
- HPO: mehrere Durchläufe wählen garkeinen Offset\\
- restliche Tests vergleichen zwischen mit Offset und ohne\\
- Rohdatenanalyse: Rekombinationserfolge, wenn kein Offset da ist

\section{Ergebnisse Rohdatenanalyse}
\label{sec:ergebnisseRohdaten}

\subsection{Rohdatenanalyse: Parity}
\label{subsec:rohdatenParity}

\begin{table}[H]
	\centering
	\begin{tabular}{c | c | c | c | c | c | c}
		\begin{turn}{270} \textbf{CGP-Konfigurationen} \end{turn} & \begin{turn}{270} \textbf{Anzahl Rechenknoten} \end{turn} & \begin{turn}{270} \textbf{Populationsgröße} \end{turn} & \begin{turn}{270} \textbf{Start-Rekombinationsrate} \end{turn} & \begin{turn}{270} \textbf{Delta Rekombinationsrate} \end{turn} & \begin{turn}{270} \textbf{Elitisten} \end{turn} & \begin{turn}{270} \textbf{Offset} \end{turn}\\
		\hline
		keine Rekombination & 1000 & 20 & - & - & 18 & -\\
		\hline
		One-Point Konstant kein Offset & 1350 & 52 & 0,30 & - & 16 & - \\
		\hline
		One-Point Konstant mit Offset & 1500 & 50 & 0,8 & - & 18 & 25 \\
		\hline
		One-Point Clegg kein Offset & 1950 & 42 & - & 0,035 & 14 & - \\
		\hline
		One-Point Clegg mit Offset & 1600 & 58 & - & 0,01 & 20 & 25 \\
		\hline
		One-Point One-Fifth kein Offset & 1150 & 46 & 0,55 & - & 6 & - \\
		\hline
		One-Point One-Fifth mit Offset & 1100 & 54 & 0,35 & - & 20 & 30 \\
		\hline
		Two-Point Konstant kein Offset & 1900 & 58 & 0,4 & - & 20 & - \\
		\hline
		Two-Point Konstant mit Offset & 850 & 54 & 0,3 & - & 16 & 15 \\
		\hline
		Two-Point Clegg kein Offset & 750 & 56 & - & 0,02 & 18 & - \\
		\hline
		Two-Point Clegg mit Offset & 1700 & 28 & - & 0,04 & 12 & 20 \\
		\hline
		Two-Point One-Fifth kein Offset & 1800 & 44 & 0,55 & - & 16 & - \\
		\hline
		Two-Point One-Fifth mit Offset & 950 & 58 & 0,7 & - & 10 & 15 \\
		\hline
		Uniform Konstant kein Offset & 1800 & 58 & 0,8 & - & 20 & - \\
		\hline
		Uniform Konstant mit Offset & 1400 & 54 & 0,3 & - & 20 & 45 \\
		\hline
		Uniform Clegg kein Offset & 1600 & 50 & - & 0,015 & 20 & - \\
		\hline
		Uniform Clegg mit Offset & 1500 & 52 & - & 0,04 & 18 & 25 \\
		\hline
		Uniform One-Fifth kein Offset & 650 & 52 & 0,75 & - & 14 & - \\
		\hline
		Uniform One-Fifth mit Offset & 250 & 50 & 0,35 & - & 16 & 40 \\
	\end{tabular}
	\caption{Parity: Ergebnis Hyperparameteranalyse}
	\label{table:parityHPO}
\end{table}

TODO: Tabellen erklären (Spaltennamen, etc)\\
TODO: für HPO beschreiben welche Daten wie ansonsten gefüllt werden, falls leer (Start Rekombinationsrate für Clegg)\\
TODO: beschreiben, dass mehrere Offsets als (ohne Offset gesetzt wurden)

\begin{table}[H]
	\centering
	\begin{tabular}{c | c | c | c | c | c }
		\begin{turn}{270} \textbf{CGP-Konfigurationen} \end{turn} & \begin{turn}{270} \textbf{Anzahl pos. Mutationen} \end{turn} & \begin{turn}{270} \textbf{Anzahl pos. Rekomb.} \end{turn} & \begin{turn}{270} \textbf{Anzahl neg. Mutationen} \end{turn} & \begin{turn}{270} \textbf{Median Iter. pos. Rekomb.} \end{turn} & \begin{turn}{270} \textbf{Median Iter. bis Konv.} \end{turn}\\
		\hline
		keine Rekombination & 133 & 0 & 0 & - & 58,5\\
		\hline
		One-Point Konstant kein Offset & 126 & 8 & 2 & 5,5 & 44\\
		\hline
		One-Point Konstant mit Offset & 136 & 0 & 0 & - & 49\\
		\hline
		One-Point Clegg kein Offset & 123 & 13 & 3 & 4 & 74\\
		\hline
		One-Point Clegg mit Offset & 126 & 0 & 0 & - & 43,5\\
		\hline
		One-Point One-Fifth kein Offset & 116 & 2 & 1 & 2 & 37\\
		\hline
		One-Point One-Fifth mit Offset & 127 & 0 & 0 & - & 27,5\\
		\hline
		Two-Point Konstant kein Offset & 108 & 10 & 2 & 3 & 33,5\\
		\hline
		Two-Point Konstant mit Offset & 129 & 0 & 0 & - & 64\\
		\hline
		Two-Point Clegg kein Offset & 106 & 20 & 8 & 6,5 & 40\\
		\hline
		Two-Point Clegg mit Offset & 125 & 0 & 0 & - & 70,5\\
		\hline
		Two-Point One-Fifth kein Offset & 126 & 4 & 2 & 3 & 59,5\\
		\hline
		Two-Point One-Fifth mit Offset & 125 & 0 & 0 & - & 43\\
		\hline
		Uniform Konstant kein Offset & 85 & 44 & 15 & 14 & 55\\
		\hline
		Uniform Konstant mit Offset & 131 & 0 & 0 & - & 37,5\\
		\hline
		Uniform Clegg kein Offset & 101 & 41 & 15 & 4 & 31\\
		\hline
		Uniform Clegg mit Offset & 130 & 0 & 0 & - & 51,5\\
		\hline
		Uniform One-Fifth kein Offset & 108 & 22 & 12 & 5 & 69,5\\
		\hline
		Uniform One-Fifth mit Offset & 123 & 0 & 0 & - & 54\\
	\end{tabular}
	\caption{Parity: Auswertung der Rohdaten}
	\label{table:parityRohdaten}
\end{table}

\subsection{Rohdatenanalyse: Keijzer}
\label{subsec:rohdatenKeijzer}

\begin{table}[H]
	\centering
	\begin{tabular}{c | c | c | c | c | c | c}
		\begin{turn}{270} \textbf{CGP-Konfigurationen} \end{turn} & \begin{turn}{270} \textbf{Anzahl Rechenknoten} \end{turn} & \begin{turn}{270} \textbf{Populationsgröße} \end{turn} & \begin{turn}{270} \textbf{Start-Rekombinationsrate} \end{turn} & \begin{turn}{270} \textbf{Delta Rekombinationsrate} \end{turn} & \begin{turn}{270} \textbf{Elitisten} \end{turn} & \begin{turn}{270} \textbf{Offset} \end{turn}\\
		\hline
		keine Rekombination & 1850 & 44 & - & - & 20 & -\\
		\hline
		One-Point Konstant kein Offset & 600 & 50 & 0,2 & - & 20 & - \\
		\hline
		One-Point Konstant mit Offset & 1200 & 50 & 1 & - & 4 & 120 \\
		\hline
		One-Point Clegg kein Offset & 1100 & 60 & - & 0,05 & 16 & - \\
		\hline
		One-Point Clegg mit Offset & 850 & 60 & - & 0,005 & 16 & 300 \\
		\hline
		One-Point One-Fifth kein Offset & 350 & 34 & 0,75 & - & 18 & - \\
		\hline
		One-Point One-Fifth mit Offset & 2000 & 28 & 0,65 & - & 18 & 180 \\
		\hline
		Two-Point Konstant kein Offset & 1700 & 60 & 0,5 & - & 8 & - \\
		\hline
		Two-Point Konstant mit Offset & 600 & 48 & 0,3 & - & 16 & 180 \\
		\hline
		Two-Point Clegg kein Offset & 750 & 34 & - & 0,005 & 14 & - \\
		\hline
		Two-Point Clegg mit Offset & 800 & 36 & - & 0,045 & 20 & 210 \\
		\hline
		Two-Point One-Fifth kein Offset & 1350 & 40 & 0,35 & - & 20 & - \\
		\hline
		Two-Point One-Fifth mit Offset & 700 & 52 & 0,7 & - & 8 & 30 \\
		\hline
		Uniform Konstant kein Offset & 1100 & 52 & 0,8 & - & 8 & - \\
		\hline
		Uniform Konstant mit Offset & 1800 & 60 & 0,1 & - & 20 & 90 \\
		\hline
		Uniform Clegg kein Offset & 850 & 26 & - & 0,05 & 18 & - \\
		\hline
		Uniform Clegg mit Offset & 2000 & 44 & - & 0,05 & 14 & 180 \\
		\hline
		Uniform One-Fifth kein Offset & 1150 & 54 & 0,75 & - & 18 & - \\
		\hline
		Uniform One-Fifth mit Offset & 900 & 48 & 0,35 & - & 20 & 90 \\
	\end{tabular}
	\caption{Keijzer: Ergebnis Hyperparameteranalyse}
	\label{table:keijzerHPO}
\end{table}

TODO: Tabellen erklären (Spaltennamen, etc)\\
TODO: für HPO beschreiben welche Daten wie ansonsten gefüllt werden, falls leer (Start Rekombinationsrate für Clegg)\\
TODO: beschreiben, dass mehrere Offsets als (ohne Offset gesetzt wurden)

\begin{table}[H]
	\centering
	\begin{tabular}{c | c | c | c | c | c }
		\begin{turn}{270} \textbf{CGP-Konfigurationen} \end{turn} & \begin{turn}{270} \textbf{Anzahl pos. Mutationen} \end{turn} & \begin{turn}{270} \textbf{Anzahl pos. Rekomb.} \end{turn} & \begin{turn}{270} \textbf{Anzahl neg. Mutationen} \end{turn} & \begin{turn}{270} \textbf{Median Iter. pos. Rekomb.} \end{turn} & \begin{turn}{270} \textbf{Median Iter. bis Konv.} \end{turn}\\
		\hline
		keine Rekombination & 2260 & 0 & 0 & - & 624\\
		\hline
		One-Point Konstant kein Offset & 1183 & 26 & 16 & 13 & 756\\
		\hline
		One-Point Konstant mit Offset & 1531 & 0 & 0 & - & 374\\
		\hline
		One-Point Clegg kein Offset & 1584 & 59 & 28 & 5 & 243\\
		\hline
		One-Point Clegg mit Offset & 1534 & 0 & 0 & - & 396,5\\
		\hline
		One-Point One-Fifth kein Offset & 1307 & 35 & 18 & 3 & 766\\
		\hline
		One-Point One-Fifth mit Offset & 1776 & 0 & 0 & - & 229\\
		\hline
		Two-Point Konstant kein Offset & 1854 & 79 & 25 & 15 & 803\\
		\hline
		Two-Point Konstant mit Offset & 1361 & 0 & 0 & - & 548\\
		\hline
		Two-Point Clegg kein Offset & 1467 & 106 & 62 & 18,5 & 735\\
		\hline
		Two-Point Clegg mit Offset & 1862 & 0 & 0 & - & 201\\
		\hline
		Two-Point One-Fifth kein Offset & 1788 & 17 & 10 & 4 & 753\\
		\hline
		Two-Point One-Fifth mit Offset & 1124 & 0 & 0 & - & 158\\
		\hline
		Uniform Konstant kein Offset & 1224 & 480 & 239 & 24 & 363\\
		\hline
		Uniform Konstant mit Offset & 1890 & 0 & 0 & - & 260,5\\
		\hline
		Uniform Clegg kein Offset & 1386 & 77 & 57 & 6 & 723\\
		\hline
		Uniform Clegg mit Offset & 2232 & 0 & 0 & - & 385,5\\
		\hline
		Uniform One-Fifth kein Offset & 1491 & 106 & 50 & 8 & 155\\
		\hline
		Uniform One-Fifth mit Offset & 1749 & 0 & 0 & - & 445,5\\
	\end{tabular}
	\caption{Keijzer: Auswertung der Rohdaten}
	\label{table:keijzerRohdaten}
\end{table}

\subsection{Rohdatenanalyse: Encode}
\label{subsec:rohdatenEncode}

\begin{table}[H]
	\centering
	\begin{tabular}{c | c | c | c | c | c | c}
		\begin{turn}{270} \textbf{CGP-Konfigurationen} \end{turn} & \begin{turn}{270} \textbf{Anzahl pos. Mutationen} \end{turn} & \begin{turn}{270} \textbf{Anzahl pos. Rekomb.} \end{turn} & \begin{turn}{270} \textbf{Anzahl neg. Mutationen} \end{turn} & \begin{turn}{270} \textbf{Median Iter. pos. Rekomb.} \end{turn} & \begin{turn}{270} \textbf{Median Iter. bis Konv.} \end{turn} & \begin{turn}{270} \textbf{Stopp-Kriterium erfüllt} \end{turn}\\
		\hline
		One-Point Konstant: 0,125 & 1126 & 20 & 4 & 9.5 & 3362 & 9\\
		\hline
		One-Point Konstant: 0,25 & 1100 & 42 & 5 & 12.5 & 1963 & 9\\
		\hline
		One-Point Konstant: 0,375 & 1120 & 43 & 5 & 6 & 4578.5 & 8\\
		\hline
		One-Point Konstant: 0,5 & 1104 & 34 & 2 & 10.0 & 936 & 9\\
		\hline
		One-Point Konstant: 0,625 & 1096 & 45 & 4 & 12 & 3293.0 & 6\\
		\hline
		One-Point Konstant: 0,75 & 1103 & 55 & 11 & 16 & 1898.5 & 6\\
		\hline
		One-Point Konstant: 0,875 & 1093 & 59 & 6 & 15 & 1913.0 & 10\\
		\hline
		One-Point Konstant: 1,0 & 1086 & 58 & 9 & 8.0 & 3754.5 & 8\\
		\hline
		One-Point Clegg: 0,0005 & 1125 & 48 & 6 & 11.0 & 1905.5 & 12\\
		\hline
		One-Point Clegg: 0,0015 & 1062 & 51 & 5 & 12 & 2754.5 & 6\\
		\hline
		One-Point Clegg: 0,0025 & 1063 & 39 & 2 & 9 & 1530 & 5\\
		\hline
		One-Point Clegg: 0,0035 & 1114 & 34 & 2 & 10.5 & 2840.5 & 6\\
		\hline
		One-Point Clegg: 0,0045 & 1127 & 38 & 2 & 10.0 & 2558 & 9\\
		\hline
		One-Point Clegg: 0,0055 & 1079 & 55 & 7 & 9 & 4007.5 & 8\\
		\hline
		One-Point One-Fifth: 0,125 & 1138 & 19 & 1 & 7 & 3950 & 9\\
		\hline
		One-Point One-Fifth: 0,25 & 1135 & 22 & 3 & 7.0 & 4111.5 & 12\\
		\hline
		One-Point One-Fifth: 0,375 & 1155 & 20 & 1 & 4.0 & 1685 & 7\\
		\hline
		One-Point One-Fifth: 0,5 & 1123 & 26 & 0 & 5.0 & 3205 & 11\\
		\hline
		One-Point One-Fifth: 0,625 & 1100 & 35 & 3 & 6 & 3809 & 11\\
		\hline
		One-Point One-Fifth: 0,75 & 1089 & 25 & 1 & 6 & 795 & 7\\
		\hline
		One-Point One-Fifth: 0,875 & 1126 & 32 & 5 & 8.0 & 5213 & 5\\
		\hline
		One-Point One-Fifth: 1,0 & 1112 & 42 & 3 & 5.5 & 2863 & 5\\
	\end{tabular}
	\caption{Encode One-Point Rekombination: Auswertung der Rohdaten}
	\label{table:encodeOnePointRohdaten}
\end{table}

\begin{table}[H]
	\centering
	\begin{tabular}{c | c | c | c | c | c | c}
		\begin{turn}{270} \textbf{CGP-Konfigurationen} \end{turn} & \begin{turn}{270} \textbf{Anzahl pos. Mutationen} \end{turn} & \begin{turn}{270} \textbf{Anzahl pos. Rekomb.} \end{turn} & \begin{turn}{270} \textbf{Anzahl neg. Mutationen} \end{turn} & \begin{turn}{270} \textbf{Median Iter. pos. Rekomb.} \end{turn} & \begin{turn}{270} \textbf{Median Iter. bis Konv.} \end{turn} & \begin{turn}{270} \textbf{Stopp-Kriterium erfüllt} \end{turn}\\
		\hline
		Two-Point Konstant: 0,125 & 1044 & 27 & 2 & 6 & 4039.5 & 2\\
		\hline
		Two-Point Konstant: 0,25 & 1062 & 35 & 9 & 7 & 4292 & 3\\
		\hline
		Two-Point Konstant: 0,375 & 1076 & 45 & 4 & 5 & 1840 & 5\\
		\hline
		Two-Point Konstant: 0,5 & 1041 & 50 & 12 & 7.5 & 1666 & 3\\
		\hline
		Two-Point Konstant: 0,625 & 1050 & 61 & 15 & 7 & 7370 & 3\\
		\hline
		Two-Point Konstant: 0,75 & 1019 & 61 & 14 & 10 & 3138 & 1\\
		\hline
		Two-Point Konstant: 0,875 & 1023 & 73 & 12 & 9 & 4445 & 5\\
		\hline
		Two-Point Konstant: 1,0 & 1031 & 65 & 10 & 5 & 3476.5 & 6\\
		\hline
		Two-Point Clegg: 0,0005 & 1041 & 70 & 7 & 6.0 & 3383 & 5\\
		\hline
		Two-Point Clegg: 0,0015 & 1015 & 53 & 8 & 8 & 2524.5 & 2\\
		\hline
		Two-Point Clegg: 0,0025 & 1026 & 75 & 8 & 9 & 6024 & 3\\
		\hline
		Two-Point Clegg: 0,0035 & 977 & 64 & 9 & 7.0 & 2061 & 3\\
		\hline
		Two-Point Clegg: 0,0045 & 1046 & 51 & 9 & 8 & 2616 & 5\\
		\hline
		Two-Point Clegg: 0,0055 & 1019 & 52 & 8 & 9.5 & 3475 & 7\\
		\hline
		Two-Point One-Fifth: 0,125 & 1067 & 10 & 3 & 5.0 & 575.0 & 2\\
		\hline
		Two-Point One-Fifth: 0,25 & 1077 & 16 & 4 & 4.0 & 707 & 1\\
		\hline
		Two-Point One-Fifth: 0,375 & 1067 & 17 & 1 & 4 & 7538 & 1\\
		\hline
		Two-Point One-Fifth: 0,5 & 1063 & 24 & 5 & 5.0 & 3363 & 3\\
		\hline
		Two-Point One-Fifth: 0,625 & 1075 & 32 & 6 & 4.0 & 4919 & 5\\
		\hline
		Two-Point One-Fifth: 0,75 & 1078 & 34 & 8 & 4.0 & 1312.0 & 6\\
		\hline
		Two-Point One-Fifth: 0,875 & 1088 & 36 & 4 & 5.0 & 2529.5 & 6\\
		\hline
		Two-Point One-Fifth: 1,0 & 1049 & 43 & 7 & 4 & 1140 & 5\\
	\end{tabular}
	\caption{Encode Two-Point Rekombination: Auswertung der Rohdaten}
	\label{table:encodeTwoPointRohdaten}
\end{table}

\begin{table}[H]
	\centering
	\begin{tabular}{c | c | c | c | c | c | c}
		\begin{turn}{270} \textbf{CGP-Konfigurationen} \end{turn} & \begin{turn}{270} \textbf{Anzahl pos. Mutationen} \end{turn} & \begin{turn}{270} \textbf{Anzahl pos. Rekomb.} \end{turn} & \begin{turn}{270} \textbf{Anzahl neg. Mutationen} \end{turn} & \begin{turn}{270} \textbf{Median Iter. pos. Rekomb.} \end{turn} & \begin{turn}{270} \textbf{Median Iter. bis Konv.} \end{turn} & \begin{turn}{270} \textbf{Stopp-Kriterium erfüllt} \end{turn}\\
		\hline
		Uniform Konstant: 0,125 & 1105 & 52 & 12 & 7.0 & 4058 & 9\\
		\hline
		Uniform Konstant: 0,25 & 1099 & 69 & 3 & 13 & 3831.5 & 10\\
		\hline
		Uniform Konstant: 0,375 & 1076 & 64 & 10 & 9.0 & 4071.5 & 6\\
		\hline
		Uniform Konstant: 0,5 & 1105 & 68 & 2 & 12.5 & 993.5 & 8\\
		\hline
		Uniform Konstant: 0,625 & 1099 & 78 & 3 & 9.0 & 1306.5 & 6\\
		\hline
		Uniform Konstant: 0,75 & 1085 & 81 & 6 & 10 & 3459.0 & 8\\
		\hline
		Uniform Konstant: 0,875 & 1088 & 83 & 1 & 10 & 2490.5 & 6\\
		\hline
		Uniform Konstant: 1,0 & 1081 & 67 & 5 & 10 & 4785 & 9\\
		\hline
		Uniform Clegg: 0,0005 & 1092 & 76 & 5 & 8.0 & 2562 & 13\\
		\hline
		Uniform Clegg: 0,0015 & 1099 & 78 & 2 & 10.5 & 1777 & 11\\
		\hline
		Uniform Clegg: 0,0025 & 1096 & 74 & 7 & 11.0 & 2401.5 & 10\\
		\hline
		Uniform Clegg: 0,0035 & 1095 & 56 & 4 & 12.0 & 4674.5 & 4\\
		\hline
		Uniform Clegg: 0,0045 & 1080 & 70 & 2 & 13.0 & 5118.0 & 10\\
		\hline
		Uniform Clegg: 0,0055 & 1122 & 68 & 10 & 9.5 & 4044.5 & 8\\
		\hline
		Uniform One-Fifth: 0,125 & 1152 & 18 & 2 & 6.5 & 3876 & 7\\
		\hline
		Uniform One-Fifth: 0,25 & 1126 & 25 & 6 & 6 & 3393 & 11\\
		\hline
		Uniform One-Fifth: 0,375 & 1128 & 29 & 2 & 8 & 988.5 & 6\\
		\hline
		Uniform One-Fifth: 0,5 & 1156 & 44 & 5 & 7.0 & 3968 & 11\\
		\hline
		Uniform One-Fifth: 0,625 & 1122 & 40 & 4 & 8.0 & 1781 & 7\\
		\hline
		Uniform One-Fifth: 0,75 & 1104 & 47 & 4 & 6 & 3385.5 & 12\\
		\hline
		Uniform One-Fifth: 0,875 & 1075 & 51 & 10 & 10 & 1831.5 & 6\\
		\hline
		Uniform One-Fifth: 1,0 & 1129 & 55 & 7 & 6 & 4972 & 9\\
	\end{tabular}
	\caption{Encode Uniform Rekombination: Auswertung der Rohdaten}
	\label{table:encodeUniformRohdaten}
\end{table}

Encode ohne Rekombination:
\begin{itemize}
	\item Median Iterationen bis Konvergenz: 3847,5
	\item Stopp-Kriterium erfüllt: 8
\end{itemize}


\subsection{Rohdatenanalyse: Koza}
\label{subsec:rohdatenKoza}

Koza ohne Rekombination:
\begin{itemize}
	\item Median Iterationen bis Konvergenz: 208
	\item Stopp-Kriterium erfüllt: 48
\end{itemize}


\subsection{Rohdatenanalyse: Zusammenfassung}
\label{subsec:rohdatenZusammenfassung}

\section{Ergebnisse Bayes'sche Analyse}
\label{sec:ergebnisseBayes}

\subsection{Bayes'sche Analyse: Parity}
\label{subsec:bayesParity}

\subsection{Bayes'sche Analyse: Keijzer}
\label{subsec:bayesKeijzer}

\subsection{Bayes'sche Analyse: Encode}
\label{subsec:bayesEncode}

\subsection{Bayes'sche Analyse: Koza}
\label{subsec:bayesKoza}

\subsection{Bayes'sche Analyse: Zusammenfassung}
\label{subsec:bayesZusammenfassung}

\section{Ergebnisse Graphische Evaluation}
\label{sec:ergebnissePlots}

\subsection{Graphische Evaluation: Parity}
\label{subsec:plotsParity}

\subsection{Graphische Evaluation: Keijzer}
\label{subsec:plotsKeijzer}

\subsection{Graphische Evaluation: Encode}
\label{subsec:plotsEncode}

\subsection{Graphische Evaluation: Koza}
\label{subsec:plotsKoza}

\subsection{Graphische Evaluation: Zusammenfassung}
\label{subsec:plotsZusammenfassung}

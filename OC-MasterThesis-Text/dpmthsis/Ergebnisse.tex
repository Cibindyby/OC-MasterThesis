\chapter{Ergebnisse}
\label{Ergebnisse}

In den folgenden Abschnitten werden die Ergebnisse dieser Arbeit aufgezeigt und analysiert.
Sie sind unterteilt in die beiden Evaluations-Strategien, die in Abschnitt \ref{sec:Evaluation} erläutert wurden: Analyse der Rohdaten und Bayes'sche Analyse.
Zur Vereinfachung wird in den Ergebnistabellen die konstante Rekombinationsrate kurz \glqq Konstant\grqq\space genannt, die linear fallende Rekombinationsrate \glqq Clegg \grqq\space und die Rekombinationsrate mit der One-Fifth Regel \glqq One-Fifth\grqq.

\section{Ergebnisse Rohdatenanalyse}
\label{sec:ergebnisseRohdaten}

Die Analyse der Rohdaten besteht aus zwei Teilen.
Die Ergebnisse der Hyperparameteranalyse werden für Parity und Keijzer näher betrachtet.
Es werden dabei folgende Werte für jede CGP-Konfiguration aufgezeigt:
\begin{itemize}
	\item \textbf{Anzahl der Rechenknoten:} Anzahl der Rechenknoten in jedem Chromosom des CGP-Modells
	\item \textbf{$\lambda$ (Nachwuchs):} Anzahl der Nachwuchs-Chromosomen in jeder Generation; $\lambda$-Wert in der ($\mu$+$\lambda$)-ES
	\item \textbf{Start-Rekombinationsrate:} Rekombinationsrate bei der Initialisierung; nur relevant für die konstante Rekombinationsrate und die One-Fifth Regel (linear fallende Rekombinationsrate wird mit 0,9 initialisiert)
	\item \textbf{Delta Rekombinationsrate:} Dieser Wert wird bei linear fallender Rekombinationsrate in jeder Generation von der bisherigen Rekombinationsrate abgezogen.
	\item \textbf{$\mu$ (Elitisten):} Anzahl der Elitisten in jeder Generation; $\mu$-Wert in der ($\mu$+$\lambda$)-ES
	\item \textbf{Offset:} Gibt an in wie vielen Trainings-Iterationen nach Initialisierung des CGP-Modells keine Rekombination ausgeführt wird.
\end{itemize}

Des Weiteren werden für alle Testszenarien die Ergebnisse des CGP-Trainings näher betrachtet.
Dabei werden folgende Metriken für jede CGP-Konfiguration aufgezeigt:
\begin{itemize}
	\item \textbf{Anzahl positive Mutationen:} Gibt an wie häufig Mutation zu einer Verbesserung der Fitness geführt hat (summiert über alle 50 Durchläufe).
	\item \textbf{Anzahl positive Rekombination:} Anzahl der Rekombinationsschritte, die die Fitness verbessert haben (summiert über alle 50 Durchläufe)
	\item \textbf{Anzahl negative Mutationen:} Anzahl der Mutationsschritte, bei denen der vorherige Rekombinationsschritt eine Verbesserung der Fitness erzielt hat, die durch Mutation reduziert oder vollständig aufgehoben wurde (summiert über alle 50 Durchläufe)
	\item \textbf{Median Iterationen positive Rekombination:} Median der Iterationen, bei denen der Rekombinationsschritt zur Verbesserung der Fitness beigetragen hat (über 50 Durchgänge hinweg)
	\item \textbf{Median Iterationen bis Konvergenz:} Median der Iterationen bis das Stopp-Kri\-te\-ri\-um erfüllt wurde (über 50 Durchläufe hinweg)
	\item \textbf{Stopp-Kriterium erfüllt:} Gibt an wie häufig das Stopp-Kriterium bei 50 Durchläufen erfüllt wurde. Dieser Wert wird verwendet, um Entscheidungen darüber zu treffen, welche Metrik in der Bayes'schen Analyse betrachtet werden soll (für die komplexeren Testszenarien).
\end{itemize}

\subsection{Rohdatenanalyse: Parity}
\label{subsec:rohdatenParity}
\subsubsection{Analyse der Hyperparameteroptimierung}

Die Tabelle \ref{table:parityHPO} zeigt die Hyperparameter, die für die Ausführung des CGP-Trainings für Parity verwendet werden.
Dabei werden die Ergebnisse der Hyperparameteranalyse verwendet, um die effizientesten CGP-Konfigurationen miteinander zu vergleichen.
Zu beachten ist, dass bei der Optimierung des Rekombinations-Offsets dem Optimierer ebenfalls die Möglichkeit gegeben wurde den Offset auf 0 zu stellen und somit auszuschalten.
Diese Möglichkeit wurde bei 2 aus 9 CGP-Konfigurationen genutzt.
Für diese Fälle wurden die nächstbesten Hyperparameter aus der Hyperparameteroptimierung gewählt, die einen Offset enthalten.
Die jeweiligen Offset-Werte sind in der Tabelle rot markiert.
Alle Beobachtungen in diesem Abschnitt beziehen sich ausschließlich auf das Parity-Testszenario.
Basierend auf diesen Daten lassen sich demnach keine allgemeingültigen Schlüsse ziehen und alle Aussagen müssen kritisch begutachtet werden.
\paragraph{Anzahl der Rechenknoten:}
Betrachtet wird zuerst die Anzahl der Rechenknoten aus Tabelle \ref{table:parityHPO}. 
Zu beobachten ist hier, dass die CGP-Konfiguration ohne Rekombination weniger Rechenknoten benötigt als der Mittelwert aller Verfahren von ca. 1305 Rechenknoten pro Chromosom.
Die Modelle, die die One-Point Rekombination verwenden, weisen mit ca 1442 den höchsten Mittelwert an Rechenknoten auf.
Mit durchschnittlich 1350 Rechenknoten haben die Modelle mit Two-Point Rekombination ein wenig mehr Rechenknoten benötigt als der Durchschnitt.
Mit 1200 Rechenknoten im Mittel haben Modelle, welche die Uniform Rekombination verwenden weniger Rechenknoten als der Durchschnitt benötigt.
Da eine höhere Anzahl der Rechenknoten eine höhere Rechenzeit impliziert, ist es von Vorteil diese reduzieren zu können.
Unter dem Aspekt lässt sich in der Analyse der Hyperparameteroptimierung von Parity ein Trend schätzen: wird keine Rekombination verwendet, könnten Rechenzeit und Speicherressourcen gespart werden, indem weniger Rechenknoten benötigt werden. 
Falls Rekombination verwendet werden soll, könnte es sich lohnen die aufwendigeren Rekombinationsalgorithmen einzusetzen, da es den Anschein hat, dass diese weniger Rechenknoten benötigen als die einfacheren Rekombinationsverfahren.
Diese Aussagen können allerdings nicht allein auf der Auswertung eines einzelnen Testszenarios basieren, liefern aber Denkanstöße für weitere Betrachtungen.\\
Des Weiteren fällt auf, dass die Streuung der Anzahl der Rechenknoten immer größer zu werden scheint, umso komplexer das Rekombinationsverfahren gewählt wird. 
Während bei One-Point Rekombination alle Werte relativ nah am Mittelwert liegen, entsteht bei der Uniform Rekombination ein großer Unterschied zwischen extrem großen und sehr kleinen Werten. 
So könnte es sich im Hinblick auf die verwendeten Ressourcen lohnen Rekombination einzusetzen, wenn durch eine gute Hyperparameteranalyse festgestellt werden kann, mit welcher CPG-Konfiguration das Einsparpotential am besten ausgeschöpft werden könnte.
Zu beachten ist hierbei allerdings auch, dass komplexere Rekombinationsalgorithmen (und Rekombination im allgemeinen) grundsätzlich mehr Rechenzeit benötigen, als wenn dieser Rekombinationsschritt ausgelassen wird.
Diese Metriken müssten sinnvoll miteinander abgeglichen werden.
\paragraph{Populationsgröße:}
Wird die Populationsgröße ($\mu$ + $\lambda$) näher betrachtet, fällt auf, dass das Verfahren ohne Rekombination eine deutlich kleinere Population benötigt als die meisten anderen Verfahren.
Auch dieser Faktor spart Ressourcen ein und kann für die Entscheidung für oder gegen ein Verfahren eine Rolle spielen.
Die CGP-Konfiguration \glqq Two-Point Clegg mit Offset\grqq\space hat ebenfalls eine sehr niedrige Populationsgröße erreicht, allerdings handelt es sich bei den Werten dieses Verfahrens nicht um die beste Parametrierung der Konfiguration.
Hier wäre der Offset wie vorher beschrieben auf 0 gesetzt worden, weshalb nur der zweitbeste Parametersatz gewählt wurde.
Der beste Parametersatz hat die Werte $\lambda=44$ und $\mu=12$, was wiederum eine größere Populationsgröße ergibt.
\paragraph{Rekombinationsraten:}
Bei der Begutachtung der verschiedenen Rekombinationsraten (Start-Rekombinationsrate und Delta Rekombinationsrate) können keine genauen Zusammenhänge zwischen den CGP-Konfigurationen und der Höhe der Rekombinationsraten erkannt werden.
Für die Start-Rekombinationsrate kann allerdings bei allen Verfahren mit konstanter Rekombinationsrate beobachtet werden, dass extrem hohe oder sehr niedrige Raten vermieden wurden.
Bei der One-Fifth Regel wurde der gesamte freigegebene Wertebereich für die initiale Rekombinationsrate ausgenutzt.
In weiteren Tests könnte überprüft werden, ob die vorgegebenen Wertebereiche für die One-Fifth Regel sinnvoll gewählt wurden oder ob die Hyperparameteranalyse für diese Fälle Werte außerhalb des vorgegebenen Intervalls bevorzugen würde.
Auch bei der linear fallenden Rekombinationsrate wurde nahezu der gesamte Definitionsbereich der Hyperparameteranalyse ausgenutzt.
\paragraph{Offset:}
Zuletzt wird die Offset-Spalte betrachtet.
Der Mittelwert der gewählten Offsets liegt bei ca. 27 Iterationen.
Am geringsten fällt der Mittelwert für die Two-Point Rekombination aus mit einem Durchschnittswert von ca. 17 Iterationen.
Werden nicht die nachbearbeiteten Zeilen der Hyperparameteranalyse betrachtet, sondern die originalen Ergebnisse, ist der Mittelwert sogar noch kleiner, mit 10 Iterationen ohne Rekombination.
Im Vergleich dazu liegt die Uniform Rekombination mit durchschnittlich ca. 37 Iterationen ohne Rekombination weit darüber.
Auf Basis dessen stellt sich die Frage, ob die Uniform Rekombination für Standard-CGP weniger effizient ist als beispielsweise die Two-Point Rekombination, da mehr Rekombinationsschritte ausgesetzt wurden.
Die zwei neu eingepflegten Offset-Werte (rot) entsprechen ungefähr dem allgemeinen Mittelwert.

\subsubsection{Analyse der Rohdatentabelle}
Die folgend bewertete Tabelle \ref{table:parityRohdaten} gibt einen Einblick in die Rohdaten der CGP-Trainings.
Alle CGP-Konfigurationen wurden dabei 50 mal am Parity-Datensatz getestet.
Die Tabelle zeigt einige wichtige Merkmale des CGP-Trainings auf.\\
Bei der Betrachtung der Tabelle wird deutlich, dass die Mutationsschritte deutlich häufiger zu einer Verbesserung der Fitness geführt haben als die Rekombinationsschritte.
Für die CGP-Konfigurationen mit Offset kann beobachtet werden, dass keine einzige Rekombination die Fitness verbessern konnte. 
Dies deutet darauf hin, dass Rekombination in den ersten Iterationen des Trainings effizienter ist, da sie eigenständig zu einer besseren Fitness führt und nicht nur die Kombination aus Rekombination und anschließender Mutation die Fitness verbessert.
Um diesen Zusammenhang zu validieren, wird der Median der Iterationen betrachtet, bei denen die positiven Rekombinationen auftreten.
Wird dieser Wert mit dem Median der Iterationen bis zur Konvergenz verglichen, kann festgestellt werden, dass die Rekombination besonders in frühen Trainingsphasen einen höheren Mehrwert bringt.
Für CGP-Konfigurationen mit linear fallender Rekombinationsrate wurden relativ hohe Werte an positiver Rekombination erreicht.
Dies könnte damit zusammenhängen, dass bei der Initialisierung eine hohe Rekombinationsrate (0,9) gewählt und somit die Rekombination in den anfänglichen Iterationen wahrscheinlicher ausgeführt wurde.
Für die konstante Rekombinationsrate ist dies nicht der Fall, da die Rekombinationsrate laut Tabelle \ref{table:parityHPO} recht niedrig gewählt wurde (0,3).
Werden die Mediane der Iterationen bis zur Konvergenz betrachtet, sollte dementsprechend festgestellt werden können, dass die CGP-Modelle mit linear fallender Rekombinationsrate deutlich geringere Werte aufweisen als die restlichen Modelle.
Dies ist allerdings nicht Fall.
Alle CGP-Modelle mit Rekombination brauchen ähnlich viele Iterationen bis zur Konvergenz unabhängig von der Rekombinationsraten-Art.
Dies kann an mehreren Gründen liegen.
Einer davon kann sein, dass der Median der Iterationen bis zur Konvergenz nur diejenigen Trainingsdurchläufe einbezieht, die das Stopp-Kriterium erfüllen. 
Es kann also sein, dass linear fallende Rekombinationsraten dazu führen, dass mehr Testdurchläufe konvergieren, ohne dass dies in der Tabelle deutlich wird.
Ein anderer Grund könnte sein, dass nicht alle positiven Rekombinationen aus Tabelle \ref{table:parityRohdaten} auch in das Training des CGP-Modells einfließen. 
Die Spalte \glqq Anzahl negative Mutationen\grqq\space gibt an, in wie vielen Fällen eine positive Rekombination durch Mutation zerstört wurde.
Das heißt in solchen Trainingsschritten wurde durch Rekombination eine Verbesserung der Fitness bewirkt.
Anschließend wurde in der gleichen Iteration ein Mutationsschritt ausgeführt, der die Fitness wieder verschlechtert.
Da die Selektion der neuen Elitisten nur nach dem Mutationsschritt stattfindet, können Fortschritte aus dem Rekombinationsschritt verloren gehen.
Zu beobachten ist, dass die Anzahlen der verlorenen positiven Rekombinationen höhere Werte aufweisen, wenn bereits mehr Rekombinationsschritte zu einer Verbesserung der Fitness beigetragen haben.
Dies könnte ein Grund dafür sein, dass sich die Anzahl der positiven Rekombinationen nicht auf die Iterationen bis zur Konvergenz auswirkt.\\
Bei Verwendung der One-Fifth Regel haben die CGP-Modelle einen geringeren Wert an positiven Rekombinationen erreicht, was ein Zeichen dafür sein kann, dass diese Rekombinationsrate die Effizienz der Rekombination nicht sinnvoll steigert.
Außerdem kann festgestellt werden, dass komplexere Rekombinationsalgorithmen mehr positive Rekombinationen aufweisen leichtere.
Dies kann darauf hindeuten, dass komplexere Rekombinationsarten, wie die Uniform Rekombination, die Fitness mit einer höheren Wahrscheinlichkeit verbessern.
Allerdings wurden für Uniform Rekombinationen ohne Offset laut Tabelle \ref{table:parityHPO} relativ hohe Rekombinationsraten gewählt, die auch einen Grund für die hohe Anzahl an positiven Rekombinationen darstellen könnten.\\
Wird der \glqq Median Iterationen bis Konvergenz\grqq\space von CGPs ohne Rekombination mit den Durchschnittswerten pro Rekombinationsart (One-Point, Two-Point und Uniform) verglichen, dann fällt auf, dass die CGPs ohne Rekombination mehr Iterationen brauchen bis sie konvergieren.
Das deutet darauf hin, dass es sich lohnen könnte Rekombination anzuwenden, da diese im Durchschnitt weniger Iterationen bis zur Konvergenz benötigen.
Allerdings werden hier wieder nur die Durchläufe gewertet, die das Stopp-Kriterium erreicht haben.
Diese Aussage muss also mit der Bayes'schen Analyse erneut bewertet werden.
\paragraph{Analyse des Offsets}
Wenn die Auswirkungen des Offsets mit den Ergebnissen der Hyperparameteroptimierung verglichen werden, wirken die Ergebnisse vorerst widersprüchlich.
Da die Offsets den Mehrwert der Rekombination blockieren besteht die Frage, wieso 7 aus 9 CGP-Kon\-fi\-gu\-ra\-tionen trotzdem einen Offset verwenden sollen.
Dies kann einerseits daran liegen, dass die Hyperparameteranalyse mehr Parametersätze hätte testen müssen, um auf effizientere Ergebnisse zu stoßen.
Ein anderer Grund könnte sein, dass der Offset im Zusammenhang mit den negativen Mutationen steht.
Es könnte sein, dass die positiven Ergebnisse der Rekombination mögliche positive Ergebnisse der Mutation beeinflussen und anschließend (bei einer Kombination aus beiden genetischen Operatoren) eine schlechtere Fitness erzielt wird.
In diesen Fällen würde die Rekombination Fortschritte bringen, wenn anschließend keine Mutation ausgeführt werden würde.
Wenn aber eine Mutation ausgeführt wird, kann nicht nur der Fortschritt der Rekombination aufgehoben werden, sondern auch der potentielle Fortschritt der Mutation entfallen.
Die Uniform Rekombination weist hohe Werte auf, wenn es um die Anzahl der negativen Mutationen geht. 
Werden die Ergebnisse auf der Hyperparameteranalyse in Tabelle \ref{table:parityHPO} hinzugezogen, kann beobachtet werden, dass Uniform die höchsten Werte an Offsets aufweist.
Außerdem musste keiner der Uniform-Offset-Werte korrigiert werden, da keine CGP-Konfiguration mit Uniform Rekombination den Offset anfänglich ausgesetzt hat.
Der Zusammenhang zwischen negativen Mutationen und Offset lässt sich allerdings nicht für alle Werte beobachten und muss demnach weiter erforscht werden.

\subsection{Rohdatenanalyse: Keijzer}
\label{subsec:rohdatenKeijzer}
\subsubsection{Analyse der Hyperparameteroptimierung}
Die Umstände der Hyperparameteroptimierung für den Keijzer-Testfall sind identisch zu denen des Parity-Testszenarios und können in Abschnitt \ref{subsec:rohdatenParity} nachgelesen werden.
Wie bei den Ergebnissen bei Parity gibt es auch bei der Offset-Optimierung im Keijzer-Test\-sze\-na\-rio 2 aus 9 Fälle, bei denen der Offset ursprünglich auf 0 gesetzt wurde.
Diese Parametersätze wurden durch den jeweils zweitbesten Parametersatz ersetzt und sind in der Tabelle rot markiert.
Die Beobachtungen in diesem Abschnitt beziehen sich nur auf den Keijzer-Benchmarktest.
In Abschnitt \ref{subsec:rohdatenZusammenfassung} werden abschließende Erkenntnisse aus den Hyperparameteranalysen von Parity und Keijzer zusammengetragen.
\paragraph{Anzahl der Rechenknoten:}
Die Tabelle \ref{table:keijzerHPO} zeigt die Ergebnisse der Hyperparameteranalyse des Keijzer-Test\-sze\-narios.
Wie bereits in Abschnitt \ref{subsec:rohdatenParity} erklärt, kann eine Reduktion der Rechenknoten in einem CGP-Modell dazu führen, dass Systemressourcen und Rechenzeit eingespart werden können.
Dies gilt natürlich nur unter Anbetracht, dass die Lösungsentwicklung des CGPs nicht darunter leidet.
Wird der allgemeine Mittelwert aller verwendeten Rechenknoten gebildet, liegt der Wert bei ungefähr 1305. 
Zu beobachten ist, dass die Anzahl der Rechenknoten beim CGP-Verfahren ohne Rekombination deutlich höher als dieser Durchschnittswert ausfällt.
Alle Mittelwerte der Anzahl der Rechenknoten für die einzelnen Rekombinationsalgorithmen liegen unterhalb des allgemeinen Mittelwerts.
Das in diesem Sinne beste Verfahren ist hier die Two-Point Rekombination.
Dieser Rekombinationsalgorithmus braucht durchschnittlich ca. 983 Rechenknoten und liegt damit weit unter dem allgemeinen Mittelwert.
Mit ca. 1117 Rechenknoten im Mittel folgt die One-Point Rekombination, während mit durchschnittlich 1300 Rechenknoten die Uniform Rekombination ungefähr dem allgemeinen Mittelwert entspricht.
Bei Betrachtung der Werte fällt außerdem auf, dass die One-Point Rekombination einen großen Unterschied zwischen höchster und niedrigster Anzahl der Rechenknoten aufweist:
Einerseits wird der allgemein niedrigste Wert für die verwendeten Rechenknoten aufgezeigt (350), andererseits wird die obere Grenze ausgereizt (2000).
Bei der Two-Point Rekombination kann dieser Effekt nicht beobachtet werden.
Außer eines Ausreißers (1700) sind alle Werte relativ ähnlich.
Allerdings kann auf Basis der Werte keine Aussage darüber getroffen werden, welche Parameter mit diesen Beobachtungen zusammenhängen.
Weder die Rekombinationsrate noch andere Parameter weisen offensichtliche Korrelationen auf.
Dieser Aspekt könnte in weiteren Untersuchungen näher betrachtet werden.
\paragraph{Populationsgröße:}
Ein Faktor, der ebenfalls Rechenzeit einsparen kann ist die Populationsgröße ($\mu$+$\lambda$).
Diese weisen im Keijzer-Testszenario keine deutlich ersichtlichen Zusammenhänge zu den CGP-Konfigurationen auf.
Die $\mu$- und $\lambda$-Werte weichen nicht stark voneinander ab, mit einigen unregelmäßigen Schwankungen, bei denen die Werte deutlich kleiner werden.
Allerdings können diese Schwankungen nicht gleichzeitig bei $\mu$ und $\lambda$ festgestellt werden.
Für Two-Point und Uniform Rekombination verringert sich der $\lambda$-Wert bei der linear fallenden Rekombinationsrate ohne Offset.
Allerdings kann durch diese Ergebnisse nicht näher ergründet werden, ob darin ein näherer Zusammenhang liegt.
\paragraph{Rekombinationsraten:}
Bei den Rekombinationsraten fällt auf, dass für die Start-Re\-kom\-bi\-na\-tions\-ra\-te der One-Fifth Regel nur sehr hohe oder sehr niedrige Raten gewählt wurden.
Dies könnte darauf hindeuten, dass noch extremere Werte gewählt worden wären, wenn die Grenzen der Start-Rekombinationsrate einen größeren Spielraum zugelassen hätten.
Für die One-Point und Uniform Rekombination wurden für die konstanten Start-Rekombinationsraten ebenfalls Werte nahe des Definitionsbereichs gewählt.
Für die Two-Point Rekombination trifft dies nicht zu.
Ebenso werden bei der linear fallenden Rekombinationsrate solche extremen Werte beobachtet.
Zusammenfassend lässt sich feststellen, dass für das Keijzer-Testszenario überwiegend die Grenzen des Definitionsbereichs als Rekombinationsraten gewählt wurden.
\paragraph{Offsets:}
Die Offsets wurden durchschnittlich bei 153 Iterationen gewählt, wobei zu beachten ist, dass zwei Parametersätze ursprünglich bessere Ergebnisse erbracht hätten, ohne die Rekombination zu Beginn auszusetzen.
Die in der Tabelle \ref{table:keijzerHPO} rot markierten Offsets wären dementsprechend ursprünglich gleich 0 gewesen.
Mit den originalen Parametersätzen wäre der Mittelwerts des Offsets bei 100 gelegen.
Werden die Werte der Tabelle \ref{table:keijzerHPO} mit den angepassten (roten) Einträgen betrachtet, fällt auf, dass der Offset-Mittelwert der CGP-Konfigurationen mit Uniform Rekombination kleiner ist als derjenige von One-Point und Two-Point Rekombination.
Werden diese verbesserten Parametersätze wieder auf ihren Originalwert (0) zurückgesetzt, ändert sich dies allerdings: Dadurch wird der durchschnittliche Offset-Wert der Uniform Rekombination höher als die jeweils anderen.
In diesem Fall ist der Durchschnitts-Offset der Two-Point Rekombination der geringste.
Es fällt außerdem auf, dass die zweitbesten Parametersätze Offset-Werte besitzen, die deutlich höher sind als der Mittelwert aller Offsets.
Der Offset der One-Point Rekombination mit linear fallender Rekombinationsrate ist dabei sogar der maximal mögliche Offset-Wert (300).
Diese extreme Schwankung der Offsets zwischen erstbestem und zweitbestem Parametersatz spricht dafür, dass die Güte eines Standard-CGP-Modells nicht kausal mit der Höhe des Offsets zusammenhängt.

\subsubsection{Analyse der Rohdatentabelle}
Folgend werden die Rohdaten der Keijzer-Testdurchläufe anhand von Tabelle \ref{table:keijzerRohdaten} bewertet.
Wie für Parity wurden dabei 50 Durchläufe für jede CGP-Konfiguration ausgeführt, um statistische Abweichungen einzufangen.
Im allgemeinen lässt sich durch die Betrachtung der Tabelle feststellen, dass die Mutation deutlich öfter eine bessere Fitness hervorbringt als die Rekombination.
Außerdem tritt keine Verbesserung der Fitness durch Rekombination ein, wenn ein Offset eingesetzt wurde.
Dieser Fakt und derjenige, dass der Median der Iterationen für positive Rekombinationen deutlich geringere Werte hervorbringt als der Median der Iterationen bis zur Konvergenz, lässt darauf schließen, dass Rekombinationsschritte zu Beginn des Trainings eine höhere Wahrscheinlichkeit aufweisen die Fitness zu verbessern als zu späteren Iterationen.\\
Ein Trend, der sich durch die Ergebnisse aufzeigt, ist, dass mehr positive Rekombinationen auftreten, umso komplexer die Rekombinationsarten gewählt werden.
So weisen CGP-Konfigurationen mit Uniform Rekombination deutlich mehr positive Rekombinationen auf als solche mit One-Point Rekombination.
Außerdem treten vermehrt negative Mutationen auf, wenn mehr positive Rekombinationen beobachtet werden.
Für die One-Point und Two-Point Rekombination führt die linear fallende Rekombinationsrate ohne Offset zu den meisten positiven Mutationen.
Dies lässt sich dadurch erklären, dass die linear fallende Rekombinationsrate einen hohen Initialisierungswert (0,9) aufweist und die anderen Rekombinationsraten-Arten in diesen Fällen einen niedrigeren Wert durch die Hyperparameteranalyse erhalten haben.
Zu beobachten ist, dass die konstante Rekombinationsrate für die Uniform Rekombination eine höhere Start-Rekombinationsrate verwendet (0,8) und deutlich mehr positive Rekombinationen erzielt als die linear fallende Rekombinationsrate.
Die Rekombination mit der One-Fifth Regel erzielt auch mit hohen Start-Rekombinationsraten relativ wenige Rekombinationserfolge.
Es könnte demnach ratsam sein eine konstant hohe Rekombinationsrate zu Beginn des Trainings einzusetzen und diese anschließend vollkommen auszusetzen.\\
Wird der Median der Iterationen bis zur Konvergenz betrachtet, weist das CGP-Modell ohne Rekombination einen deutlich höheren Wert auf, als der Durchschnitt der Mediane pro Rekombinationsart. 
So benötigt die Uniform Rekombination durchschnittlich ca. 389 Iterationen im Median bis zur Konvergenz, die One-Point Rekombination ca. 461 und die Two-Point Rekombination 533.
Dies kann darauf hindeuten, dass es sich im Durchschnitt lohnt Rekombination auszuführen, allerdings muss beachtet werden, dass hierbei nur die konvergierten Trainings betrachtet werden.
Diese Aussage muss also in weiteren Analysen validiert werden.
\paragraph{Analyse des Offsets:}
Wird der Einfluss bewertet, der der Offset auf die Mediane der Iterationen bis zur Konvergenz hat, kann zusammengefasst werden, dass die meisten CGP-Konfigurationen mit Offset weniger Iterationen bis zur Konvergenz brauchen als die jeweils gleichen CGP-Konfigurationen ohne Offset.
Es könnte sein, dass positive Rekombinationen die Mutation negativ beeinflussen, sodass eine Rekombination zwar die Fitness verbessert aber sich negativ auf die nachfolgende Mutation auswirkt.
Demnach könnte es sein, dass die alleinige Mutation eine größere Verbesserung der Fitness hervorrufen könnten als es eine positive Rekombination und anschließende Mutation schafft.
Da in den Ergebnissen nur quantitative Bewertungen stattfinden und keine qualitativen, lässt sich dieser Fall nicht ausschließen oder bestätigen.
Zu beachten ist außerdem, dass hierbei nur die Durchläufe betrachtet werden, die das Stopp-Kriterium bereits erreicht haben.\\
Neben dieser Beobachtung zum Offset-Verhalten, lässt dich erkennen, dass die Offsets in den meisten Fällen größer ausfallen, wenn der Anteil der verlorenen Rekombinationserfolge ($= \frac{\text{Anzahl negative Mutationen}}{\text{Anzahl positive Rekombinationen}}$) höher ist und gleichzeitig Rekombinationsschritte zum Erfolg geführt hätten.
Dies kann ein Indiz dafür sein, dass der Offset in der Hyperparameteranalyse nur gewählt wurde, weil Rekombination und Mutation in einigen Iterationen Gegenspieler sind, statt einen gemeinsamen Erfolg zu erzielen. In diesen Fällen könnte es besser sein die Rekombination komplett auszusetzen.


\subsection{Rohdatenanalyse: Encode}
\label{subsec:rohdatenEncode}

In diesem Abschnitt werden die Rohdaten der Encode Benchmark Tests ausgewertet.
Zu beachten ist, dass sich dieser Abschnitt nur auf die Ergebnisse von Encode bezieht und alle Schlussfolgerungen und Thesen in Abschnitt \ref{subsec:rohdatenZusammenfassung} erneut evaluiert werden.
Denkanstöße und Beobachtungen jeder Art sind in diesem Abschnitt nur auf die jeweiligen Ergebnisse von Encode bezogen und stellen keine allgemeinen Aussagen dar.\\
Die Tabellen \ref{table:encodeOnePointRohdaten}, \ref{table:encodeTwoPointRohdaten} und \ref{table:encodeUniformRohdaten} zeigen die Ergebnisse von jeweils 50 Testdurchläufen mit 10000 Iterationen für die unterschiedlichen CGP-Konfigurationen. 
Dabei ist jede Tabelle auf eine Rekombinationsart eingeschränkt.
Ziel ist es mehr über das Zusammenspiel von Rekombinationsart und Einstellungen der Rekombinationsrate zu erfahren.
Der Offset wurde hier nicht weiter betrachtet, da die Ergebnisse von Parity und Keijzer zeigen, dass der Offset dazu führt, dass die Rekombination die Fitness des CGPs nicht verbessern kann.
Außerdem schließt der Offset die Rekombination für einige Iterationen aus, was den Einblick auf die Auswirkungen der Rekombinationsraten auf den Erfolg des CGPs einschränkt.
Aus diesen Gründen wird der Offset für diesen Abschnitt ausgeschlossen.

Werden die drei Tabellen \ref{table:encodeOnePointRohdaten}, \ref{table:encodeTwoPointRohdaten} und \ref{table:encodeUniformRohdaten} betrachtet, lässt sich feststellen, dass innerhalb der 10000 Iterationen nur die wenigsten CGP-Modelle konvergiert sind.
Aus diesem Grund wird für den Encode Benchmark bei der Bayes'schen Analyse die Wahrscheinlichkeitsfunktion der Fitness nach der 10000. Iteration gefittet und analysiert.\\
Um Zusammenhänge zwischen den CGP-Konfigurationen und der Güte der CGP-Modelle zu erforschen werden Muster in verschiedenen Spalten der Tabellen miteinander abgeglichen.
Dabei werden die Spalten \glqq Anzahl positive Rekombinationen\grqq, \glqq Anzahl negativer Mutationen\grqq\space und \glqq Median Iterationen positive Rekombination\grqq\space mit den zwei Spalten \glqq Median Iterationen bis Konvergenz\grqq\space und \glqq Stopp-Kriterium erfüllt\grqq\space verglichen.
Bei diesem Musterabgleich wird kein Hinweis darauf gefunden, dass die Spalten sich gegenseitig beeinflussen.\\
Bei der One-Point Rekombination haben durchschnittlich ca. 8 Durchläufe das Stopp-Kri\-te\-rium erfüllt, bei der Two-Point Rekombination 3,7 und bei der Uniform Rekombination 8,5. 
Die CGP-Modelle mit Two-Point Rekombination erfüllen damit deutlich weniger Stopp-Kriterien als die CGP-Modelle ohne Rekombination.
Dies gilt nicht nur für den Durchschnitt, sondern auch jeweils für alle CGP-Konfigurationen mit Two-Point Rekombination.
Wird der Durchschnitt der konvergierten CGP-Modelle für jede Re\-kom\-bi\-na\-tions\-ra\-ten-Art (konstant, linear fallend und mit One-Fifth Regel) für jede Rekombinationsart verglichen, kann festgestellt werden, dass sich die Mittelwerte kaum voneinander unterscheiden.
Beispielsweise konvergieren für die One-Point Rekombination mit konstanter Rekombinationsrate durchschnittlich 8,1 Durchläufe, bei der linear fallenden Rekombinationsrate 7,7 Durchläufe und bei der Rekombinationsrate mit One-Fifth Regel 8,4 Durchläufe. 
Dies gilt für alle Rekombinationsarten gleichermaßen.
Auf den ersten Blick scheint es, als ob die Rekombinationsraten-Arten keinen großen Unterschied zueinander aufweisen, wenn es darum geht wie schnell die CGP-Modelle konvergieren.
Außerdem scheint die Anzahl der positiven Rekombinationen keinen Einfluss auf die Anzahl der erfüllten Stopp-Kriterien zu haben, da bei der One-Point Rekombination mit One-Fifth Regel vergleichsweise wenig positive Rekombinationen zu einer ähnlichen Anzahl an konvergierten CGP-Modellen führt.
Über diese beiden Punkte lässt sich allerdings vorerst keine Aussage treffen, da nur ersichtlich ist, wie viele Durchläufe bis zur 10000. Iteration konvergieren.
Es wird nicht aufgezeigt wie schnell die CGP-Modelle danach konvergieren.
Ein Vergleich der Rekombinationsraten-Arten für das Encode-Testszenario muss demnach später bei weiteren Analysen stattfinden.
Für weitere Testfälle könnten mehr Erkenntnisse gewonnen werden, wenn mehr Durchgänge das Stopp-Kriterium erfüllen.\\
Wird versucht die Anzahl der konvergierten CGP-Modelle mit der Höhe der jeweiligen Rekombinationsrate in Verbindung zu bringen, ergeben sich folgende Erkenntnisse: Für konstante Rekombinationsraten wurden die meisten Stopp-Kriterien bei besonders hohen Rekombinationsraten erfüllt. 
Außerdem werden hierfür in der ersten Hälfte des Definitionsbereichs (Rekombinationsrate gleich 0,125-0,5) sporadisch auch Erfolge in der Konvergenz festgestellt.
Für die linear fallende Rekombinationsrate werden bei besonders kleinen und großen Schrittweiten, mit denen die Rekombinationsrate verkleinert wird, eine größere Anzahl an erfüllten Stopp-Kriterien beobachtet.
Für die One-Fifth Regel können keine Muster gefunden werden.
Allerdings unterscheiden sich die Anzahlen der konvergenten Durchgänge nicht stark voneinander, was auch bedeuten kann, dass diese Schwankungen rein zufälliger Natur sind.\\
Die unterschiedlichen Mediane der Iterationen, bei denen die Re\-kom\-bi\-na\-tions\-er\-folge entstehen, unterscheiden sich nicht stark voneinander.
Diese Beobachtung kann für alle CGP-Konfigurationen getroffen werden, unabhängig davon welche Rekombinationsart oder Re\-kom\-bi\-na\-tions\-ra\-ten-Art verwendet wird.
Dies weist darauf hin, dass das Fenster an Iterationen, in denen die Rekombination die Fitness verbessern kann, nicht mit Hilfe der Rekombinationsrate beeinflusst wurde.\\
Die folgende Abbildung \ref{fig:encodePosRekombination} beschreibt die Zusammenhänge aus Rekombinationsraten-Arten und der Anzahl positiver Rekombinationen.

\begin{figure}[H]
	\centering
	\begin{subfigure}[b]{0.32\textwidth}
		\includegraphics[width=\textwidth]{Bilder/EncodePlotPositiveRekombinationKonstant.png}
		\caption{Konstant}
		\label{fig:encodePosRekombinationKonstant}
	\end{subfigure}%
	\hfill
	\begin{subfigure}[b]{0.32\textwidth}
		\includegraphics[width=\textwidth]{Bilder/EncodePlotPositiveRekombinationClegg.png}
		\caption{linear fallend}
		\label{fig:encodePosRekombinationClegg}
	\end{subfigure}%
	\hfill
	\begin{subfigure}[b]{0.32\textwidth}
		\includegraphics[width=\textwidth]{Bilder/EncodePlotPositiveRekombinationOneFifth.png}
		\caption{One-Fifth Regel}
		\label{fig:encodePosRekombinationOneFifth}
	\end{subfigure}
	\caption{Encode: Anzahl positive Rekombinationen für die verschiedenen Arten der Rekombinationsrate}
	\label{fig:encodePosRekombination}
\end{figure}

In Abbildung \ref{fig:encodePosRekombination} kann beobachtet werden, dass für jede Rekombinationsraten-Art die Uniform Rekombination die höchsten Anzahlen an positiver Rekombination produziert.
Für die konstante und linear fallende Rekombinationsrate ist die Two-Point Rekombination in dieser Hinsicht besser als die One-Point Rekombination.
Außerdem wird ersichtlich, dass die Rekombinationsrate mit der One-Fifth Regel zu niedrigeren Rekombinationserfolgen führt als die anderen Rekombinationsraten-Arten.\\
Eine höhere Rekombinationsrate bedeutet zudem eine höhere Chance, dass die Rekombination ausgeführt wird und somit die Fitness verbessert.
Die positiven Steigungen in den Abbildungen \ref{fig:encodePosRekombinationKonstant} und \ref{fig:encodePosRekombinationOneFifth} können dadurch erklärt werden.
Allerdings bleibt unklar, warum die konstanten Rekombinationsraten von 1,0 weniger Rekombinationserfolge hervorbringen als die konstanten Rekombinationsraten von 0,875.
Dies gilt für alle Rekombinationsarten mit der konstanten Rekombinationsrate.\\
Ebenfalls überraschend sind die kurzfristigen lokalen Maxima, die die Abbildungen \ref{fig:encodePosRekombinationKonstant} und \ref{fig:encodePosRekombinationOneFifth} bei geringen Rekombinationsraten aufzeigen.
Kleine Rekombinationsraten führen dazu, dass die Rekombination unwahrscheinlicher ausgeführt wird.
Dass für eine kleinere Rekombinationsrate mehr Rekombinationserfolge vermerkt werden als für eine höhere Rekombinationsrate ist demnach ungewöhnlich.
Dies könnte darauf hindeuten, dass die selteneren Rekombinationsschritte jeweils effektiver waren als diejenigen Rekombinationsschritte, die häufiger ausgeführt wurden.


\subsection{Rohdatenanalyse: Koza}
\label{subsec:rohdatenKoza}

Die Tabellen \ref{table:kozaOnePointRohdaten}, \ref{table:kozaTwoPointRohdaten} und \ref{table:kozaUniformRohdaten} zeigen die Rohdaten-Auswertungen des Koza-Testszenarios.
Für jede CGP-Konfiguration wurden 50 Modelle innerhalb von maximal 50000 Iterationen trainiert.
Jede der Tabellen stellt eine Rekombinationsart mit verschiedenen Parametrierungen der Rekombinationsraten dar.
Wie in Abschnitt \ref{subsec:rohdatenEncode} wurde für das Koza-Testszenario der Offset als Parameter nicht weiter betrachtet.
Dies geht aus den Ergebnissen von Parity und Keijzer hervor, bei denen keine erfolgreichen Rekombinationsschritte ausgeführt wurden, wenn ein Offset aktiv war.
Da mit dieser Arbeit die Effizienz der Rekombination ausgewertet werden soll, ist es sinnvoll den Offset nicht näher zu betrachten, um den vollen Umfang der Fitnessverbesserungen durch Rekombination zu bewerten.
Ziel des Abschnittes ist es, die Zusammenhänge zwischen verschiedenen Rekombinationsraten-Arten und dem Erfolg des CGP-Trainings zu finden.
Außerdem soll bestimmt werden, welche Metrik für die Auswertung der Bayes'schen Analyse verwendet werden soll.
In diesem Abschnitt werden nur die Ergebnisse von Koza ausgewertet und erlauben noch keine allgemeinen Aussagen.
Diese werden in Abschnitt \ref{subsec:rohdatenZusammenfassung} ausgewertet, indem die Ergebnisse aus allen Testszenarien zusammengetragen und verglichen werden.

Aus den Spalten \glqq Stopp-Kriterium erfüllt\grqq\space der drei Tabellen \ref{table:kozaOnePointRohdaten}, \ref{table:kozaTwoPointRohdaten} und \ref{table:kozaUniformRohdaten} geht hervor, dass relativ viele Durchgänge des CGP-Trainings konvergiert sind.
Deswegen wird für die Bayes'sche Analyse in Abschnitt \ref{subsec:bayesKoza} die Anzahl an Iterationen bis zur Konvergenz als Metrik verwendet.
Außerdem lässt sich erkennen, dass mehr Mutationen zu einer Verbesserung der Fitness geführt haben als Rekombinationsschritte.
Gleichzeitig gehen einige Erfolge der Rekombinationsschritte verloren, indem anschließend noch eine Mutation ausgeführt wurde, wie in der Spalte \glqq Anzahl negative Mutationen\grqq\space zu sehen ist.
Dies kann dadurch passieren, dass die Auswahl der Elitisten erst nach der Mutation getroffen wird und so keine Evaluation der Fitness zwischen Rekombination und Mutation stattfindet.
Eine Möglichkeit dies zu beheben wäre es, die Elitisten nach der Rekombination mit den Elitisten nach der Mutation zu vergleichen und die besten Chromosomen zu verwenden, die aus beiden genetischen Operationen resultieren.\\
Die Rekombination ist dabei vor allem in den frühen Iterationen erfolgreich.
Dabei unterscheiden sich die unterschiedlichen Rekombinationsraten-Arten.
Die Rekombination mit der One-Fifth Regel weist vor allem in den ersten Iterationen Erfolge auf, mit der linear fallenden Rekombinationsrate ist der Median etwas höher und mit der konstanten Rekombinationsrate sind teilweise relativ hohe Iterationen noch erfolgreich.
Für die konstante Rekombinationsrate werden diese hohen Iterationsmediane allerdings nur für die One-Point und Two-Point Rekombination gesichtet.
Dabei fällt auf, dass diese späten Rekombinationserfolge im Training nicht im Zusammenhang mit der Höhe der Rekombinationsrate stehen, da einige erhöhte Werte in den Spalten \glqq Median Iterationen positive Rekombination\grqq\space auch für kleinere Rekombinationsraten aufgeführt werden.
Außerdem kann kein Zusammenhang zwischen den Spalten \glqq Median Iterationen positive Rekombination\grqq\space und \glqq Median Iterationen bis Konvergenz\grqq\space gefunden werden.
Das bedeutet, dass die Rekombinationserfolge nicht durch längeres Training automatisch in späteren Iterationen stattfinden, sondern dass dieses Verhalten unabhängig von der Trainingsdauer auftritt.
Dieses Wissen könnte genutzt werden, um so Rekombination gezielter einzusetzen.
Beispielsweise könnten die frühen Rekombinationserfolge der linear fallenden Rekombinationsrate genutzt werden, wobei diese ähnlich viele positive Rekombinationen aufweist wie die konstante Rekombinationsrate.
So könnten die Erfolge der Rekombination in den ersten Iterationen genutzt und anschließend die Rekombinationsrate schneller reduziert werden.\\
Werden die Iterationen betrachtet, die die Modelle bis zur Konvergenz gebraucht haben, lässt sich feststellen, dass die CGP-Modelle mit Two-Point Rekombination durchschnittlich weniger Iterationen (278) gebraucht hat als die anderen.
Darauf folgt die Uniform Rekombination (349 Iterationen) und anschließend die One-Point Rekombination (835 Iterationen).
Demnach sind alle Mittelwerte der Iterationsmediane bis zur Konvergenz größer als der Median der Anzahl an Iterationen, die das CGP-Modell ohne Rekombination gebraucht hat (208).
Dabei muss beachtet werden, dass bei der CGP-Konfiguration ohne Rekombination alle Hyperparameter optimiert wurden.
Demnach liefern die CGP-Modelle ohne Rekombination optimierte Ergebnisse während alle anderen CGP-Konfigurationen dieses Potential noch ausschöpfen können.
Trotzdem können für einige vereinzelte CGP-Konfigurationen bereits frühere Konvergenzen bei gleicher Anzahl an erfüllten Stopp-Kriterien beobachtet werden.
Ein Beispiel bietet die Konfiguration \glqq Uniform Clegg: 0,0015\grqq\space aus Tabelle \ref{table:kozaUniformRohdaten}.
Dies ist ein Anzeichen dafür, dass CGP-Modelle durch Rekombination effizienter werden können, wenn die Rekombination sinnvoll eingesetzt wird.\\
Des weiteren werden beim Vergleich der Spalten \glqq Anzahl positive Rekombinationen\grqq, \glqq Anzahl negative Mutationen\grqq\space und \glqq Median Iterationen positive Rekombinationen\grqq\space mit den Spalten \glqq Median Iterationen bis Konvergenz\grqq\space und \glqq Stopp-Kriterium erfüllt\grqq\space keine zusammenhängenden Muster erkannt.
Eine Möglichkeit dafür, dass die Anzahl der positiven Rekombinationen keine deutliche Veränderungen in der Effizienz des CGPs hervorruft, könnten die negativen Mutationen sein.
In weiteren Tests könnte dieser Zusammenhang untersucht werden, indem keine Mutation ausgeführt wird, wenn bereits eine Rekombination die Fitness des Modells verbessert hat.
So könnte überprüft werden, ob die CGP-Modelle mit mehr Rekombinationserfolgen auch bessere Ergebnisse liefern oder ob Mutationserfolge relevanter für ein erfolgreiches CGP-Training sind.\\
Ein weiterer Punkt der auffällt sind die relativ hohen Zahlen der negativen Mutationen im Koza-Testszenario.
Dies könnte daran liegen, dass die Fitness-Werte in SR Benchmarks kontinuierlich sind, während sie in Booleschen Problemen durch diskrete Werte abgebildet werden.
Dadurch können auch kleinere Trainingserfolge durch Rekombination in den Tabellen abgebildet werden.
Dies könnte bedeuten, dass vor allem bei SR die Evaluierung nach der Rekombination als Zwischenschritt für Standard-CGP sinnvoll sein könnte.\\
Um die positiven Rekombinationen zu bewerten wird folgende Abbildung \ref{fig:kozaPosRekombination} eingeführt:

\begin{figure}[H]
	\centering
	\begin{subfigure}[b]{0.32\textwidth}
		\includegraphics[width=\textwidth]{Bilder/KozaPlotPositiveRekombinationKonstant.png}
		\caption{Konstant}
		\label{fig:kozaPosRekombinationKonstant}
	\end{subfigure}%
	\hfill
	\begin{subfigure}[b]{0.32\textwidth}
		\includegraphics[width=\textwidth]{Bilder/KozaPlotPositiveRekombinationClegg.png}
		\caption{linear fallend}
		\label{fig:kozaPosRekombinationClegg}
	\end{subfigure}%
	\hfill
	\begin{subfigure}[b]{0.32\textwidth}
		\includegraphics[width=\textwidth]{Bilder/KozaPlotPositiveRekombinationOneFifth.png}
		\caption{One-Fifth Regel}
		\label{fig:kozaPosRekombinationOneFifth}
	\end{subfigure}
	\caption{Koza: Anzahl positive Rekombinationen für die verschiedenen Arten der Rekombinationsrate}
	\label{fig:kozaPosRekombination}
\end{figure}

Abbildung \ref{fig:kozaPosRekombination} beschreibt die positiven Rekombinationsraten pro CGP-Konfiguration, aufgeteilt in die verschiedenen Rekombinationsraten-Arten. 
Dabei sind die verschiedenen Rekombinationsarten mit unterschiedlichen Farben dargestellt.
Zu erkennen ist, dass die Uniform Rekombination für jede Rekombinationsraten-Art mehr positive Rekombinationen erzeugt als es die anderen Rekombinationsarten schaffen.
Die Rekombination mit der One-Fifth Regel schneidet in dieser Betrachtung schlechter ab als die konstante und linear fallende Rekombinationsrate.
Zu beachten ist, dass es sich dabei um eine quantitative Bewertung handelt.
Es wird also nicht ersichtlich wie hoch die jeweiligen Trainingserfolge der Rekombinationen ausfallen.
Es macht Sinn dies besonders für SR Benchmarks in weiteren Tests näher zu betrachten, da hier die Fitness einen kontinuierlichen Wert einnimmt und sich die Qualität Trainingserfolge stark voneinander unterscheiden kann.\\
Für die konstante Rekombinationsrate und diejenige mit der One-Fifth Regel, werden mit höheren Rekombinationsraten auch höhere Chancen gegeben, dass eine Rekombination die Fitness verbessert.
Demnach macht es Sinn, dass für höhere Rekombinationsraten die Anzahl an positiven Rekombinationen steigt, wie es in den Abbildungen \ref{fig:kozaPosRekombinationKonstant} und \ref{fig:kozaPosRekombinationOneFifth} der Fall ist.
Interessant werden diejenigen lokalen Maxima, die bereits entstehen, bevor jeweils die maximale Rekombinationsrate erreicht wurde.
In diesen Punkten könnte die Rekombinationsrate für diese Konfigurationen der Hyperparameter besonders gut ausfallen, da hier prozentual mehr ausgeführte Rekombinationen zu einer Verbesserung der Fitness führen.


\subsection{Rohdatenanalyse: Zusammenfassung}
\label{subsec:rohdatenZusammenfassung}
\subsubsection{Zusammenfassung: Analyse der Hyperparameteroptimierung}
Zuerst sollen die Ergebnisse der Hyperparameteranalyse von Parity und Keijzer miteinander verglichen werden, um Zusammenhänge zu erschließen und einzelne Beobachtungen validieren zu können.
Die folgenden Abschnitte beziehen sich dabei auf die jeweiligen Spalten der Tabellen \ref{table:parityHPO} und \ref{table:keijzerHPO}.
\paragraph{Rechenknoten:} Bei Parity kann beobachtet werden, dass die CGP-Konfiguration ohne Rekombination am wenigsten Rechenknoten aufweist. 
Außerdem kann der Trend beobachtet werden, dass komplexere Rekombinationsalgorithmen durchschnittlich weniger Rechenknoten zugeschrieben bekommen als einfachere Algorithmen.
Dabei ist die Streuung um den Mittelwert höher, umso komplexer die Rekombinationsalgorithmen sind.
Dies könnte darauf hindeuten, dass eine größere Anzahl an Rechenknoten zu einer geringeren Streuung der Rechenknotenanzahl pro Rekombinationsart führt.
Im Vergleich dazu ergeben die Ergebnisse von Keijzer ein völlig anderes Bild: gegensätzlich zu Parity hat die CGP-Konfiguration ohne Rekombination die höchste Anzahl an Rechenknoten im Vergleich zu den Mittelwerten über die unterschiedlichen Rekombinationsarten.
Den bei Parity beobachteten Trend, dass komplexere Rekombinationen eine niedrigere Anzahl an Rechenknoten erfordern könnten, kann bei Keijzer nicht bestätigt werden.
Hier erfordert die Uniform-Rekombination die höchste durchschnittliche Anzahl an Rechenknoten.
Ebenso ergibt sich für Keijzer ein anderes Bild als bei Parity, wenn die Streuung der Rechenknotenanzahl betrachtet wird: Die Two-Point Rekombination weist zwar die geringste Anzahl an Rechenknoten auf, zeichnet sich aber gleichzeitig durch die niedrigste Streuung aus.
Zusammenfassend lässt sich für die Anzahl der Rechenknoten sagen, dass Parity und Keijzer vollkommen unterschiedliche Ergebnisse liefern.
Demnach können keine Aussagen getroffen werden, wie die Anzahl der Rechenknoten mit den unterschiedlichen CGP-Konfigurationen zusammenhängen.
Die Möglichkeit besteht ebenfalls, dass überhaupt kein Zusammenhang zwischen diesen beiden Komponenten besteht.
Durch weitere Tests könnte dies näher betrachtet werden.
\paragraph{Populationsgröße:} Bei Betrachtung der Populationsgröße kann für Parity zusammengefasst werden, dass die CGP-Konfiguration ohne Rekombination die deutlich kleinste Populationsgröße benötigt hat.
Für Keijzer kann diese Beobachtung nicht gemacht werden.
Umgekehrt kann bei Parity nicht bestätigt werden, dass Two-Point und Uniform Rekombination mit linear fallender Rekombinationsrate ohne Offset einen geringeren $\lambda$-Wert aufweisen.
Zusammenfassend lässt sich also auch für die Populationsgröße kein Zusammenhang zu den unterschiedlichen Rekombinationsarten herstellen.\\
\paragraph{Rekombinationsraten:} Bei den Ergebnissen der Hyperparameteroptimierung vom Pa\-ri\-ty-Test\-sze\-na\-rio werden von konstanten Rekombinationsalgorithmen besonders hohe und besonders niedrige Rekombinationsraten vermieden.
Für die linear fallende Rate und die Rekombinationsrate mit der One-Fifth Regel wird nahezu der gesamte Definitionsbereich genutzt.
Wieder im Gegensatz dazu stehen die Ergebnisse des Keijzer-Testszenarios.
Hier werden für nahezu alle CGP-Konfigurationen sehr hohe oder niedrige Werte für die Rekombinationsrate verwendet.
Somit kann durch die Ergebnisse der Hyperparameteranalyse nicht bewertet werden, welche Bereiche der Rekombinationsrate besonders effektiv sind.\\
\paragraph{Offsets:} Bei Parity wird der geringste Offset bei der Two-Point Rekombination und der höchste Offset bei der Uniform Rekombination verwendet.
Werden bei Keijzer die originalen Ergebnisse verwendet, bei denen ein Offset von 0 zugelassen wird, kann dieses Verhalten ebenfalls beobachtet werden.
Dies könnte bedeuten, dass die Two-Point Rekombination für Standard-CGP effizienter ist als die Uniform Rekombination, da bei ersterem weniger Rekombinationsschritte ausgesetzt werden.
Die für den Offset neu eingefügten Werte, falls ein Offset gleich 0 bestimmt wird, sind bei Parity ähnlich zum allgemeinen Mittelwert des Offsets.
Ein anderes Verhalten kann für das Keijzer-Testszenario beobachtet werden, bei dem die zweitbesten Parametersätze sehr hohe Offset-Werte verwenden.
Dies ergibt also einen extrem hohen Sprung im Offset zwischen erstbestem und zweitbestem Parametersatz.
Das ist ein Indiz dafür, dass die Güte eines CGPs nicht kausal mit der Höhe des Offsets zusammenhängt.\\
Zu beachten ist, dass die Ergebnisse der Hyperparameteranalyse kritisch begutachtet werden müssen.
Um die Ergebnisse und die daraus abgeleiteten Aussagen zu validieren ist es sinnvoll eine umfangreichere Hyperparameteranalyse auszuführen.
Diese sollte mit Hilfe von mehr Rechenkapazität ausgeführt werden, damit mehr Hyperparameter getestet werden können.
Außerdem kann die Anzahl an ausgeführten CGP-Trainings pro Bewertungsschritt in der Hyperparameteranalyse höher gesetzt werden.
Durch diese Schritte könnten sich die Ergebnisse von Parity und Keijzer aneinander angleichen, wodurch bessere Bewertungen stattfinden könnten.
Andernfalls könnte die Aussage gestärkt werden, dass die CGP-Konfigurationen keinen Zusammenhang zur Auswahl einzelner Hyperparameter aufweisen.

\subsubsection{Zusammenfassung: Analyse der Rohdatentabellen}
Folgend sollen die Rohdatenanalysen der vier Testszenarien Parity, Keijzer, Encode und Koza zusammengefasst und miteinander verglichen werden.\\
Für alle Testszenarien wird festgestellt, dass deutlich mehr Mutationen zu einer Verbesserung der Fitness geführt haben als es für Rekombinationen der Fall war.
Dies kann damit zusammenhängen, dass die Rekombinationserfolge bei allen Testszenarien in den frühen Testphasen stattgefunden haben.
Diese Beobachtung wird bei Parity und Keijzer dadurch gestützt, dass die Rekombination kein einziges mal die Fitness verbessern konnte, wenn ein Offset eingesetzt wurde.
Für die Testszenarien Keijzer und Koza wurden mehr Rekombinationserfolge aufgezeichnet als für die Booleschen Probleme.
Das kann damit zusammenhängen, dass die Fitness in Booleschen Problemen einen diskreten Wertebereich aufweist.
Für Tests aus der symbolischen Regression ergeben sich damit mehr Möglichkeiten die Fitness zu verbessern, indem die Verbesserungen kleinschrittiger ablaufen können.
Zu beachten ist, dass die Bewertungen in diesem Abschnitt \ref{sec:ergebnisseRohdaten} quantitativ sind und nicht qualitativ.
Demnach wird nur betrachtet wie viele Rekombinationen und Mutationen die Fitness verbessert haben.
Dabei lässt sich keine Aussage darüber treffen, wie groß die Rekombinations- und Mutationserfolge ausfallen.
Dies sollte in weiteren Tests näher untersucht werden, um näher beurteilen zu können, wie gut die Rekombination abschneidet.\\
Im Koza-Testszenario treten vereinzelt CGP-Konfigurationen mit konstanter Rekombinationsrate auf, in denen die erfolgreichen Rekombinationen etwas später stattfinden.
Diese leicht erhöhte Iterationenzahl hat allerdings keinen Zusammenhang zu der Anzahl an Iterationen bis zur Konvergenz.
Demnach ist es unwahrscheinlich, dass die Rekombinationserfolge nur später auftreten, weil das Training an sich länger lief.
Ein Zusammenhang zwischen den späteren Rekombinationserfolgen und der konstanten Rekombinationsrate kann nicht vollends bestätigt werden und kann in weiteren Tests näher untersucht werden.
Falls dieser oder ein ähnlicher Zusammenhang festgestellt werden kann, könnte die Rekombination gezielter verwendet werden.
So könnte beispielsweise die linear fallende Rekombinationsrate verwendet werden, die ähnliche Anzahlen einer positiven Rekombination aufweisen, allerdings frühere Rekombinationserfolge bietet.
So könnte die Rekombination früher ausgesetzt werden, ohne Erfolge der Rekombination einzubüßen.
Allerdings tritt beim Encode-Testszenario so ein Fall mit späteren Rekombinationserfolgen nicht auf.
Dies kann aber daran liegen, dass im Encode-Testszenario insgesamt weniger Rekombinationserfolge vermerkt wurden als beim Koza-Testfall.
Es ist also empfohlen dieses Verhalten bei weiteren Testfällen mit SR zu prüfen.\\
Ein weiterer Punkt, der für alle Testszenarien beobachtet wird ist, dass es Iterationen gibt, in denen die verbesserte Fitness durch die Mutation verloren gegangen ist.
Dies kann auftreten, da die Auswahl der Elitisten erst nach der Rekombination ausgeführt wird und somit im Normalfall keine Evaluation nach der Rekombination und vor der Mutation stattfindet.
Dieser Missstand könnte dadurch gelöst werden, indem die Fitness bereits nach der Rekombination evaluiert wird und hier bereits Elitisten zwischengespeichert werden.
Anschließend könnten die Elitisten nach der Mutation mit ersteren verglichen werden, um so die besten Elitisten zu wählen.\\
Des weiteren fällt auf, dass die CGP-Konfigurationen mit Rekombination nur gelegentlich eine schnellere Konvergenz erzielen, obwohl neben der Mutation auch die Rekombination die Fitness verbessert.
In den Testfällen Encode und Koza konnte außerdem festgestellt werden, dass die Spalten, die die Rekombination beschreiben keine zusammenhängenden Muster zu den Ergebnis-Spalten aufweisen.
In den Hyperparameteranalysen von Parity und Keijzer werden außerdem jeweils 7 aus 9 CGP-Konfigurationen mit einem Offset versehen und so der gesamte Rekombinationserfolg übergangen.
Für den Keijzer-Testfall konnte auch beobachtet werden, dass diejenigen CGP-Konfigurationen mit Offset weniger Iterationen bei konvergierenden Durchgängen brauchten als die jeweiligen gleichen CGP-Konfigurationen ohne Offset.
Dies könnte ein Indiz dafür sein, dass nicht nur die Mutation die Rekombination negativ beeinflusst, sondern auch umgekehrt die Rekombination den Erfolg der Mutation verringert.
Eine Lösungsmöglichkeit wäre womöglich eine Spaltung von Rekombination und Mutation, wie sie bereits für GP verwendet wird \cite{eiben_introduction_2015}.
Dabei würden die beiden genetischen Operatoren nicht nacheinander ablaufen, sondern gleichzeitig.
Dafür wird die Population gespalten und ein Teil des Nachwuchs wird durch Rekombination und der andere Teil durch Mutation erzeugt.
So würden gleichzeitig die großen Veränderungen der Chromosomen eingebracht werden, die die Rekombination erzeugt, aber auch die kleineren Anpassungen, die aus der Mutation herrühren.
So könnte der Lösungsraum gleichzeitig groß- und kleinschrittig abgesucht werden.\\
Als letzte wichtige Erkenntnis aus diesem Abschnitt kann zusammengefasst werden, dass die Uniform Rekombination deutlich mehr positive Rekombinationen hervorgebracht hat als die anderen Rekombinationsarten.
Während die konstante und linear fallende Rekombination ähnlich viele Rekombinationserfolge erzielt haben, hat die Rekombination mit der One-Fifth Regel in deutlich weniger Iterationen die Fitness verbessert.
Demnach hat die Uniform Rekombination mit konstanter oder linear fallender Rekombinationsrate das größte Potential die Rekombination effizienter zu gestalten.
Interessant ist die Beobachtung, dass nicht nur höhere Rekombinationsraten zu mehr positiven Rekombinationen geführt haben, obwohl eine höhere Rekombinationsrate die Wahrscheinlichkeit für eine effiziente Rekombination steigert.
Im Keijzer-Testszenario konnte die konstante Rekombinationsrate von 0,8 mehr positiver Rekombinationen erzielen als die linear fallende Rekombinationsrate, die mit 0,9 initialisiert wird.
Auch bei den Testfällen Encode und Koza konnten in den Plots zu positiven Rekombinationen über Rekombinationsraten lokale Maxima gefunden werden.
Dabei handelt es sich Hochpunkte für positive Rekombinationen, die für relativ geringe Rekombinationsraten entstanden sind.
Diese Rekombinationsraten könnten weiter in anderen Testszenarien untersucht werden, um Muster zu erkennen und gegebenenfalls eine ideale Rekombinationsrate zu finden oder mehr Verständnis aufzubauen.\\
Die Ergebnisse in diesem Abschnitt zeigen, dass im Gebrauch der Rekombinationsalgorithmen mehr Potential steckt als mit dem Standard-CGP-Vorgehen aus Abbildung \ref{fig:aufbauCGP} genutzt wird.
Der in diesem Abschnitt erläuterte Ausblick kann verwendet werden, um weitere Erkenntnisse zu sammeln und die Rekombination im allgemeinen weiter zu fördern.
Trotzdem, dass das Potential nicht voll genutzt wird, konnten die bisherigen Ergebnisse Grund zu der Vermutung geben, dass auch für das Standard-CGP-Vorgehen die Effizienz eines CGPs gesteigert werden kann, wenn die Rekombination richtig gewählt und parametriert wird.
Diese Ergebnisse wurden für Encode und Koza auch erzielt, obwohl nur die Hyperparameter der CGP-Konfigurationen ohne Rekombination optimiert wurden.
Der folgende Abschnitt \ref{sec:ergebnisseBayes} wertet den Ist-Stand der Rekombination in Standard-CGP weiter aus.


\section{Ergebnisse Bayes'sche Analyse}
\label{sec:ergebnisseBayes}

In diesem Abschnitt werden die Ergebnisse der Bayes'schen Analyse aufgezeigt und bewertet.
Dafür werden die Tabellen aus dem Anhang \ref{sec:appendixTabellenBayes} verwendet.
Aus den Ergebnissen der Rohdatenanalyse von Encode (Abschnitt \ref{subsec:rohdatenEncode}) und Koza (Abschnitt \ref{subsec:rohdatenKoza}) geht hervor, dass für Encode die Fitness bei der 10000. Iteration bewertet wird und für Koza die Anzahl der Iterationen bis zur Konvergenz.
Für die restlichen Testszenarien Parity und Keijzer wird ebenfalls die Anzahl der Iterationen bis zur Konvergenz analysiert.
Ziel der Bayes'schen Analyse ist es herauszufinden, ob Rekombination die Effizienz von CGPs verbessern.
Aus Abschnitt \ref{subsec:rohdatenZusammenfassung} geht hervor, dass Rekombination ein vermutlich größeres Potential besitzt als mit Standard-Vorgehensweisen genutzt wird.
Die Bayes'sche Analyse wird verwendet, um die Effizienz der Standard-Vorgehensweisen mit und ohne Rekombination abzugleichen.\\
Folgend werden die Spalten der Tabellen näher erläutert:
\begin{itemize}
	\item \textbf{CGP-Konfiguration:} Es handelt sich um die jeweils für das Training verwendete CGP-Konfiguration.
	Für Parity und Keijzer sind das die durch die Hyperparameteranalyse optimierten CGP-Konfigurationen mit und ohne Offset.
	Bei Encode und Koza wird wie in den Abschnitten \ref{subsec:rohdatenEncode} und \ref{subsec:rohdatenKoza} beschrieben, kein Offset betrachtet.
	Dafür werden jeweils verschiedene Werte der Rekombinationsraten analysiert.
	\item \textbf{HPDI (Iter. / Fitn.):} Diese Spalte gibt den HPDI von 95\% an.
	\glqq Iter.\grqq\space und \glqq Fitn.\grqq\space gibt dabei an, ob dabei die Fitness oder die Anzahl der Iterationen bewertet werden.
	Die HPDI-Berechnung kann ebenfalls mit zensierten Werten umgehen.
	Dies wird für die Analyse der Iterationen bis zur Konvergenz relevant.
	Dabei wird dem Modell mitgegeben wie viele CGP-Modelle nicht konvergiert sind und bei ungefähr welcher Iteration alle Durchläufe das Stopp-Kriterium erfüllen müssten.
	Diese Bewertung basiert demnach unter anderem auf Schätzwerten und sind mit Vorsicht zu bewerten.
	Die Berechnung der Schätzwerte basiert auf der Information wie hoch die Fitness zur Abbruchiteration ist.
	Dadurch kann die hohe der Fitness zur Abbruchiteration ebenfalls in die Bewertung einfließen.
	Da für alle bewerteten CGP-Konfigurationen das gleiche Verfahren angewendet wurde, sollten die Ergebnisse miteinander vergleichbar sein.
	\item \textbf{MW:} Diese Spalte gibt den Mittelwert der gefitteten Wahrscheinlichkeitsverteilung an.
	\item \textbf{PL-Platz:} Dabei handelt es sich um die Platzierung der CGP-Konfigurationen mit Hilfe des Plackett-Luce-Modells.
	Die Platzierung gibt eine Wahrscheinlichkeit an, dass die jeweilige CGP-Konfiguration die besten Ergebnisse liefert.
	Demnach werden Werte von 0 bis 1 vergeben, die zusammen 100\% ergeben.
	In dem Plackett-Luce-Modell können zensierte Daten nicht betrachtet werden.
	Deswegen wurden dem Modell die jeweils vorliegenden Ergebnissen zur Abbruchiteration mitgegeben.
\end{itemize}

\subsection{Bayes'sche Analyse: Parity}
\label{subsec:bayesParity}

Dieser Abschnitt bezieht sich auf die Tabelle \ref{table:parityBayesian} und bewertet lediglich die Ergebnisse der Bayes'schen Analyse von Parity.
Jegliche Aussagen können vorerst nicht auf die Allgemeinheit geschlossen werden und werden in Abschnitt \ref{subsec:bayesZusammenfassung} näher bewertet.
\paragraph{Bewertung MW und PL-Plätze}
Werden die MW der verschiedenen CGP-Modelle betrachtet, wird deutlich, dass nur 2 der 18 CGP-Konfigurationen mit Rekombination schlechter abschneiden als diejenige ohne Rekombination.
Das ist ein Zeichen dafür, dass Rekombination durchaus die Effizienz von CGP steigern kann, wenn die Rekombination sinnvoll konfiguriert wird.\\
Die jeweils drei niedrigsten MW und drei höchsten PL-Plätze sind in der Tabelle \ref{table:parityBayesian} grün markiert, während die drei höchsten MW und drei niedrigsten PL-Plätze rot sind.
Zu sehen ist, dass die MW und PL-Plätze nicht genau die gleichen CGP-Konfigurationen als gleich gut oder schlecht bewerten.
Dies kann daher kommen, dass in die Berechnung der MW Schätzwerte einfließen, wann ein CGP-Modell konvergiert.
Für die Berechnung der PL-Plätze ist das nicht der Fall.
Die MW geben demnach mehr Eindrücke darauf wie nah die CGP-Modelle an der Konvergenz sind, wenn sie das Stopp-Kriterium noch nicht erfüllt haben.
Allerdings handelt es sich dabei wie bereits erwähnt um einen Schätzwert, der durchaus mit Vorsicht betrachtet werden muss.
Die Tabelle \ref{table:parityRohdaten} zeigt jedoch, dass Trends ob eine CGP-Konfiguration gute oder schlechte Ergebnisse hervorbringt, durch MW und PL-Plätze gleich bewertet werden.
Bei den Unterschieden handelt es sich nur um einzelne Platzierungen der CGP-Konfigurationen.
Durch die gemeinsame Betrachtung von MW und PL-Plätze kann demnach ein gutes Bild davon gegeben werden, wie effizient einzelne CGP-Modelle trainieren.
Werden die besten und schlechtesten MW und PL-Plätze in der Tabelle betrachtet, wird deutlich, dass die CGP-Konfigurationen mit der One-Point Rekombination deutlich bessere Ergebnisse liefert als die anderen Rekombinationsarten.
Aus Abschnitt \ref{subsec:rohdatenParity} ist bekannt, dass für die One-Point Rekombination deutlich weniger Rekombinationserfolge auftreten als bei den anderen Rekombinationsarten.
Deswegen kann davon ausgegangen werden, dass bei diesen CGP-Konfigurationen die Rekombination weniger Einfluss auf das Training hat.
Werden die schlechtesten Verfahren nach MW und PL-Plätzen mit Rekombination ohne Offset aus Tabelle \ref{table:parityBayesian} mit der Tabelle \ref{table:parityRohdaten} verglichen kann beobachtet werden, dass der Anteil der negativen Mutationen pro positive Rekombinationen relativ hoch ist.
Das kann dafür sprechen, dass Rekombination mit dem Standard-CGP-Verfahren mit weniger einflussreicher Rekombination besser funktioniert, da Rekombination und Mutation nicht ausreichend aufeinander abgestimmt sind und damit einflussreichere Rekombination (mit mehr möglichen positiven Rekombinationsschritten) weniger gute Ergebnisse liefern.
Wenn sich dieser Ansatz bewahrheitet, könnte demnach durch eine Anpassung des CGP-Verfahrens Rekombination sinnvoller eingesetzt werden, sodass einflussreichere Rekombinationsarten wie die Uniform Rekombination einen größeren Mehrwert generieren.\\
Des Weiteren werden die Ergebnisse jeweils bezüglich MW und PL-Plätzen von Verfahren mit Offset mit den Ergebnissen der jeweils gleichen CGP-Konfigurationen ohne Offset verglichen.
Dabei weisen jeweils sechs aus neun CGP-Konfigurationen ohne Offset bessere Ergebnisse auf als diejenigen mit Offset.
Dies spricht dafür, dass ein Rekombinationsoffset der Effizienz eines CGP-Modells schadet.
Ein Grund dafür könnte sein, dass damit die positiven Einflüsse der Rekombination gestrichen werden, während die nachteiligen Effekte des Zusammenspiels von Rekombination und Mutation weiterhin bestehen bleiben.
\paragraph{Bewertung HPDI}
Analysiert werden nun die HPDI-Werte aus Tabelle \ref{table:parityBayesian}.
Dabei werden die HPDI-Werte mit dem kleinsten Umfang grün markiert und die mit dem größten rot.
Diese werden verwendet, um die Ergebnisse stichprobenartig zu bewerten.
Es fällt auf, dass die One-Point Rekombination die besten drei Werte beinhaltet, wenn der HPDI-Umfang bewertet wird.
Wenn aus allen HPDI-Werten die unteren Grenzen betrachtet und verglichen werden, kann außerdem beobachtet werden, dass die drei kleinsten unteren Grenzen Teil der grün markierten HPDIs sind.
Umgekehrt lässt sich beobachten, dass besonders große untere Grenzen zu großen Umfängen der HPDI führen.
Bei den drei größten unteren Grenzen ist dies in einem Fall keine rot markierte HPDI, allerdings handelt es sich dabei um einen ähnlich großen HPDI-Umfang, wodurch ein Trend erkennbar wird.
Dieses Verhalten weist darauf hin, dass die frühe Konvergenz der frühesten erfolgreichen Trainings ein Anzeichen dafür ist, dass die meisten CGP-Modelle relativ früh konvergieren.
Zu beachten ist, dass es sich hierbei nicht um Ausreißer handelt, da diese in der 95\%-igen HPDI nicht erfasst werden.
Dies kann dafür sprechen, dass frühe Trainingserfolge besonders wichtig sind, wenn eine verlässliche CGP-Konfiguration generiert werden soll, deren Modelle in einer relativ kleinen Iterationsspanne konvergieren sollen.
Da Rekombination laut Abschnitt \ref{subsec:rohdatenZusammenfassung} besonders in früheren Iterationen ein hohes Potential aufweist die Fitness zu verbessern, ist es demnach sinnvoll die Rekombination und deren Einsatz effizienter zu gestalten.

\subsection{Bayes'sche Analyse: Keijzer}
\label{subsec:bayesKeijzer}

Dieser Abschnitt beurteilt die Bayes'sche Analyse des Keijzer-Testszenarios und bezieht sich dabei auf die Tabelle \ref{table:keijzerBayesian}.
Alle Aussagen und Bewertungen der Ergebnisse beziehen sich demnach nur auf den Keijzer-Testfall und können nicht direkt auf die Allgemeinheit bezogen werden.
In Abschnitt \ref{subsec:bayesZusammenfassung} werden die Ergebnisse aller Testszenarien zusammengefasst und bewertet.
\paragraph{Bewertung MW und PL-Plätze}
Im ersten Schritt wird der MW aller CGP-Konfigurationen bewertet.
Bei 12 aus 18 Verfahren mit Rekombination sind die MW kleiner und damit besser als die der CGP-Konfiguration ohne Rekombination.
Bei Betrachtung der Mittelwerte der Spalte MW über die verschiedenen Rekombinationsarten hinweg ergeben sich folgende Durchschnittswerte: One-Point Rekombination braucht durchschnittlich ca. 8200 Iterationen, Two-Point ca. 8670 und Uniform ungefähr 8650.
Damit liegen alle Mittelwerte unterhalb von ca. 9552 Iterationen, die das CGP-Modell ohne Rekombination benötigt.
Das heißt, dass es sich beim Standard-CGP-Vorgehen im Durchschnitt lohnt Rekombination anzuwenden.\\
Die jeweils drei besten MW und PL-Plätze sind in der Tabelle grün markiert, während die schlechtesten rot sind.
Zu beobachten ist, dass die jeweils besten und schlechtesten Werte der beiden Spalten nicht immer übereinstimmen.
Dies ist der Fall, da bei den PL-Plätzen keine Schätzung darüber abgegeben wird, wann nicht-konvergente CGP-Modelle das Stopp-Kriterium erfüllen. 
Hier wird nur die maximale Iterationsgrenze (6500 Iterationen) angegeben, falls ein CGP-Modell nicht fertig trainiert wurde.
Demnach geben die Werte der Spalte MW besser wieder, wie hoch die Fitness beim Abbruch des CGP-Trainings war.
Allerdings besteht bei der Schätzung der Iterationen ein gewisser Unsicherheitsfaktor hinsichtlich der Entwicklung der Ergebnisse des CGP-Modelle.
Deswegen werden beide Spalten in die Bewertung einbezogen.
Zu sehen ist, dass die schlechtesten Ergebnisse von MW und PL-Platz in den Rekombinationsraten Two-Point und Uniform auftreten.
Allerdings treten ebenfalls die besten Ergebnisse vor allem in diesen Rekombinationsarten auf.
Dies zeigt auf, wie wichtig es ist nicht nur die Rekombinationsart richtig zu wählen, sondern auch die Rekombinationsraten-Arten und deren Parametrierungen abzuwägen.\\
Wird ein näherer Blick auf diejenigen CGP-Konfigurationen aus Tabelle \ref{table:keijzerRohdaten} mit Rekombination ohne Offset geworfen, bei denen die Spalten MW und PL-Platz aus Tabelle \ref{table:keijzerBayesian} besonders gut oder besonders schlecht ausfallen, fällt auf, dass die schlechten Ergebnisse vor allem bei denjenigen CGP-Konfigurationen auftreten, in denen viele negative Mutationen auftreten (Two-Point Clegg und Uniform Konstant).
Dies kann bedeuten, dass in diesen CGP-Modellen die Rekombination und Mutation häufiger gegeneinander gearbeitet haben, anstatt gemeinsam bessere Ergebnisse zu liefern.
Die besseren Ergebnisse nach MW und PL-Plätzen treten in denjenigen Konfigurationen auf, in denen die positiven Rekombinationserfolge entweder relativ selten auftreten oder in denen die negativen Mutationen anteilig weniger auftreten (One-Point Konstant und Uniform One-Fifth).
Das könnten diejenigen CGP-Konfigurationen sein, in denen die Rekombination weniger einflussreich ist oder sich die Rekombination und Mutation weniger negativ beeinflussen.
Da bei der CGP-Konfiguration \glqq Uniform One-Fifth ohne Offset\grqq\space in Tabelle \ref{table:keijzerRohdaten} eine sehr frühe Iterationenzahl für positive Rekombination erfasst wird, kann es sein, dass die Rekombination anschließend mit Hilfe der One-Fifth Regel so reduziert wurde, sodass die Mutation nicht mehr negativ beeinflusst werden konnte.
Dieses Szenario kann allerdings mit den vorliegenden Daten nicht untersucht und demnach nicht evaluiert werden.\\
Des Weiteren werden alle Ergebnisse aus MW und PL-Plätzen derjenigen CGP-Kon\-fi\-gu\-ra\-tionen mit Offset mit den Ergebnissen der gleichen Konfigurationen ohne Offset verglichen.
Für den MW sind die Ergebnisse von sechs aus neun Konfigurationen ohne Offset besser.
Werden die PL-Plätze bewertet sind es nur fünf aus neun.
Die Ergebnisse zeigen, dass es im Durchschnitt besser ist die Rekombination nicht auszusetzen.
Trotzdem zeigen diese Ergebnisse auch, dass das Potential von Rekombination noch weiter ausgebaut werden muss, um eine verlässliche Lösung anzubieten, bei der kein Offset mehr benötigt wird.
\paragraph{Bewertung HPDI}
Für die nähere Betrachtung der HPDI-Werte aus Tabelle \ref{table:keijzerBayesian} werden die drei HPDI-Ranges mit dem geringsten Umfang grün, während diejenigen mit dem größten Umfang rot markiert werden.
Es fällt auf, dass die Uniform Rekombination relativ kleine HPDI-Ranges aufweist, bis auf die zwei sehr großen Umfänge.
Die One-Point Rekombination weist zwei sehr geringe HPDI-Umfänge auf, während die restlichen relativ hoch ausfallen.
In der näheren Analyse wird kein Grund gefunden, wieso diese CGP-Konfigurationen dermaßen niedrige oder hohe HPDI-Ranges aufweisen. 
Werden die drei höchsten unteren Grenzen der HPDI-Werte näher betrachtet, fällt auf, dass diese zu den HPDI-Werten mit den größten Umfängen gehören.
Wenn umgekehrt die niedrigsten unteren Grenzen gesammelt werden, passen zwei von drei zu den HPDI-Werten mit den geringsten Umfängen, während der dritte einen ähnlich kleinen HPDI-Umfang aufzeigt.
Dies zeigt, dass wie bereits in \ref{subsec:bayesParity} erläutert, frühe Konvergenzen zu einem kleineren HPDI-Umfang führen.
So ist es zum Beispiel unwahrscheinlich, dass ein CGP bei mehreren Durchläufen relativ lange bis zur Konvergenz braucht aber dafür alle Durchläufe ähnlich lange benötigen.
Demnach ist es wichtig frühe Iterationen sinnvoll für das Training zu nutzen, um eine verlässliche Konvergenzrate zu erzielen.
Da Abschnitt \ref{subsec:rohdatenZusammenfassung} zufolge die Rekombination besonders in frühen Iterationen Potential aufzeigt die Fitness zu verbessern, ist es umso wichtiger die Rekombination effizienter zu gestalten.


\subsection{Bayes'sche Analyse: Encode}
\label{subsec:bayesEncode}

Dieser Abschnitt befasst sich mit der Evaluierung der Bayes'schen Analyse von Encode und bezieht sich auf die Tabellen \ref{table:encodeOnePointBayesian}, \ref{table:encodeTwoPointBayesian} und \ref{table:encodeUniformBayesian}.
In der Bayes'schen Analyse von Encode wird bewertet wie hoch die Fitness-Werte der CGP-Modelle bei der 10000. Iteration sind.
Zu beachten ist, dass sich die Aussagen innerhalb dieses Abschnittes nur auf den Encode-Testfall beziehen und vorerst keine allgemeinen Schlussfolgerungen zulassen.
In Abschnitt \ref{subsec:bayesZusammenfassung} werden alle Ergebnisse der Bayes'schen Analysen zusammengetragen und verglichen.\\
Innerhalb der Tabellen \ref{table:encodeOnePointBayesian}, \ref{table:encodeTwoPointBayesian} und \ref{table:encodeUniformBayesian} wurden die besten drei Ergebnisse pro Spalte grün markiert und die schlechtesten rot.
Zu beachten ist, dass die CGP-Konfiguration ohne Rekombination durch eine Hyperparameteranalyse optimiert wurde.
Die restlichen CGP-Konfigurationen haben die Hyperparameter übernommen und die Rekombinationsrate belegt.
\paragraph{Bewertung MW und PL-Plätze}
Werden die MW der CGP-Konfigurationen verglichen, kann beobachtet werden, dass für die One-Point und Uniform Rekombination jeweils 20 der 22 Verfahren mit Rekombination eine bessere Fitness aufweisen als die CGP-Konfiguration ohne Rekombination.
Für die Two-Point Rekombination sind alle Verfahren schlechter, wenn Rekombination ausgeführt wird.
Werden die Mittelwerte der Spalte MW über jede Rekombinationsraten-Art gebildet, ergibt sich folgende Tabelle \ref{table:encodeMW}:

\begin{table}[H]
	\centering
	\begin{tabular} {c | c | c | c}
		& \textbf{Konstant} & \textbf{Clegg} & \textbf{One-Fifth} \\
		\hline
		\textbf{One-Point Rekombination} & 0,03967 & 0,04082 & 0,03803\\
		\hline
		\textbf{Two-Point Rekombination} & 0,05690 & 0,05618 & 0,05387\\
		\hline
		\textbf{Uniform Rekombination} & 0,03858 & 0,03582 & 0,03748\\
	\end{tabular}
	\caption{Encode: Durchschnittliche Fitness nach 10000. Iteration}
	\label{table:encodeMW}
\end{table}

Es ist zu erkennen, dass für den Encode-Testfall die Two-Point Rekombination deutlich schlechtere Ergebnisse liefert als die anderen Rekombinationsarten.
Für jede Re\-kom\-bi\-na\-tions\-ra\-ten-Art bietet die Uniform Rekombination die besten Ergebnisse.
Werden die Re\-kom\-bi\-na\-tions\-ra\-ten-Ar\-ten bewertet, weist die One-Fifth Rekombinationsrate für die One-Point Rekombination das beste Ergebnis auf, gefolgt von der konstanten Rekombinationsrate und zuletzt der linear fallenden Rekombinationsrate.
Die Two-Point Rekombination weist für alle Rekombinationsraten sehr ähnliche Werte auf.
Die Reihenfolge derer ist: One-Fifth, linear fallend, konstant.
Bei der Uniform Rekombination ist die linear fallende Rekombinationsrate die beste.
Es folgt die One-Fifth Rekombinationsrate und anschließend die konstante Rekombinationsrate.
Es lässt sich also keine einheitliche Reihenfolge bilden, die die Effizienz der Rekombinationsraten-Arten abbildet.
Ebenfalls lassen sich durch die Auswertung der Tabellen \ref{table:encodeOnePointBayesian}, \ref{table:encodeTwoPointBayesian} und \ref{table:encodeUniformBayesian} keine Muster finden wie die verschiedenen Re\-kom\-bi\-na\-tions\-ra\-ten-Ar\-ten am besten zu parametrieren sind.
Dementsprechend wäre es für das Standard-CGP-Verfahren empfehlenswert verschiedene Rekombinationsraten-Arten innerhalb der Hyperparameteranalyse zu testen.
\paragraph{Bewertung HPDI}
Für die Testszenarien Parity und Keijzer wurde in den Evaluationen der Bayes'schen Analyse in den Abschnitten \ref{subsec:bayesParity} und \ref{subsec:bayesKeijzer} herausgefunden, dass kleine untere Grenzen in den HPDI-Werten sehr wahrscheinlich eine kleine Intervallgröße hervorbringen.
Bei diesen Bayes'schen Analysen wurde allerdings die Anzahl der Iteration bis zur Konvergenz betrachtet, im Gegensatz zu Encode, bei dem die Fitness zu einer bestimmten Iteration betrachtet wird.
Demnach soll diese Beobachtung für den Fall der Fitness-Bewertung auch betrachtet werden.
Allerdings fällt bei der Betrachtung der Tabelle \ref{table:encodeTwoPointBayesian} auf, dass diese Beobachtung für die Fitness-Bewertung nicht zutrifft:
Ein Gegenbeispiel ergeben die drei fett gedruckten HPDI-Werte dieser Tabelle, die alle drei die kleinsten untersten Grenzen abbilden.
Diese kleinen unteren Grenzen führen allerdings zu relativ großen HPDI-Ranges.
Dementsprechend kann keine Aussage darüber getroffen werden, wie gut die Fitnesswerte der anderen Durchläufe zu einer bestimmten Iteration sind, wenn bereits bekannt ist, dass einige Fitness-Werte sehr klein sind.


\subsection{Bayes'sche Analyse: Koza}
\label{subsec:bayesKoza}

Dieser Abschnitt evaluiert die Ergebnisse der Bayes'schen Analyse des Koza-Testszenarios und bezieht sich dabei auf die Tabellen \ref{table:kozaOnePointBayesian}, \ref{table:kozaTwoPointBayesian} und \ref{table:kozaUniformBayesian}.
Die Bayes'sche Analyse des Koza-Testfalls bewertet wie in Abschnitt \ref{subsec:rohdatenKoza} beschrieben die Anzahl der Iterationen bis zur Konvergenz.
Alle Aussagen und Ergebnisse in diesem Abschnitt beziehen sich ausschließlich auf die Auswertung des Koza-Testfalls und können nicht als allgemeine Aussagen gewertet werden.
In Abschnitt \ref{subsec:bayesZusammenfassung} werden die Evaluationsergebnisse aller Testszenarien für die Bayes'sche Analyse zusammengefasst und bewertet.
\paragraph{Bewertung MW und PL-Plätze}
Wird die Spalte MW aus den jeweils genannten Tabellen betrachtet und einzeln miteinander verglichen, werden folgende Ergebnisse sichtbar: für die One-Point Rekombination sind keine Verfahren mit Rekombination besser als das ohne Rekombination, für die Two-Point Rekombination sind fünf aus 22 CGP-Modelle mit Rekombination besser, für die Uniform Rekombination sind es lediglich drei Verfahren.
Dies zeigt, dass für den Koza-Testfall Verfahren mit eingefügtem Rekombinationsschritt deutlich schlechter abschneiden als ohne Rekombination.
Dabei ist allerdings zu beachten, dass die CGP-Konfiguration ohne Rekombination mit Hilfe einer Hyperparameteranalyse optimiert wurde, während die Verfahren mit Rekombination lediglich diese Parametrierung übernommen und die Rekombinationsrate ergänzt haben.
Demnach kann keine Aussage darüber getroffen werden wie effizient die Rekombination tatsächlich ist.
Da für Parity (Abschnitt \ref{subsec:bayesParity}) und Keijzer (Abschnitt \ref{subsec:bayesKeijzer}) deutlich bessere Ergebnisse beobachtet werden konnten, könnte es sein, dass eine Hyperparameteranalyse für alle verwendeten CGP-Verfahren die Ergebnisse verbessern könnte.
Dies führt zu der Annahme, dass bei einer Hyperparameteroptimierung alle Parameter einer CGP-Konfiguration auf einmal betrachtet werden müssen und somit keine Hyperparameter einer anderen Optimierung verwendet werden können, um lediglich einen einzelnen Parameter zu optimieren.
Diese Annahme müsste in weiteren Tests geprüft werden.
Es kann allerdings auch sein, dass der Koza-Testfall deutlich schlechtere Werte für Verfahren mit Rekombination aufweist, da in diesem Testfall deutlich mehr Iterationen bis zur Konvergenz benötigt werden als bei Parity und Keijzer.
In diesen weiteren Iterationen werden nach Abschnitt \ref{subsec:rohdatenZusammenfassung} Mutationsschritte relevanter als Rekombinationsschritte für die Effizienz des Trainings.
Demnach ist es auch möglich, dass die Rekombinationen die Mutation in den späteren Iterationen so negativ beeinflussen, dass das Verfahren ohne Rekombination besser abschneidet.\\
Die folgende Tabelle \ref{table:kozaMW} zeigt die ungefähren Mittelwerte aller Rekombinationsarten pro Rekombinationsraten-Art. 

\begin{table}[H]
	\centering
	\begin{tabular} {c | c | c | c}
		& \textbf{Konstant} & \textbf{Clegg} & \textbf{One-Fifth} \\
		\hline
		\textbf{One-Point Rekombination} & 22164 & 19272 & 14804\\
		\hline
		\textbf{Two-Point Rekombination} & 9510 & 8509 & 8444 \\
		\hline
		\textbf{Uniform Rekombination} & 12599 & 7015 & 11273 \\
	\end{tabular}
	\caption{Koza: Durchschnittliche Iterationen bei Konvergenz}
	\label{table:kozaMW}
\end{table}

Zu beobachten ist, dass die One-Point Rekombination für alle Rekombinationsraten-Arten besonders schlechte Werte hervorbringt.
Die Two-Point Rekombination ergibt im Vergleich zu den anderen Rekombinationsarten relativ gute Ergebnisse für alle Re\-kom\-bi\-na\-tions\-raten-Arten.
Für die Uniform Rekombination führt nur die linear fallende Rekombinationsrate zu guten Ergebnissen.
Diese sind deutlich besser als die Ergebnisse der Two-Point Rekombination.
Dieser Erfolg der linear fallenden Rekombinationsrate für die Uniform Rekombination könnte darauf zurückgeführt werden, dass die Start-Rekombinationsrate mit 0,9 hoch gewählt wurde.
Allerdings sind die Ergebnisse der konstanten Rekombinationsrate für hohe Rekombinationsraten aus Tabelle \ref{table:kozaUniformBayesian} besonders schlecht.
Dementsprechend ist das konsequente Verringern der Rekombinationsrate in diesem Fall der relevante Unterschied zu den anderen Rekombinationsraten-Arten.
Selbst für eine kleine Delta-Rekombinationsrate von 0,0005 für linear fallende Rekombinationen wird die Rekombination nach spätestens 1800 Iterationen vollständig ausgesetzt.
Dies könnte der Grund sein, warum die linear fallende Rekombination für einfachere Testfälle wie Parity oder Keijzer nicht deutlich effizienter als die anderen Re\-kom\-bi\-na\-tions\-raten-Arten abschneidet.
Denn diese Testfälle brauchen deutlich weniger Iterationen als der Koza-Testfall bis zur Konvergenz, wodurch die Rekombination womöglich nicht früh genug abgestellt wurde.
Werden alle Werte der linear fallenden Rekombination aus Tabelle \ref{table:kozaMW} betrachtet, fällt auf, dass die Mittelwerte für komplexere Rekombinationsarten besser ausfallen.
Dies könnte daran liegen, dass komplexere Rekombinationsarten laut Abschnitt \ref{subsec:rohdatenKoza} deutlich mehr Rekombinationserfolge vermerken können als die weniger komplexen Rekombinationsarten.
Wenn die Rekombination also rechtzeitig vollständig unterbrochen wird, könnten besonders komplexere Rekombinationsarten einen höheren Erfolg einbringen, da ihr erhöhter Einfluss auf das Trainingsverhalten in den ersten Trainingsphasen sinnvoll genutzt werden könnte ohne anschließend negative Effekte auf die Mutation zu erzeugen.\\
Die Betrachtung der PL-Plätze aus den Tabellen \ref{table:kozaOnePointBayesian}, \ref{table:kozaTwoPointBayesian} und \ref{table:kozaUniformBayesian} hat keine weiteren Erkenntnisse oder Muster hervorgebracht.
\paragraph{Bewertung HPDI}
Für die Bewertung der HPDI-Werte wurden jeweils die kleinsten unteren und oberen Grenzen grün markiert, während die größten unteren und oberen Grenzen rot sind.
Zu beobachten ist, dass in den meisten Fällen die jeweils kleinsten und Größten Grenzen aufeinander treffen.
In den Ausnahmen werden ähnlich kleine/große Werte markiert.
Für die markieren HPDI-Werte wurden die HPDI-Umfänge berechnet, wobei sich herausgestellt hat, dass die kleinen Grenzen kleine Ranges hervorbringen und die großen Grenzen große Ranges aufweisen.
Diese Bewertung ergibt das gleiche Ergebnis wie die Abschnitte \ref{subsec:bayesParity} und \ref{subsec:bayesKeijzer}, obwohl eine anderen Herangehensweise genutzt wurde, um dieses hervorzubringen:
Frühe Konvergenzen innerhalb der unteren Grenze des HPDI führen zu kleineren Streuungen der Ergebnisse.
Demnach ist es wichtig bereits in den frühen Iterationen gute Trainingsergebnisse zu erzielen, um eine große Streuung der Ergebnisse vorzubeugen.
Deswegen ist es sinnvoll die Rekombination und deren Einsatz weiter zu optimieren, die nach Abschnitt \ref{subsec:rohdatenZusammenfassung} in den ersten Iterationen das größte Potential besitzt einen Mehrwert zu liefern.
Des Weiteren können durch diese Erkenntnis aus den HPDI-Werten die Streuungen der Ergebnisse frühestmöglich eingeschätzt werden, wenn noch nicht alle CGP-Modelle trainiert wurden.
Wenn für CGP-Konfigurationen bereits einige Durchläufe deutlich früher konvergiert sind als in anderen, ist es wahrscheinlich, dass die folgenden Durchläufe zu einer relativ ähnlichen Iteration konvergieren.
Umgekehrt wird es für die folgenden Durchläufe der langsameren CGP-Modelle wahrscheinlicher, dass diese umso später konvergieren.


\subsection{Bayes'sche Analyse: Zusammenfassung}
\label{subsec:bayesZusammenfassung}

Dieser Abschnitt vergleicht die Ergebnisse und Aussagen der Bayes'schen Analyse aller Testszenarien und fasst die Erkenntnisse zusammen.
\paragraph{Erkenntnisse zum Offset}
Zuerst werden die Ergebnisse der Analysen aus den Abschnitten \ref{subsec:bayesParity} und \ref{subsec:bayesKeijzer} betrachtet, um die Verwendung eines Offsets für die Rekombination zu bewerten.
Für Parity und Keijzer wurde anhand des MW und PL-Platzes bestimmt, welche jeweils gleichen CGP-Konfigurationen mit oder ohne Offset bessere Ergebnisse geliefert haben.
Das Ergebnis zeigt, dass es durchschnittlich besser ist keinen Offset anzuwenden.
Vermutlich ist dies darauf zurückzuführen, dass so die positiven (aber auch negativen) Einflüsse der Rekombination zu Beginn des Trainings blockiert werden, während die negativen Folgen zu den späteren Iterationen wieder eintreten.
Trotzdem haben ein paar der CGP-Modelle mit Offset bessere Ergebnisse geliefert als ohne Offset.
In diesen Fällen überwiegen wahrscheinlich die negativen Einflüsse der Rekombination auf die Mutation in den ersten Iterationen, sodass ein Offset sinnvoll ist.
Ein Ziel muss daher also sein, die Rekombination oder die CGP-Trainingsverfahren so zu modifizieren, dass die Rekombination effizienter wird.
\paragraph{Erkenntnisse aus MW und PL-Plätzen}
Folgend werden die Ergebnisse bezüglich der Spalten \glqq MW\grqq\space und \glqq PL-Platz\grqq\space ausgewertet.
Bei den Testszenarien Parity und Keijzer, bei denen alle CGP-Konfigurationen durch eine Hyperparameteranalyse optimiert wurden, konnte festgestellt werden, dass die Verwendung von Rekombination durchaus Sinn macht und in relativ vielen Fällen zu schnelleren Konvergenzen führt.
Dabei war vor allem die One-Point Rekombination effektiver als die anderen Rekombinationsarten.
Die Two-Point Rekombination führte zu besonders guten und auch besonders schlechten Ergebnissen.
Dies zeigt auf, dass es sinnvoll wäre bei den Hyperparameteroptimierungen auch unterschiedliche Rekombinationsraten-Arten zu testen, da diese den Trainingserfolg stark beeinflussen können.
Bei der Untersuchung vom Keijzer-Testfall ist aufgefallen, dass schlechte Ergebnisse vor allem durch diejenigen CGP-Konfigurationen ausgelöst wurden, die viele negative Mutation hervorgerufen haben, also Mutationen die die Fitnessverbesserung der Rekombination teilweise oder vollständig ausgelöscht haben.
Dies sind wahrscheinlich diejenigen CGP-Modelle, bei denen die Rekombination und Mutation besonders stark gegeneinander arbeiten.
Besonders gute Ergebnisse wurden beobachtet, wenn entweder wenige Rekombinationsschritte zu einer Verbesserung der Fitness geführt haben (positive Rekombinationen) oder anteilig weniger negative Mutationen im Vergleich zu den positiven Rekombinationen festgestellt wurden.
Dabei könnte es sich um CGP-Modelle handeln, bei denen die Rekombination entweder weniger Einfluss auf das Training ausübt oder die positiven Einflüsse der Rekombination die negativen Effekte auf die Mutation überwiegen.
Diese Beobachtung deckt sich ebenfalls damit, dass die One-Point Rekombination als die durchschnittlich beste Rekombinationsart hervorgegangen ist.
Wie in Abschnitt \ref{sec:ergebnisseRohdaten} zu sehen ist, bringen einfache Rekombinationsarten weniger Trainingserfolge und haben damit einen geringeren Einfluss auf den Trainingserfolg.
Dadurch kollidieren die wahrscheinlich weniger oder überhaupt nicht mit den Einflüssen der Mutation auf das Training.
Für das Standard-CGP-Verfahren mit Rekombination und anschließender Mutation scheint für weniger komplexe Testszenarien die One-Point Rekombination die effektivste Rekombinationsart zu sein.
Allerdings könnten Veränderung des Trainingsverfahrens dazu führen, dass komplexere Rekombinationsarten mit mehr Einfluss auf das Trainingsgeschehen wie die Uniform Rekombination zu noch besseren Ergebnissen führen.\\
Bei den komplexeren Testszenarien Encode und Koza konnten andere Ergebnisse festgestellt werden.
Zu beachten ist, dass in diesen Testfällen nur die CGP-Konfiguration ohne Rekombination mit einer Hyperparameteranalyse optimiert wurden.
Die anderen CGP-Modelle haben diese Parameter übernommen und jeweils eine Rekombinationsrate eingefügt.
Für den Encode-Testfall haben die One-Point und Uniform Rekombination zu sehr guten Ergebnissen geführt, während bei der Two-Point Rekombination alle Tests schlechter ausgefallen sind als bei Tests ohne Rekombination.
Da für den Encode-Testfall jedoch nur die Fitness zur 10000. Iteration betrachtet wurde, kann die Effizienz der Rekombination gegenüber des Trainings nicht sinnvoll ausgewertet werden.
Dadurch, dass die Mutation in den späteren Iterationen zu mehr Trainingserfolgen führt als die Rekombination, kann es passieren, dass bei der Betrachtung der Iterationen bis zur Konvergenz die CGP-Konfiguration ohne Rekombination bessere Ergebnisse liefert, da die Mutation in diesem Verfahren besser funktionieren könnte.
Für den Koza-Testfall haben sehr wenige CGP-Modelle mit Rekombination besser abgeschnitten als diejenigen ohne Rekombination.
Einerseits könnten die CGP-Konfigurationen mit Rekombination durch eine Hyperparameteroptimierung verbessert werden.
Andererseits kann es auch sein, dass komplexere Testszenarien wie Encode und Koza andere Rekombinationsverfahren benötigen als die einfachen Testfälle.
Während die One-Point Rekombination bei den einfachen Testfällen die beste Rekombinationsart zu sein scheint, sind die Ergebnisse der komplexeren Rekombinationsarten bei Koza besser als die der One-Point Rekombination.
Für Encode ergibt die Uniform Rekombination für alle Rekombinationsraten-Arten die durchschnittlich besten Ergebnisse.
Für den Koza-Testfall führt die Two-Point Rekombination für alle Rekombinationsraten-Arten zu durchschnittlich relativ guten Ergebnissen.
Für beide Testfälle (Encode und Koza) werden die durchschnittlich besten Ergebnisse von der Uniform Rekombination mit linear fallender Rekombinationsrate erzeugt.
Da für die konstante Rekombinationsrate die meisten Rekombinationsarten durchschnittlich relativ schlechte Ergebnisse liefern, ist es wahrscheinlich dass die komplexen Testszenarien nicht von der hohen Start-Rekombinationsrate, sondern von der Verkleinerung der Rekombinationsrate profitieren.
Eine mögliche Ursache dafür, dass komplexere Testszenarien im Vergleich zu den einfachen Testfällen vor allem komplexere Rekombinationsarten mit linear fallender Rekombinationsrate bevorzugen, könnte sein, dass die Testfälle insgesamt mehr Iterationen bis zur Konvergenz benötigen als die einfacheren Testfälle Parity und Keijzer.
Bei der bisherigen Belegung der Delta-Rekombinationsrate von 0,0005-0,0055 wird die Rekombination nämlich erst nach 164-1800 Iterationen vollständig abgesetzt.
Für die einfachen Testfälle könnte dies ein zu später Zeitpunkt sein, weshalb die komplexeren Rekombinationsarten (wie Uniform Rekombination) noch in relativ späten Iterationen aktiv sind und das Training durch Mutation wahrscheinlich stärker beeinträchtigen als es die One-Point Rekombination tut.
Um das volle Potential der Rekombination zu nutzen, ist es also wahrscheinlich für alle Testfälle sinnvoll die Uniform Rekombination zu wählen, wobei die Rekombination nur bis zu einer gewissen Iteration ausgeführt wird.
Außerdem wäre ein wichtiger Schritt das Einführen eines neuen CGP-Lernverfahrens, bei dem sich die Rekombination und Mutation nicht mehr gegenseitig negativ beeinflussen können.
Eine Möglichkeit für die Umsetzung könnte sein, Rekombination und Mutation nicht mehr nacheinander auszuführen werden sondern parallel, während beide genetische Operatoren eigene Nachwuchschromosome erzeugen, die anschließend verglichen werden.
Aus den beiden Teilpopulationen können anschließend die Elitisten ausgesucht werden.
\paragraph{Erkenntnisse aus HPDI}
Abschließend werden die Ergebnisse zusammengefasst, die aus der Bewertung der HPDI-Werte folgen.
Bei denjenigen Testszenarien, für die die Iterationen bis zur Konvergenz bewertetet wurden (Parity, Keijzer und Koza) konnte festgestellt werden, dass kleine untere Grenzen der HPDI-Werte zu kleinen HPDI-Umfängen führen.
Umgekehrt wurde ebenfalls festgestellt, dass große untere Grenzen zu großen HPDI-Ranges führen.
Demnach können für CGP-Tests bei denen viele Durchläufe pro CGP-Konfiguration durchgeführt werden, relativ früh Erkenntnisse über ausstehende Durchläufe gewonnen werden.
Wenn mehrere Durchläufe besonders früh konvergieren, werden die anderen wahrscheinlich ähnlich früh konvergieren, da eine kleine Streuung der Ergebnisse zu erwarten ist.
Umgekehrt wird die Streuung größer, wenn die ersten Durchläufe relativ spät konvergieren.
Wichtig dabei ist, dass es sich nicht um Ausreißer handeln darf, da diese bei der 95\%-igen HPDI nicht betrachtet werden.
Des Weiteren kann daraus abgeleitet werden, dass Trainingserfolge besonders in den frühen Iterationen wichtig sind, da späte Konvergenzen und somit späte Trainingserfolge eine größere Streuung aufweisen und damit unsicherer sind.
Da die Rekombination vor allem in den ersten Iterationen einen größeren Trainingsfortschritt erzielen kann, ist es wichtig die Rekombination weiter auszubauen, um die Streuung der Iterationen bis zur Konvergenz zu reduzieren.\\
Für den Encode-Testfall wurde in der Bayes'schen Analyse die Fitness zur 10000. Iteration bewertet.
Die gleiche Analyse ergibt hier, dass es keinen Zusammenhang zwischen den unteren Grenzen der HPDI-Werte zu deren Ranges gibt.
Demnach kann keine Aussage zu der Streuung der Fitness-Werte in einer Iteration getroffen werden, wenn bekannt ist, dass die kleinsten Fitness-Werte der CGP-Modelle besonders groß oder klein sind.


\chapter{Ergebnisse}
\label{Ergebnisse}
Ergebnisse

TODO: Argumentation, dass Offset nicht so viel bringt (damit Kapitel \glqq praktischer Teil\grqq\space darauf verweisen kann)\\
- HPO: mehrere Durchläufe wählen garkeinen Offset\\
- restliche Tests vergleichen zwischen mit Offset und ohne

\section{Ergebnisse: Boolesche Probleme}
\label{sec:ergebnisseBP}

\subsection{Ergebnisse: Parity}
\label{subsec:ergebnisseParity}
\subsubsection{Analyse der Rohdaten}
\label{parity:analyseRohdaten}

\begin{table}[H]
	\centering
	\begin{tabular}{c | c | c | c | c | c | c}
		\begin{turn}{270} \textbf{CGP-Konfigurationen} \end{turn} & \begin{turn}{270} \textbf{Anzahl Rechenknoten} \end{turn} & \begin{turn}{270} \textbf{Populationsgröße} \end{turn} & \begin{turn}{270} \textbf{Start-Rekombinationsrate} \end{turn} & \begin{turn}{270} \textbf{Delta Rekombinationsrate} \end{turn} & \begin{turn}{270} \textbf{Elitisten} \end{turn} & \begin{turn}{270} \textbf{Offset} \end{turn}\\
		\hline
		keine Rekombination & 1000 & 20 & - & - & 18 & -\\
		\hline
		One-Point Konstant kein Offset & 1350 & 52 & 0,30 & - & 16 & - \\
		\hline
		One-Point Konstant mit Offset & 1500 & 50 & 0,8 & - & 18 & 25 \\
		\hline
		One-Point Clegg kein Offset & 1950 & 42 & - & 0,035 & 14 & - \\
		\hline
		One-Point Clegg mit Offset & 1600 & 58 & - & 0,01 & 20 & 25 \\
		\hline
		One-Point One-Fifth kein Offset & 1150 & 46 & 0,55 & - & 6 & - \\
		\hline
		One-Point One-Fifth mit Offset & 1100 & 54 & 0,35 & - & 20 & 30 \\
		\hline
		Two-Point Konstant kein Offset & 1900 & 58 & 0,4 & - & 20 & - \\
		\hline
		Two-Point Konstant mit Offset & 850 & 54 & 0,3 & - & 16 & 15 \\
		\hline
		Two-Point Clegg kein Offset & 750 & 56 & - & 0,02 & 18 & - \\
		\hline
		Two-Point Clegg mit Offset & 1700 & 28 & - & 0,04 & 12 & 20 \\
		\hline
		Two-Point One-Fifth kein Offset & 1800 & 44 & 0,55 & - & 16 & - \\
		\hline
		Two-Point One-Fifth mit Offset & 950 & 58 & 0,7 & - & 10 & 15 \\
		\hline
		Uniform Konstant kein Offset & 1800 & 58 & 0,8 & - & 20 & - \\
		\hline
		Uniform Konstant mit Offset & 1400 & 54 & 0,3 & - & 20 & 45 \\
		\hline
		Uniform Clegg kein Offset & 1600 & 50 & - & 0,015 & 20 & - \\
		\hline
		Uniform Clegg mit Offset & 1500 & 52 & - & 0,04 & 18 & 25 \\
		\hline
		Uniform One-Fifth kein Offset & 650 & 52 & 0,75 & - & 14 & - \\
		\hline
		Uniform One-Fifth mit Offset & 250 & 50 & 0,35 & - & 16 & 40 \\
	\end{tabular}
	\caption{Parity: Ergebnis Hyperparameteranalyse}
	\label{table:parityHPO}
\end{table}

TODO: Tabellen erklären (Spaltennamen, etc)\\
TODO: für HPO beschreiben welche Daten wie ansonsten gefüllt werden, falls leer (Start Rekombinationsrate für Clegg)\\
TODO: beschreiben, dass mehrere Offsets als (ohne Offset gesetzt wurden)

\begin{table}[H]
	\centering
	\begin{tabular}{c | c | c | c | c | c }
		\begin{turn}{270} \textbf{CGP-Konfigurationen} \end{turn} & \begin{turn}{270} \textbf{Anzahl pos. Mutationen} \end{turn} & \begin{turn}{270} \textbf{Anzahl pos. Rekomb.} \end{turn} & \begin{turn}{270} \textbf{Anzahl neg. Mutationen} \end{turn} & \begin{turn}{270} \textbf{Median Iter. pos. Rekomb.} \end{turn} & \begin{turn}{270} \textbf{Median Iter. bis Konv.} \end{turn}\\
		\hline
		keine Rekombination & 133 & 0 & 0 & - & 58,5\\
		\hline
		One-Point Konstant kein Offset & 126 & 8 & 2 & 5,5 & 44\\
		\hline
		One-Point Konstant mit Offset & 136 & 0 & 0 & - & 49\\
		\hline
		One-Point Clegg kein Offset & 123 & 13 & 3 & 4 & 74\\
		\hline
		One-Point Clegg mit Offset & 126 & 0 & 0 & - & 43,5\\
		\hline
		One-Point One-Fifth kein Offset & 116 & 2 & 1 & 2 & 37\\
		\hline
		One-Point One-Fifth mit Offset & 127 & 0 & 0 & - & 27,5\\
		\hline
		Two-Point Konstant kein Offset & 108 & 10 & 2 & 3 & 33,5\\
		\hline
		Two-Point Konstant mit Offset & 129 & 0 & 0 & - & 64\\
		\hline
		Two-Point Clegg kein Offset & 106 & 20 & 8 & 6,5 & 40\\
		\hline
		Two-Point Clegg mit Offset & 125 & 0 & 0 & - & 70,5\\
		\hline
		Two-Point One-Fifth kein Offset & 126 & 4 & 2 & 3 & 59,5\\
		\hline
		Two-Point One-Fifth mit Offset & 125 & 0 & 0 & - & 43\\
		\hline
		Uniform Konstant kein Offset & 85 & 44 & 15 & 14 & 55\\
		\hline
		Uniform Konstant mit Offset & 131 & 0 & 0 & - & 37,5\\
		\hline
		Uniform Clegg kein Offset & 101 & 41 & 15 & 4 & 31\\
		\hline
		Uniform Clegg mit Offset & 130 & 0 & 0 & - & 51,5\\
		\hline
		Uniform One-Fifth kein Offset & 108 & 22 & 12 & 5 & 69,5\\
		\hline
		Uniform One-Fifth mit Offset & 123 & 0 & 0 & - & 54\\
	\end{tabular}
	\caption{Parity: Auswertung der Rohdaten}
	\label{table:parityRohdaten}
\end{table}

\subsubsection{Bayes'sche Analyse}
\label{parity:bayes}
\subsubsection{Graphische Evaluation}
\label{parity: graphische Evaluation}

\subsection{Ergebnisse: Encode}
\label{subsec:ergebnisseEncode}
\subsubsection{Analyse der Rohdaten}
\label{encode:analyseRohdaten}
\subsubsection{Bayes'sche Analyse}
\label{encode:bayes}
\subsubsection{Graphische Evaluation}
\label{encode: graphische Evaluation}


\subsection{Ergebnisse: Decode}
\label{subsec:ergebnisseDecode}
\subsubsection{Analyse der Rohdaten}
\label{decode:analyseRohdaten}
\subsubsection{Bayes'sche Analyse}
\label{decode:bayes}
\subsubsection{Graphische Evaluation}
\label{decode: graphische Evaluation}

TODO: ggf mehr Testprobleme einfügen

\section{Ergebnisse: Symbolische Regression}
\label{sec:ergebnisseSR}

\subsection{Ergebnisse: Keijzer}
\label{subsec:ergebnisseKeijzer}
\subsubsection{Analyse der Rohdaten}
\label{keijzer:analyseRohdaten}

\begin{table}[H]
	\centering
	\begin{tabular}{c | c | c | c | c | c | c}
		\begin{turn}{270} \textbf{CGP-Konfigurationen} \end{turn} & \begin{turn}{270} \textbf{Anzahl Rechenknoten} \end{turn} & \begin{turn}{270} \textbf{Populationsgröße} \end{turn} & \begin{turn}{270} \textbf{Start-Rekombinationsrate} \end{turn} & \begin{turn}{270} \textbf{Delta Rekombinationsrate} \end{turn} & \begin{turn}{270} \textbf{Elitisten} \end{turn} & \begin{turn}{270} \textbf{Offset} \end{turn}\\
		\hline
		keine Rekombination & 1850 & 44 & - & - & 20 & -\\
		\hline
		One-Point Konstant kein Offset & 600 & 50 & 0,2 & - & 20 & - \\
		\hline
		One-Point Konstant mit Offset & 1200 & 50 & 1 & - & 4 & 120 \\
		\hline
		One-Point Clegg kein Offset & 1100 & 60 & - & 0,05 & 16 & - \\
		\hline
		One-Point Clegg mit Offset & 850 & 60 & - & 0,005 & 16 & 300 \\
		\hline
		One-Point One-Fifth kein Offset & 350 & 34 & 0,75 & - & 18 & - \\
		\hline
		One-Point One-Fifth mit Offset & 2000 & 28 & 0,65 & - & 18 & 180 \\
		\hline
		Two-Point Konstant kein Offset & 1700 & 60 & 0,5 & - & 8 & - \\
		\hline
		Two-Point Konstant mit Offset & 600 & 48 & 0,3 & - & 16 & 180 \\
		\hline
		Two-Point Clegg kein Offset & 750 & 34 & - & 0,005 & 14 & - \\
		\hline
		Two-Point Clegg mit Offset & 800 & 36 & - & 0,045 & 20 & 210 \\
		\hline
		Two-Point One-Fifth kein Offset & 1350 & 40 & 0,35 & - & 20 & - \\
		\hline
		Two-Point One-Fifth mit Offset & 700 & 52 & 0,7 & - & 8 & 30 \\
		\hline
		Uniform Konstant kein Offset & 1100 & 52 & 0,8 & - & 8 & - \\
		\hline
		Uniform Konstant mit Offset & 1800 & 60 & 0,1 & - & 20 & 90 \\
		\hline
		Uniform Clegg kein Offset & 850 & 26 & - & 0,05 & 18 & - \\
		\hline
		Uniform Clegg mit Offset & 2000 & 44 & - & 0,05 & 14 & 180 \\
		\hline
		Uniform One-Fifth kein Offset & 1150 & 54 & 0,75 & - & 18 & - \\
		\hline
		Uniform One-Fifth mit Offset & 1850 & 44 & 0 & - & 20 & - \\
	\end{tabular}
	\caption{Keijzer: Ergebnis Hyperparameteranalyse}
	\label{table:keijzerHPO}
\end{table}

TODO: Tabellen erklären (Spaltennamen, etc)\\
TODO: für HPO beschreiben welche Daten wie ansonsten gefüllt werden, falls leer (Start Rekombinationsrate für Clegg)\\
TODO: beschreiben, dass mehrere Offsets als (ohne Offset gesetzt wurden)

\begin{table}[H]
	\centering
	\begin{tabular}{c | c | c | c | c | c }
		\begin{turn}{270} \textbf{CGP-Konfigurationen} \end{turn} & \begin{turn}{270} \textbf{Anzahl pos. Mutationen} \end{turn} & \begin{turn}{270} \textbf{Anzahl pos. Rekomb.} \end{turn} & \begin{turn}{270} \textbf{Anzahl neg. Mutationen} \end{turn} & \begin{turn}{270} \textbf{Median Iter. pos. Rekomb.} \end{turn} & \begin{turn}{270} \textbf{Median Iter. bis Konv.} \end{turn}\\
		\hline
		keine Rekombination & 2260 & 0 & 0 & - & 624\\
		\hline
		One-Point Konstant kein Offset & 1183 & 26 & 16 & 13 & 756\\
		\hline
		One-Point Konstant mit Offset & 1531 & 0 & 0 & - & 374\\
		\hline
		One-Point Clegg kein Offset & 1584 & 59 & 28 & 5 & 243\\
		\hline
		One-Point Clegg mit Offset & 1534 & 0 & 0 & - & 396,5\\
		\hline
		One-Point One-Fifth kein Offset & 1307 & 35 & 18 & 3 & 766\\
		\hline
		One-Point One-Fifth mit Offset & 1776 & 0 & 0 & - & 229\\
		\hline
		Two-Point Konstant kein Offset & 1854 & 79 & 25 & 15 & 803\\
		\hline
		Two-Point Konstant mit Offset & 1361 & 0 & 0 & - & 548\\
		\hline
		Two-Point Clegg kein Offset & 1467 & 106 & 62 & 18,5 & 735\\
		\hline
		Two-Point Clegg mit Offset & 1862 & 0 & 0 & - & 201\\
		\hline
		Two-Point One-Fifth kein Offset & 1788 & 17 & 10 & 4 & 753\\
		\hline
		Two-Point One-Fifth mit Offset & 1124 & 0 & 0 & - & 158\\
		\hline
		Uniform Konstant kein Offset & 1224 & 480 & 239 & 24 & 363\\
		\hline
		Uniform Konstant mit Offset & 1890 & 0 & 0 & - & 260,5\\
		\hline
		Uniform Clegg kein Offset & 1386 & 77 & 57 & 6 & 723\\
		\hline
		Uniform Clegg mit Offset & 2232 & 0 & 0 & - & 385,5\\
		\hline
		Uniform One-Fifth kein Offset & 1491 & 106 & 50 & 8 & 155\\
		\hline
		Uniform One-Fifth mit Offset & 1749 & 0 & 0 & - & 445,5\\
	\end{tabular}
	\caption{Keijzer: Auswertung der Rohdaten}
	\label{table:keijzerRohdaten}
\end{table}

\subsubsection{Bayes'sche Analyse}
\label{keijzer:bayes}
\subsubsection{Graphische Evaluation}
\label{keijzer: graphische Evaluation}


\subsection{Ergebnisse: Koza}
\label{subsec:ergebnisseKoza}
\subsubsection{Analyse der Rohdaten}
\label{koza:analyseRohdaten}
\subsubsection{Bayes'sche Analyse}
\label{koza:bayes}
\subsubsection{Graphische Evaluation}
\label{koza: graphische Evaluation}

TODO: ggf mehr Testprobleme einfügen

\section{Zusammenfassung der Ergebnisse}
\label{sec:zusammenfassungErgebnisse}

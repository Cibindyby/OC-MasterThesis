\section*{Zusammenfassung}

Cartesian Genetic Programming (CGP) kann verwendet werden, um verschiedene Eingang-Ausgang-Beziehungen automatisiert durch Training eines CGP-Modells herzustellen.
Teil davon sind die genetischen Operatoren der Mutation und Rekombination.
Aussagen darüber, dass Rekombination in Standard-CGP die Effektivität des Trainings nicht sinnvoll steigere, basieren auf Papern aus den Jahren 1999 und 2011, werden jedoch in modernen Arbeiten weiterhin aufgegriffen.
Jene Aussagen werden in dieser Arbeit kritisch hinterfragt und experimentell erforscht.
Außerdem werden unterschiedliche Rekombinationsstrategien miteinander verglichen und bewertet, wobei ein besonderer Augenmerk auf der Berechnung der Rekombinationsrate gelegt wird.
Die Ergebnisse in dieser Arbeit zeigen, dass Rekombination in Standard-CGPs durchaus effektiv eingebaut werden kann, wenn diese sinnvoll konfiguriert wird.
Der Fokus sollte dabei auf eine hohe Rekombinationsrate zu Beginn des Trainings gelegt werden, die in späteren Iterationen stark reduziert oder vollkommen abgeschafft wird.
Außerdem wird aufgezeigt, dass Rekombination ein Potential besitzt, das durch gezielte Änderung des standardisierten CGP-Verfahrens effektiver genutzt werden könnte.
Basierend auf den Erkenntnissen dieser Arbeit werden Empfehlungen für zukünftige Forschungsansätze gegeben, die das untersuchte Themengebiet weiter vertiefen und voranbringen können.
